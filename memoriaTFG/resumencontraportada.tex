\fontfamily{ptm}\selectfont
\setlength{\parindent}{1.5em} % Sets paragraph indentation to 1.5em
\setlength{\parskip}{1.2em} % Space between paragraphs
\large

\begin{justify}
{\color{white}\noindent\hspace{1.5em}Este Trabajo Fin de Grado desarrolla diversos programas que complementan el ya existente ``InterCom'', un sistema de comunicación bidireccional que originalmente solo transmitía audio. El principal aporte de este proyecto ha sido implementar la capacidad de transmisión simultánea de vídeo junto con el audio, permitiendo una comunicación multimedia completa en tiempo real. El sistema desarrollado está diseñado para ser eficiente y funcionar en dispositivos con recursos limitados, ofreciendo distintas configuraciones de resolución compatibles con cada cámara y un control de la tasa de fotogramas por segundo (FPS).}

{\color{white}\noindent\hspace{1.5em}El trabajo ha consistido en desarrollar tres módulos principales: ``Minimal\_Video'' como base para la transmisión del vídeo, ``Minimal\_Video\_FPS'' para controlar la tasa de fotogramas por segundo según la resolución, y ``Minimal\_Video\_Resolution'' para reescalar la resolución en caso de incompatibilidad de la resolución solicitada por el usuario. Todo el código se ha implementado proporcionando una integración completa con el programa original ``InterCom'', enfocándose en minimizar la latencia y optimizar el uso de recursos de red, resultando en una herramienta académica que permite investigar y profundizar en las tecnologías de intercomunicación multimedia.}

{\color{white}\noindent\hspace{1.5em}Todo el código fue implementado en Python, manteniendo la compatibilidad con el subsistema de audio preexistente, y la documentación fue redactada utilizando LaTeX para garantizar una presentación profesional y estructurada del trabajo académico.}
\end{justify}