\label{sec:justificacion_y_objetivos}

\subsection{Contextualización}

En el contexto tecnológico contemporáneo, los sistemas de comunicación bidireccional constituyen un pilar fundamental en la interacción personas. 

\vspace{\baselineskip}

La transmisión de voz y vídeo en tiempo real ha experimentado una evolución significativa en las últimas décadas, consolidándose como un elemento esencial tanto en entornos profesionales como educativos. Estos sistemas han adquirido mayor relevancia a partir de la acelerada digitalización, donde la necesidad de mantener comunicaciones efectivas a distancia se ha convertido en una prioridad para organizaciones e individuos.

\vspace{\baselineskip}
El presente Trabajo Fin de Grado está comprendido dentro de la disciplina de las Tecnologías Multimedia, un área que integra conocimientos fundamentales sobre la generación, almacenamiento, tratamiento, transmisión y reproducción de contenidos multimedia, abarcando tanto los aspectos técnicos como conceptuales de audio, vídeo e imagen. En la asignatura se estudian temas relacionados con la percepción audiovisual, la transducción y digitalización de señales, la codificación y compresión de contenido multimedia, el almacenamiento y las tecnologías de interconexión, así como la transmisión de datos sobre redes, poniendo especial énfasis en los protocolos y modelos de transmisión multimedia en tiempo real. 

\vspace{\baselineskip}
Específicamente, este proyecto parte de un sistema preexistente desarrollado en el contexto académico, denominado \textit{InterCom}, que proporciona distintas funcionalidades básicas de comunicación por voz entre dispositivos. Esta base constituye un punto de partida idóneo para la implementación de mejoras significativas que amplíen sus capacidades hacia un sistema de comunicación bidireccional de video que es sobre lo que trata este trabajo.

\subsection{Justificación}


La justificación de este proyecto se fundamenta en diversos factores que demuestran su relevancia académica, tecnológica y social. A continuación, se presenta un análisis detallado de cada uno de estos aspectos:

\subsubsection*{1.2.1 Relevancia académica}

Desde una perspectiva académica, este TFG representa una gran oportunidad para la aplicación práctica e integración de conocimientos adquiridos durante la formación en Ingeniería Informática. 

\vspace{\baselineskip}
La naturaleza multidisciplinar del proyecto requiere la confluencia de diversas áreas como:

\begin{itemize}
    \item \textbf{Programación avanzada en Python}: La implementación de nuevas funcionalidades exige un dominio profundo del lenguaje, especialmente en lo referente a programación concurrente y manejo de eventos.

    
    \item \textbf{Procesamiento de señales digitales}: La manipulación eficiente de flujos de audio y vídeo requiere conocimientos específicos sobre digitalización, compresión y transmisión de señales y paquetes en la red.
    
    \item \textbf{Protocolos de red}: La transmisión fiable y eficiente de datos multimedia en tiempo real exige tener una comprensión profunda de los protocolos de comunicación.
    
    \item \textbf{Virtualización}: El uso de tecnologías como Oracle VirtualBox y Xubuntu requiere conocimientos sobre sistemas operativos y entornos virtualizados.

\end{itemize}
\vspace{\baselineskip}
Esta integración de conocimientos no solo consolida el aprendizaje teórico, sino que fomenta el desarrollo de diversas competencias profesionales que resultan fundamentales en el ámbito laboral actual.

\subsubsection*{1.2.2 Relevancia tecnológica}

La mejora y ampliación de un sistema de comunicación bidireccional es un tema y Herramienta recurrente con las tendencias tecnológicas actuales, donde la comunicación en tiempo real es un campo en constante evolución. 

\vspace{\baselineskip}

Entre los aspectos tecnológicos más relevantes que justifican este proyecto destacan:

\begin{itemize}
    \item \textbf{Optimización de recursos}: La implementación del programa usando la menor cantidad de recursos posibles así como, la menor complejidad computacional y la transmisión de vídeo sin compresion, representa un desafío técnico significativo que contribuye al avance del conocimiento en el área, en concreto de la asignatura en cuestión.
    
    \item \textbf{Escalabilidad}: Diseñar una arquitectura que permita la incorporación de futuras nuevas funcionalidades constituye un ejercicio valioso para la ingeniería de software.
\end{itemize}
\vspace{\baselineskip}
La investigación sobre estos aspectos fundamentales de la programación no solo mejora y enriquece el proyecto, sino que proporciona soluciones potencialmente aplicables a situaciones reales en el desarrollo de software y tecnologías de la comunicación.

\subsubsection*{1.2.3 Relevancia social y laboral}

En el contexto social actual, los sistemas de comunicación bidireccionales han adquirido un papel clave como herramientas de mejora y solucion a distintos problemas de comunicacion entre interacciones de personas en múltiples ámbitos. Esta relevancia se manifiesta en:

\begin{itemize}
    \item \textbf{Entornos educativos}: La educación a distancia con todo lo que esto comprende (videoconferencias, campus virtuales, tutorias online, etc.), se ha consolidado como una forma de impartir la educación ampliamente adoptada sobre todo despues de la pandemia COVID-19, donde las herramientas de comunicación bidireccionales han resultado fundamentales para mantener la Comunicación entre docentes y estudiantes.
    
    \item \textbf{Contextos profesionales}: El teletrabajo y las reuniones online han transformado las dinámicas laborales, incrementando la demanda de soluciones eficientes para la comunicación audiovisual.
    
    \item \textbf{Relaciones interpersonales}: La comunicación familiar y social ya venía siendo usual y cotidiana antes de la pandemia pero despues de esta ha resultado indispensable para muchas personas, especialmente en situaciones de movilidad limitada.
\end{itemize}

\vspace{\baselineskip}

El desarrollo de competencias en este ámbito proporciona una ventaja clara en el mercado laboral, donde la demanda de profesionales capaces de implementar dichas soluciones multimedia sigue siendo una tendencia en aumento.

\subsection{Objetivos del proyecto}

El presente trabajo se realiza en torno a un objetivo general que a su vez se desarrolla en objetivos específicos, proporcionando una estructura clara para el implementación del proyecto.

\subsubsection*{1.3.1 Objetivo general}

El objetivo principal de este TFG consiste en mejorar y ampliar las capacidades de un sistema preexistente de comunicación por voz (\textit{InterCom}), mediante la incorporación de funcionalidades de vídeo y la optimización de los procesos de comunicación, para obtener un sistema integral de comunicación audiovisual bidireccional en tiempo real.

\subsubsection*{1.3.2 Objetivos específicos}

Para conseguirlo, se establecen los siguientes objetivos específicos:

\begin{itemize}
    \item \textbf{Análisis exhaustivo del programa base}: Realizar un estudio detallado de la arquitectura, funcionalidades y limitaciones del sistema \textit{InterCom} actual, identificando partes susceptibles de mejora y de integración de las nuevas funcionalidades.
    
    \item \textbf{Implementación de vídeo}: Desarrollar los sistemas necesarios para la captura, transmisión, recepción y visualización del vídeo en tiempo real, garantizando la compatibilidad con el subsistema de audio preexistente además del desarrollo de diversos módulos adicionales para mejorar/adaptar el programa teniendo en cuenta distintas situaciones posibles de ejecución del mismo.
    
    \item \textbf{Documentación y validación}: Elaborar una documentación detallada que facilite el mantenimiento y mejora futura del sistema así como, la descripción en esta memoria de un conjunto de pruebas que validen su funcionamiento en diversos escenarios de uso además de comentarios descripitivos en el propio código.
\end{itemize}

\subsection{Metodología}

Para la realización de los objetivos planteados, se adoptará una metodología de desarrollo que permita la evaluación del sistema base de audio para luego plantear posibles soluciones y responder a los desafíos técnicos que surjan durante la implementación. El enfoque usado se estructura en las siguientes fases:

\begin{itemize}
    
    \item \textbf{Fase de análisis}: Estudio detallado del sistema base de audio, identificación de los requisitos funcionales y no funcionales, y definición de la arquitectura del sistema ampliado.
    
    \item \textbf{Fase de implementación}: Desarrollo incremental de los componentes, comenzando por la integración del subsistema de vídeo, seguido por la implementación de un módulo sobre el mismo de configuración de \textit{FPS} (Frames per Second) y un módulo final de reescalado a una resolución compatible con la cámara del usuario.
    
    \item \textbf{Fase de pruebas}: Validación unitaria de cada componente implementadod además de pruebas de integración del sistema completo, y evaluación del rendimiento programa en diferentes condiciones de red.
    
    \item \textbf{Fase de documentación}: Elaboración de la memoria técnica, manuales de usuario y documentación del código para facilitar el mantenimiento futuro.
\end{itemize}
\vspace{\baselineskip}
Este enfoque metodológico garantiza un desarrollo coherente y estructurado, permitiendo la identificación de posibles problemas a la hora de la implementación de distintas soluciones adecuadas.

\subsection{Impacto esperado}

La realización exitosa de este TFG conllevará diversos beneficios tanto a nivel académico como práctico, entre los que destacan:

\begin{itemize}

    \item \textbf{Contribución académica}: La documentación del proceso de desarrollo y las soluciones implementadas consituyen un recurso valioso para futuros estudiantes de la propia asignatura Tecnologías Multimedia y de personas interesadas en el ámbito de las comunicaciones multimedia en general.
    
    \item \textbf{Base para implementaciones futuras}: La arquitectura modular planteada facilitará la incorporación de nuevas funcionalidades en trabajos posteriores.
\end{itemize}