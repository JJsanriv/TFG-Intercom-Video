% Paquetes básicos
\usepackage[utf8]{inputenc}
\usepackage{listings}
\usepackage{listingsutf8}
\usepackage[T1]{fontenc}
\usepackage[spanish,es-tabla]{babel}
\usepackage{graphicx}
\usepackage{fancyhdr}
\usepackage{geometry}
\usepackage{setspace}
\usepackage{ragged2e}
\usepackage{tikz}
\usepackage{pgfgantt}
\usepackage{anyfontsize}
\usepackage{xcolor}
\usepackage{hyperref}
\usepackage{titlesec}
\usepackage{eso-pic}
\usepackage{transparent}
\usepackage[bottom]{footmisc}  % Coloca las notas al pie en la parte inferior de la página
\usepackage{caption}
\usepackage{tocloft}
\usepackage{adjustbox}
\usepackage{graphicx}
\usepackage{subcaption}
\usepackage{pgfplots}
\usepackage{float}
\usepackage{amsmath}
\usepackage{makecell}
\usepackage{tabularx}
\usepackage{array}
\usepackage{longtable}

% Cambiar Listings por Bloque de código
\renewcommand{\lstlistingname}{Bloque de código}

% Configuración de los márgenes y la fuente
\geometry{top=25mm, bottom=25mm, left=25mm, right=25mm}
\setlength{\headheight}{1.2cm}
\setlength{\headsep}{20pt}
\renewcommand{\baselinestretch}{1.1}
\setlength{\parindent}{0pt}

\fancypagestyle{titlepage}{
    \fancyhf{}
    \thispagestyle{empty}
}

\definecolor{tfgazul}{RGB}{16, 77, 173}

\fancypagestyle{main}{
    \fancyhf{}
    \fancyhead[R]{\includegraphics[height=1.2cm]{images/logo_ual_peque.png}}
    \renewcommand{\footrulewidth}{0.8pt}
    \renewcommand{\footrule}{%
        \color{tfgazul}%
        \hrule width\headwidth height\footrulewidth \vskip2pt
    }
    \fancyfoot[L]{\small \textit{Mejora y Ampliación de un Sistema de Comunicación Bidireccional por Voz y Vídeo}}
    \fancyfoot[R]{\roman{page}}
    \renewcommand{\headrulewidth}{0pt}
}

\renewcommand{\rmdefault}{ptm}

% Marca de agua global en esquina inferior derecha (semitransparente)
\AddToShipoutPictureBG{%
  \AtPageLowerLeft{%
    \put(\LenToUnit{\paperwidth-9cm},\LenToUnit{-4cm}){%
      \transparent{0.2}\includegraphics[width=15cm,keepaspectratio]{images/logo_agua.png}
    }
  }
}

% Comando para desactivar temporalmente la marca de agua
\newcommand{\sinmarcaagua}{
  \ClearShipoutPictureBG
}

% Comando para reactivar la marca de agua
\newcommand{\conmarcaagua}{
  \AddToShipoutPictureBG{%
    \AtPageLowerLeft{%
      \put(\LenToUnit{\paperwidth-9cm},\LenToUnit{-4cm}){%
        \transparent{0.2}\includegraphics[width=15cm,keepaspectratio]{images/logo_agua.png}
      }
    }
  }
}

\setlength{\skip\footins}{1cm} % Aumenta el espacio entre texto principal y notas al pie
\setlength{\footnotesep}{0.1cm} % Aumenta el espacio entre notas al pie consecutivas

% Estilo para código Python
\definecolor{codebackground}{rgb}{0.95,0.95,0.95}
\definecolor{codekeyword}{rgb}{0.13,0.29,0.53}
\definecolor{codestring}{rgb}{0.63,0.125,0.94}
\definecolor{codecomment}{rgb}{0.0,0.5,0.0}
\definecolor{codeidentifier}{rgb}{0.0,0.0,0.0}

\lstdefinestyle{pythonstyle}{
    language=Python,
    backgroundcolor=\color{codebackground},
    basicstyle=\ttfamily\scriptsize,
    inputencoding=utf8,
    keywordstyle=\color{codekeyword}\bfseries,
    stringstyle=\color{codestring},
    commentstyle=\color{codecomment}\itshape,
    identifierstyle=\color{codeidentifier},
    numbers=left,
    numberstyle=\tiny\color{gray},
    numbersep=8pt,
    breaklines=true,
    breakatwhitespace=false,
    showspaces=false,
    showstringspaces=false,
    showtabs=false,
    tabsize=4,
    captionpos=b,
    frame=single,
    rulecolor=\color{black},
    framerule=0.5pt,
    xleftmargin=3.4pt,
    xrightmargin=3.4pt,
    literate={á}{{\'a}}1 {é}{{\'e}}1 {í}{{\'i}}1 {ó}{{\'o}}1 {ú}{{\'u}}1 {Á}{{\'A}}1 {É}{{\'E}}1 {Í}{{\'I}}1 {Ó}{{\'O}}1 {Ú}{{\'U}}1 {ñ}{{\~n}}1 {Ñ}{{\~N}}1
}

