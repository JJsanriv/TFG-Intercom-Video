\label{sec:resultados_y_discusion}

\subsection{Configuración de hardware}

Los experimentos se realizaron utilizando dos instancias de InterCom ejecutándose en ordenadores físicos distintos. La primera instancia se ejecutó en un portátil equipado con WebCam integrada, mientras que la segunda se ejecutó en un PC de escritorio con WebCam externa. Ambos sistemas utilizan el mismo entorno virtual \textit{Xubuntu} bajo \textit{Oracle VirtualBox}, con configuraciones prácticamente idénticas. La única diferencia significativa es la asignación de memoria RAM: 8 GB para el portátil y 16 GB para el PC de escritorio y sus núcleos, el portátil usará el máximo de núcleos que se le es posible y el PC usará 6 núcleos. Esta diferencia no debería influir en los resultados de las pruebas, ya que ambas configuraciones proporcionan recursos suficientes para la ejecución de los módulos.
\vspace{\baselineskip}

Como se puede ver en la Figura \ref{fig:conf_red}, para establecer la comunicación entre las instancias, se configuró un adaptador de red tipo ``puente'' (a través de la configuración en \textit{Oracle VirtualBox}) en ambas máquinas virtuales. Esta configuración permite que las máquinas virtuales, aun ejecutándose en ordenadores físicos diferentes, se comuniquen directamente a través de la red. Las direcciones IP asignadas fueron 192.168.0.58 para el portátil y 192.168.0.68 para el PC de escritorio.

\begin{figure}[htbp]
    \centering
    \begin{tikzpicture}[
        node distance=3cm,
        computer/.style={rectangle, draw, thick, minimum width=2.5cm, minimum height=1.5cm, align=center},
        vm/.style={rectangle, draw, dashed, minimum width=2cm, minimum height=1cm, align=center, fill=gray!20},
        network/.style={ellipse, draw, thick, minimum width=4cm, minimum height=1.5cm, align=center, fill=blue!10}
    ]
        
        % Ordenadores físicos
        \node[computer] (laptop) {\textbf{Portátil}\\8 GB RAM\\4 Cores};
        \node[computer, right=6cm of laptop] (pc) {\textbf{PC Escritorio}\\16 GB RAM\\6 cores};
        
        % Máquinas virtuales
        \node[vm, below=0.5cm of laptop] (vm1) {VM Xubuntu\\InterCom\\192.168.0.58};
        \node[vm, below=0.5cm of pc] (vm2) {VM Xubuntu\\InterCom\\192.168.0.68};
        
        % Red local
        \node[network, below=2.5cm of laptop, right=1.5cm] (network) {\textbf{Red Local}\\\textbf{(Adaptador Puente)}};
        
        % Conexiones
        \draw[thick] (laptop) -- (vm1);
        \draw[thick] (pc) -- (vm2);
        \draw[thick] (vm1) -- (network);
        \draw[thick] (vm2) -- (network);
        
        % Etiquetas de dispositivos
        \node[below=0.2cm of vm1] {\small WebCam integrada};
        \node[below=0.2cm of vm2] {\small WebCam externa};
        
    \end{tikzpicture}
    \caption{Configuración usada en los experimentos.}
    \label{fig:conf_red}
\end{figure}

Respecto a la presentación de resultados, únicamente se muestran las métricas de una de las instancias debido a que ambas presentan comportamientos prácticamente idénticos bajo las mismas condiciones de red, ambas máquinas con más de 100 Mbps de ancho de banda. Las pequeñas variaciones observadas (típicamente < 5\%) se deben principalmente a diferencias en el hardware de captura (WebCam integrada vs. externa) y no representan diferencias significativas en el rendimiento del protocolo, por lo que supone una comodidad y una manera mas eficiente poner los resultados de una de las instancias para no extender demasiado este documento. Además, todas las ejecuciones de los distintos módulos se han realizado con los parámetros por defecto (\textbf{320x240} y \textbf{30 FPS}), salvo que se indique lo contrario en el comando ejecución que se muestra antes de la realización del experimento, así como también el módulo de \textit{Minimal\_Video\_Resolution} que se indica en cada ejecución la resolución utilizada y FPS concretos (en la mayoría de los casos \textbf{350x250} y \textbf{12 FPS}).
\vspace{\baselineskip}

Como línea de trabajo futuro, sería recomendable implementar la funcionalidad de transmisión de vídeo pregrabado en lugar de captura en tiempo real. Esta mejora permitiría realizar pruebas más controladas y reproducibles, eliminando las variaciones introducidas por las diferencias en los dispositivos de captura y proporcionando una base de comparación más objetiva para evaluar el rendimiento del protocolo bajo diferentes condiciones de red. 


\subsection{Pruebas de robustez ante degradación de red}

Esta parte de la memoria es la más importante, ya que aquí se presentan los resultados obtenidos a partir de la implementación de los códigos ya explicados. En este capítulo, se procederá a simular distintas condiciones reales de red utilizando la herramienta ``tc''\footnote{Para más información sobre el comando, visitar: https://man7.org/linux/man-pages/man8/tc.8.html} de Linux. Se analizará el comportamiento de \textit{InterCom} bajo diferentes condiciones de ancho de banda, retardo y pérdida de paquetes. Finalmente, se extraerán las conclusiones sobre dicho comportamiento del módulo, así como, la robustez y limitaciones del sistema. 
\vspace{\baselineskip}

\textbf{Para la muestra de ejecución de todas las pruebas, se mostrará un frame recibido del Portátil en el que se me verá mover la mano para poder comprobar los efectos de los distintos parámetros que condicionan la red.} 
\vspace{\baselineskip}

A continuación, se comentan los distintos tipos de pruebas que se van realizar en cada subsección de este punto de la memoria:
\begin{itemize}
    \item \textbf{Pruebas de limitación de la tasa de transferencia}: Se simulará un ancho de banda limitado para comprobar el rendimiento del sistema bajo condiciones de red restringidas. Se probarán los siguientes anchos de banda:
    \begin{itemize}
        \item Ancho de banda muy bajo: 1 Mbps (prueba de una situación casi irreal).
        \item Ancho de banda bajo: 10 Mbps (situación común en redes WiFi).
        \item Ancho de banda medio (referencia): 50 Mbps o más (situación común en redes de fibra óptica, suficiente para la correcta ejecución del módulo).
    \end{itemize}
    \item \textbf{Pruebas de incremento del jitter} : Se simulará el efecto de una calidad de red variable, congestión o rutas inestables, observando la variación de la latencia (jitter) para analizar la cantidad de paquetes recibidos se pierden. Se probarán los siguientes escenarios con distintos niveles de latencia, lo que hace que se produzcan distintos grados de jitter:
    \begin{itemize}
      \item Jitter mínimo (Latencia 0 ms): Condiciones ideales, como una red local estable, donde la variación de latencia es despreciable.
      \item Jitter moderado (Latencia 100 ms): Simula una red con cierta inestabilidad (e.g., redes domésticas con tráfico variable, conexiones Wi-Fi con interferencias).
      \item Jitter grave (Latencia 250 ms): Representa condiciones de red muy inestables (e.g., red muy congestionada, enlaces de mala calidad), donde el jitter es alto y la llegada de paquetes es muy errática.
    \end{itemize}
    \item \textbf{Pruebas frente a pérdida de paquetes}: Se simulará la pérdida de paquetes en la red, lo que puede ocurrir en redes saturadas o de mala calidad. Se probarán los siguientes niveles de pérdida: 
    \begin{itemize}
        \item Pérdida baja: 5\% (red local, ideal).
        \item Perdida alta: 25\% (red común, ADSL, fibra, etc.).
        \item Pérdida muy alta: 50\% o más (red muy saturada, mala calidad de red).
    \end{itemize}
\end{itemize}

\newpage

Para la realización y configuración de las distintas situaciones de red, se utilizará el comando ``tc'' de Linux como ya se ha mencionado anteriormente. 
\vspace{\baselineskip}

Antes de nada, habrá que observar si hay alguna regla ya aplicada a la interfaz de red. La interfaz de red dependerá de con quien ejecutemos el módulo, si lo ejecutamos en dos instancias de nuestro propio PC, bastará con usar la interfaz \textit{loopback}. En caso de estar ejecutando el módulo junto con otro dispositivo de nuestra red, como es el caso de estas pruebas, habrá que mirar que interfaz de red saliente está utilizando el dispositivo. Para ello, se puede usar el comando ``ip addr'' para ver las interfaces de red disponibles y sus respectivas direcciones IP. 

\vspace{\baselineskip}
Entonces, una vez mirado la interfaz de red, que en mi caso es \textit{enp0s3}, se procederá a aplicar las reglas de red necesarias para simular las condiciones de red deseadas.

\vspace{\baselineskip}

Primero, habrá que mirar que no hay ninguna regla prestablecida en la interfaz de red. Para ello, se ejecutará el siguiente comando:
\begin{lstlisting}[language=bash]
sudo tc qdisc show dev enp0s3
\end{lstlisting}
Si no hay ninguna regla, el resultado será algo como así:
\begin{lstlisting}[language=bash, breaklines=true]
qdisc fq_codel 0: root refcnt 2 limit 10240p flows 1024 quantum 1514 target 5ms interval 100ms memory_limit 32Mb ecn drop_batch 64 
\end{lstlisting}

En caso de que hubiera ya alguna regla, se eliminaría con el siguiente comando:
\begin{lstlisting}[language=bash]
sudo tc qdisc del dev enp0s3 root
\end{lstlisting}

Ahora, ya se puede aplicar la regla de red deseada. Para ello, se utilizarán los siguientes comandos:

\begin{itemize}
    \item Para limitar el ancho de banda a X Mbps, donde X es el valor deseado:
    \begin{lstlisting}[language=bash, breaklines=true]
    sudo tc qdisc add dev enp0s3 root tbf rate Xmbit burst 1mbit latency 5000ms
\end{lstlisting}
    Donde el valor de \texttt{burst} es el tamaño del paquete máximo que se puede enviar en un momento dado y el valor de \texttt{latency} es el tiempo máximo que se puede esperar para enviar un paquete. En este caso, se ha puesto un valor de 1 Mbit y 5000 ms\footnote{El parámetro \texttt{latency 5000ms} define el tiempo máximo que un paquete puede estar en cola antes de ser descartado. Se usa un valor alto (5 segundos) para evitar que este timeout interfiera con las pruebas de limitación de ancho de banda, asegurando que solo se mida el efecto del parámetro \texttt{rate} que es el que nos interesa.} respectivamente, pero no alterarán el resultado de la prueba.
    \item Para añadir un retardo de X ms, donde X es el valor deseado:
    \begin{lstlisting}[language=bash]
    sudo tc qdisc add dev enp0s3 root netem delay Xms
\end{lstlisting}
    \item Para añadir una pérdida de paquetes de X\%, donde X es el valor deseado:
    \begin{lstlisting}[language=bash]
    sudo tc qdisc add dev enp0s3 root netem loss X%
\end{lstlisting}
\end{itemize}


\newpage

Antes de presentar los resultados detallados de los experimentos, es necesario definir y aclarar algunas de las métricas clave que se utilizarán para la evaluación del rendimiento de los módulos, especialmente en aquellas relacionadas con la tasa de fotogramas y reescalado de resolución. Estas definiciones ayudarán a interpretar los datos presentados en las siguientes secciones y en la Tabla~\ref{tab:resumen_pruebas_globales}.

\begin{itemize}
    \item \textbf{FPS Objetivo:} Corresponde a la tasa de fotogramas por segundo que se intenta alcanzar con la transmisión, configurada explícitamente por el usuario en todos los módulos (\texttt{Minimal\_Video}, \texttt{Minimal\_Video\_FPS} y \texttt{Minimal\_Video\_Resolution}) mediante el parámetro \verb|-z| (por defecto a 30 FPS). En \textit{Minimal\_Video} a la hora de ejecutar, no se muestran ninguna estadística relacionada con los FPS (eso lo hace \textit{Minimal\_Video\_FPS}), debido a que que no se puede establecer realmente un FPS concreto por el usuario por las limitaciones encontradas en la biblioteca OpenCV en este aspecto (ya explicado en anteriores capítulos, concretamente en el 2.2.4). Por lo tanto, se puede considerar que el FPS objetivo es 30 FPS, que es el valor por defecto de la cámara. Para saber concretamente, con todo detalle, lo relacionado con los FPS, se debe ejecutar el módulo \texttt{Minimal\_Video\_FPS} o en su defecto, \texttt{Minimal\_Video\_Resolution} que hereda del anterior.

    \item \textbf{FPS Recibido (por ciclo):} Es el número de fotogramas de vídeo que se han logrado recibir, procesar y mostrar con éxito durante cada ciclo de estadísticas (de 1 segundo de duración). Este valor puede fluctuar debido a las condiciones de la red y la carga del sistema.

    \item \textbf{FPS Real Promedio (FPS R):} Es el promedio de los FPS recibidos durante toda la transmisión. Se calcula como:
    \begin{center}
      \boxed{ \text{FPS}_R = \frac{\sum^N_{i=1}~\text{FPS}_{\text{recibidos},i}}{N}}
    \end{center}
    Donde \(N\) es el número total de ciclos de estadísticas y \(\text{FPS}_{\text{recibidos},i}\) son los FPS contabilizados como recibidos en el ciclo \(i\). El ``FPS Real Promedio'' puede diferir del ``FPS Objetivo'' por diversas razones, entre las que se incluyen:
    \begin{itemize}
        \item Limitaciones del hardware de captura (cámara) o de procesamiento (CPU) que impiden generar o procesar fotogramas a la tasa objetivo.
        \item La carga de trabajo del propio módulo (en el caso de \texttt{Minimal\_Video\_Resolution}, el reescalado).
        \item Las condiciones de la red (ancho de banda insuficiente, alta latencia, jitter o pérdida de paquetes) que pueden impedir la transmisión o recepción completa y a tiempo de todos los fotogramas generados.
    \end{itemize}

    \item \textbf{Eficiencia de FPS (FPS E):} Es el porcentaje que representa el ``FPS Real Promedio'' con respecto al ``FPS Objetivo'' (o FPS ideal) que se configuró para la transmisión. Esta métrica es la que se muestra a partir de los módulos \texttt{Minimal\_Video\_FPS} y \texttt{Minimal\_Video\_Resolution}. Se calcula como:
    \begin{center}
      \boxed{\text{FPS}_E = \frac{\text{FPS}_R}{\text{FPS}\text{~Objetivo}} \times 100}
    \end{center}

    \item \textbf{Tiempo de Reescalado (TR):} Es específico del módulo \texttt{Minimal\_Video\_Resolution}. Representa el tiempo promedio, medido en milisegundos (ms), que el módulo demora en reescalar cada fotograma de vídeo a la resolución deseada.
    \begin{center}
      \boxed{ \text{TR (ms)} = \left( \frac{\sum \text{Tiempo de reescalado por fotograma}}{\text{Número total de fotogramas reescalados}} \right) \times 1000 }
    \end{center}

    \item \textbf{Impacto en Rendimiento (IR):} Es específico del módulo \texttt{Minimal\_Video\_Resolution}. Esta métrica calcula el porcentaje del tiempo ideal disponible por fotograma (según el \texttt{FPS Objetivo}) que se consume en la operación de reescalado del vídeo. Un valor alto podría indicar que el proceso de reescalado es un cuello de botella significativo para alcanzar la tasa de fotogramas objetivo. Se calcula como:
    \begin{center}
      \boxed{\text{IR} = \frac{\text{TR (ms)} \times \text{FPS}\text{~Objetivo}}{1000} \times 100}
    \end{center}
    Donde \(\text{TR}\) es el Tiempo de Reescalado promedio en milisegundos y \(\text{FPS}\text{~Objetivo}\) es la tasa de fotogramas por segundo objetivo.
\end{itemize}

\newpage

\subsubsection{Experimentos sin limitaciones artificiales}

Antes de nada, se mostrará en la Figura~\ref{fig:ejecucion_doble} una ejecución normal de cada módulo implementado para que se observe el funcionamiento y su salida sin ningún tipo de alteración en cada uno de ellos. 
\vspace{\baselineskip}

Empezaremos por un ejemplo normal de ejecución del módulo \textit{Minimal\_Video} sin ningún tipo de restricción de red.
\begin{figure}
  \centering
  \begin{subfigure}{\textwidth}
    \centering
    \includegraphics[width=\textwidth,height=0.6\textheight,keepaspectratio]{images/pruebas/ejecuion_normal1.png}
  \end{subfigure}
  \vspace{\baselineskip}
  \begin{subfigure}{\textwidth}
    \centering
    \includegraphics[width=\textwidth,height=0.6\textheight,keepaspectratio]{images/pruebas/ejecuion_normal2.png}
  \end{subfigure}
  \caption{Ejemplo de ejecución común del módulo \textit{Minimal\_Video}.}
  \label{fig:ejecucion_doble}
\end{figure}
\vspace{\baselineskip}

\newpage

Tenemos en primer lugar la inicialización de la biblioteca \texttt{pygame}~\cite{pygame} y el mensaje de bienvenida de la comunidad. A continuación, se muestra el mensaje de inicio del módulo en modo ``verbose'', que es el modo que se ha elegido para esta prueba. En este modo, se muestran estadísticas de la red y del módulo en tiempo real.
\begin{lstlisting}[language=bash,basicstyle=\ttfamily\scriptsize]
pygame 2.6.0 (SDL 2.28.4, Python 3.12.3)
Hello from the pygame community. https://www.pygame.org/contribute.html
(WARNING) minimal: Unable to import argcomplete (optional)
Starting in Verbose mode...
\end{lstlisting}

Seguidamente, se pueden observar un mensaje de que se ha inicializado el vídeo ya que se activó el flag \verb|--show_video| y se ha intentado inicializar la cámara con el índice 1. En este caso, se ha utilizado la cámara del portátil, que es la cámara 0. Si no se encuentra la cámara, se intentará inicializar la cámara con el índice 1. Luego, se ejecuta \textit{Minimal} que es el módulo base que ejecuta el audio el cual proporciona una descripción de distintos parámetros como el chunk\_time, el seconds\_per\_cycle, el chunks\_per\_cycle y el frames\_per\_cycle. Finalmente se indica que se ha iniciado el modo ``verbose''.

\begin{lstlisting}[language=bash,basicstyle=\ttfamily\scriptsize]
Flag --show_video detected. Attempting to initialize camera with index 1...
(INFO) minimal: A minimal InterCom (no compression, no quantization, no transform, 
... only provides a bidirectional (full-duplex) transmission of raw (playable) chunks. 
(INFO) minimal: chunk_time = 0.023219954648526078 seconds
(INFO) minimal: seconds_per_cycle = 1
(INFO) minimal: chunks_per_cycle = 43.06640625
(INFO) minimal: frames_per_cycle = 44100
Verbose Mode: stats cycle = 1s
Starting video with unified loop and simplified protocol (verbose)...
Press Ctrl+C to terminate
\end{lstlisting}

A continuación, se muestra los parámetros inicializados del módulo, así como, los distintos dispositivos que esta usando el sistema.
\begin{lstlisting}[language=bash,basicstyle=\ttfamily\scriptsize]
InterCom parameters:

Namespace(input_device=None, output_device=None, list_devices=False, 
frames_per_second=44100, frames_per_chunk=1024, listening_port=4444, 
destination_address='localhost', destination_port=4444, filename=None, 
reading_time=None, number_of_channels=2, show_stats=True, 
show_samples=False, show_spectrum=False, video_payload_size=1400, 
width=320, height=240, fps=30, show_video=True, listening_video_port=4445, 
destination_video_port=4445, camera_index=1)

Using device:

   0 Intel 82801AA-ICH: - (hw:0,0), ALSA (2 in, 2 out)
   1 Intel 82801AA-ICH: MIC ADC (hw:0,1), ALSA (2 in, 0 out)
   2 Loopback: PCM (hw:1,0), ALSA (32 in, 32 out)
   3 Loopback: PCM (hw:1,1), ALSA (32 in, 32 out)
   4 sysdefault, ALSA (128 in, 128 out)
   5 front, ALSA (0 in, 2 out)
   6 lavrate, ALSA (128 in, 128 out)
   7 samplerate, ALSA (128 in, 128 out)
   8 speexrate, ALSA (128 in, 128 out)
   9 pulse, ALSA (32 in, 32 out)
  10 speex, ALSA (1 in, 1 out)
  11 upmix, ALSA (8 in, 8 out)
  12 vdownmix, ALSA (6 in, 6 out)
  13 dmix, ALSA (0 in, 2 out)
* 14 default, ALSA (32 in, 32 out)
\end{lstlisting}

Continuamos con las estadísticas de la red en cuanto al vídeo y el audio se refiere. En este caso, se muestra el número de mensajes enviados y recibidos, así como, el ancho de banda utilizado en kbps. También se muestra el uso de CPU del módulo y del sistema. En este caso, el uso de CPU del módulo es bastante alto, ya que se está utilizando la cámara y el procesamiento de vídeo en tiempo real. 
\begin{lstlisting}[language=bash,basicstyle=\ttfamily\tiny]
         |  AUDIO (msg)  |  VIDEO (msg)  |  AUDIO (kbps)   |  VIDEO (kbps)   |     CPU (%) 
   Cycle |  Sent  Recv   |  Sent  Recv   |   Sent   Recv   |   Sent   Recv   | Program System
================================================================================================
   Cycle |  Sent  Recv   |  Sent  Recv   |   Sent   Recv   |   Sent   Recv   | Program System
       1 |   10    10    |    3     3    |   303    303    |    31     31    |  33    100       
       2 |   34    34    |  152   152    |  1064   1064    |  1625   1625    |  41     66       
       3 |   40    40    |    0     0    |  1301   1301    |     0      0    |  42     65       
       4 |   44    44    |    0     0    |  1425   1425    |     0      0    |  34     68       
       5 |   42    42    |    0     0    |  1369   1369    |     0      0    |  50     72       
       6 |   43    43    |    0     0    |  1404   1404    |     0      0    |  46     71       
       7 |   40    40    |    0     0    |  1306   1306    |     0      0    |  42     68       
       8 |   39    39    |    0     0    |  1268   1268    |     0      0    |  47     72       
       9 |   38    38    |    0     0    |  1243   1243    |     0      0    |  47     75       
      10 |   40    40    |  597   596    |  1304   1304    |  6648   6637    |  41     73       
      11 |   27    27    | 3035  3036    |   879    879    | 33765  33776    |  48     70       
      12 |   39    39    | 2923  2923    |  1235   1235    | 31610  31610    |  35     75       
      13 |   27    27    | 2898  2897    |   849    849    | 31143  31132    |  54     69       
      14 |   33    33    | 2382  2383    |  1069   1069    | 26347  26358    |  48     74       
      15 |   22    22    | 1815  1815    |   715    715    | 20146  20146    |  32     71       
      16 |   32    32    | 3305  3305    |  1047   1047    | 36934  36934    |  51     74       
      17 |   35    35    | 3460  3460    |  1133   1133    | 38252  38252    |  63     74       
      18 |   27    27    | 3135  3135    |   880    880    | 34909  34909    |  50     70       
   Cycle |  Sent  Recv   |  Sent  Recv   |   Sent   Recv   |   Sent   Recv   | Program System
         |  AUDIO (msg)  |  VIDEO (msg)  |  AUDIO (kbps)   |  VIDEO (kbps)   |     CPU (%) 
===========================================================================================
\end{lstlisting}

Finalmente, se muestra el mensaje de que se ha detectado una interrupción por teclado (Ctrl+C) y se detiene la aplicación de vídeo. A continuación, se muestran las estadísticas globales de ancho de banda, donde se puede observar el ancho de banda utilizado para el audio y el vídeo, así como el tiempo total de ejecución del módulo. Finalmente, se muestra un mensaje de que el módulo ha terminado.
\begin{lstlisting}[language=bash,basicstyle=\ttfamily\scriptsize]
Video application stopped.

=== Global bandwidth statistics ===
Audio sent:       1082.76 kbps
Audio received:   1082.76 kbps
Video sent:       14317.85 kbps
Video received:   14317.85 kbps
Total time:       18.5 s
=====================================
Program terminated.
QObject::killTimer: Timers cannot be stopped from another thread
QObject::~QObject: Timers cannot be stopped from another thread
\end{lstlisting}
\vspace{\baselineskip}

A continuación, un ejemplo de ejecución común de \textit{Minimal\_Video\_FPS} en el que el módulo tratará de cambiar los FPS que vienen por defecto en la cámara (normalmente 30) por el que el usuario solicite por parámetro:
\begin{center}
	\includegraphics[width = 0.7\textwidth]{images/pruebas/ejecuion_normal_fps.png}
	\captionof{figure}{Ejemplo de ejecución común del módulo \textit{Minimal\_Video\_FPS}.}
	\label{fig:ejecucion_fps}
\end{center}
\vspace{\baselineskip}

\begin{lstlisting}[language=bash,basicstyle=\ttfamily\tiny]
(TM) jj@jj-XubuntuPC:~/TFG-Intercom-Video/src$ python minimal_video_fps.py --show_video 
--show_stats --camera_index 1 -z 12

         |  AUDIO (msg)  |  VIDEO (msg)  |  AUDIO (kbps)   |  VIDEO (kbps)   |     CPU (%) 
   Cycle |  Sent  Recv   |  Sent  Recv   |   Sent   Recv   |   Sent   Recv   | Program System
================================================================================================
       1 |   31    31    |  165   165    |  1012   1012    |  1840   1840    |  13      0       
       2 |   31    31    |  825   825    |  1000   1000    |  9088   9088    |  44     76       
       3 |   35    35    | 1980  1980    |  1145   1145    | 22124  22124    |  45     63       
       4 |   35    35    | 1833  1833    |  1126   1126    | 20144  20144    |  46     61       
       5 |   20    20    | 1137  1137    |   454    454    |  8815   8815    |  41     22       
       6 |   37    37    | 1650  1650    |  1193   1193    | 18175  18175    |  45     70       
       7 |   35    35    | 1980  1980    |  1144   1144    | 22098  22098    |  43     70       
       8 |   33    33    | 1980  1980    |  1079   1079    | 22122  22122    |  49     59       
       9 |   31    31    | 1815  1815    |  1013   1013    | 20253  20253    |  54     71       
      10 |   29    29    | 1980  1980    |   948    948    | 22105  22105    |  48     69       
      11 |   35    35    | 1950  1949    |  1144   1144    | 21775  21764    |  46     70       
      12 |   24    24    | 1845  1846    |   785    785    | 20617  20628    |  46     66       
      13 |   31    31    | 1980  1980    |  1011   1011    | 22063  22063    |  44     67       
      14 |    0     0    |  165   165    |     0      0    |  1843   1843    |   1     19       
   Cycle |  Sent  Recv   |  Sent  Recv   |   Sent   Recv   |   Sent   Recv   | Program System
         |  AUDIO (msg)  |  VIDEO (msg)  |  AUDIO (kbps)   |  VIDEO (kbps)   |     CPU (%) 
===========================================================================================
Video application stopped.

=== Global bandwidth statistics ===
Audio sent:       904.63 kbps
Audio received:   904.63 kbps
Video sent:       16151.39 kbps
Video received:   16151.39 kbps
Total time:       14.7 s
=====================================

=== FPS Statistics ===
Target FPS:       12.0
Average real FPS: 11.8
FPS efficiency:   98.7%
======================
Program terminated.
QObject::killTimer: Timers cannot be stopped from another thread
QObject::~QObject: Timers cannot be stopped from another thread
\end{lstlisting}

\newpage

Finalmente, un ejemplo de ejecución típico de \textit{Minimal\_Video\_Resolution} en el que, por ejemplo en este caso, se le pasa por parámetro que se quiere cambiar de la resolución por defecto del módulo (320x240) a 640x480:
\begin{center}
	\includegraphics[width = 0.7\textwidth]{images/pruebas/ejecuion_normal_resolution.png}
	\captionof{figure}{Ejemplo de ejecución común del módulo \textit{Minimal\_Video\_Resolution}.}
	\label{fig:ejecucion_resolution}
\end{center}
\vspace{\baselineskip}

\begin{lstlisting}[language=bash,basicstyle=\ttfamily\tiny]
(TM) jj@jj-XubuntuPC:~/TFG-Intercom-Video/src$ python minimal_video_resolution.py --show_video
--show_stats --camera_index 1 -z 12 -w 640 -g 480

         |  AUDIO (msg)  |  VIDEO (msg)  |  AUDIO (kbps)   |  VIDEO (kbps)   |     CPU (%) 
   Cycle |  Sent  Recv   |  Sent  Recv   |   Sent   Recv   |   Sent   Recv   | Program System
================================================================================================
       1 |   25    25    |  660   660    |   817    817    |  7369   7369    |  44      0       
       2 |   33    33    |    0     0    |  1079   1079    |     0      0    |  47     70       
       3 |   33    33    |    0     0    |  1077   1077    |     0      0    |  45     73       
       4 |   24    24    | 2327  2327    |   786    786    | 26022  26022    |  36     69       
       5 |   28    28    | 2972  2972    |   891    891    | 32309  32309    |  35     68       
       6 |   31    31    | 3201  3201    |   999    999    | 35234  35234    |  38     71       
       7 |   17    17    | 3203  3202    |   547    547    | 35250  35239    |  38     70       
       8 |   33    33    | 5376  5376    |  1074   1074    | 59747  59747    |  55     74       
       9 |   34    34    | 5473  5473    |  1109   1109    | 60989  60989    |  55     73       
      10 |   34    34    | 5087  5087    |  1110   1110    | 56718  56718    |  62     72       
      11 |   29    29    | 5689  5689    |   937    937    | 62808  62808    |  60     69       
      12 |   35    35    | 4748  4748    |  1142   1142    | 52914  52914    |  64     73       
      13 |   31    31    | 5399  5399    |  1014   1014    | 60315  60315    |  68     74       
      14 |   26    26    | 4141  4140    |   849    849    | 46197  46186    |  53     66       
      15 |   29    29    | 5402  5402    |   945    945    | 60144  60144    |  56     71       
      16 |    0     0    |  357   358    |     0      0    |  3976   3987    |   0     14       
   Cycle |  Sent  Recv   |  Sent  Recv   |   Sent   Recv   |   Sent   Recv   | Program System
         |  AUDIO (msg)  |  VIDEO (msg)  |  AUDIO (kbps)   |  VIDEO (kbps)   |     CPU (%) 
===========================================================================================
Video application stopped.

=== Global bandwidth statistics ===
Audio sent:       881.25 kbps
Audio received:   881.25 kbps
Video sent:       36779.77 kbps
Video received:   36779.09 kbps
Total time:       16.4 s
=====================================
\end{lstlisting}

\begin{lstlisting}[language=bash,basicstyle=\ttfamily\scriptsize]
=== FPS Statistics ===
Target FPS:       12.0
Average real FPS: 7.4
FPS efficiency:   61.8%
======================

=== Camera compatible resolutions ===
  1. 320x240
  2. 352x288
  3. 640x360
  4. 640x480 * SELECTED
  5. 800x600
  6. 1024x768
  7. 1280x720
  8. 1280x1024
  9. 1366x768
  10. 1600x900
  11. 1920x1080
  12. 2560x1440
  13. 3840x2160

Camera device: /dev/video0
======================================
Program terminated.
QObject::killTimer: Timers cannot be stopped from another thread
QObject::~QObject: Timers cannot be stopped from another thread
\end{lstlisting}
\vspace{\baselineskip}

\newpage

\subsubsection{Experimentos con limitación de la tasa de transferencia de datos}

En esta sección se presentan los resultados obtenidos al ejecutar los módulos bajo diferentes condiciones de ancho de banda. 
\vspace{\baselineskip}

\begin{itemize}
    \item \textbf{Pruebas de limitación de la tasa de transferencia}:
\end{itemize}

\textbf{Pruebas de limitación de la tasa de transferencia a 1 Mbps}
\vspace{\baselineskip}

Empezaremos por probar el módulo con un ancho de banda muy bajo, 1 Mbps. El comando usado para \textit{Minimal\_Video} ha sido el siguiente:

\begin{lstlisting}[language=bash]
python minimal_video.py -a 192.168.0.58 --show_video --show_stats
\end{lstlisting}
Donde \verb|-a| es la dirección IP del dispositivo con el que se va a comunicar el módulo, \verb|--show_video| es la opción para mostrar el vídeo en tiempo real y \verb|--show_stats| es la opción para mostrar las estadísticas de la red.
\vspace{\baselineskip}

\begin{lstlisting}[language=bash,basicstyle=\ttfamily\tiny]
         |  AUDIO (msg)  |  VIDEO (msg)  |  AUDIO (kbps)   |  VIDEO (kbps)   |     CPU (%) 
   Cycle |  Sent  Recv   |  Sent  Recv   |   Sent   Recv   |   Sent   Recv   | Program System
================================================================================================
       1 |   27     6    |  165   146    |   881    195    |  1838   1628    |  36      0       
       2 |   33    28    |    0     0    |  1073    910    |     0      0    |  41     75       
       3 |   37     1    |    0     0    |  1206     32    |     0      0    |  51     74       
   Cycle |  Sent  Recv   |  Sent  Recv   |   Sent   Recv   |   Sent   Recv   | Program System
         |  AUDIO (msg)  |  VIDEO (msg)  |  AUDIO (kbps)   |  VIDEO (kbps)   |     CPU (%) 
===========================================================================================
\end{lstlisting}


\begin{lstlisting}[language=bash,basicstyle=\ttfamily\tiny]
Socket blocked while sending fragment 20.
Socket blocked while sending fragment 21.
Socket blocked while sending fragment 22.
Socket blocked while sending fragment 23.
Socket blocked while sending fragment 24.
Socket blocked while sending fragment 25.
Socket blocked while sending fragment 26.
Socket blocked while sending fragment 27.
Socket blocked while sending fragment 28.
Socket blocked while sending fragment 29.
Socket blocked while sending fragment 30.
Socket blocked while sending fragment 31.
Socket blocked while sending fragment 32.
Socket blocked while sending fragment 33.
Socket blocked while sending fragment 34.
Socket blocked while sending fragment 35.
Socket blocked while sending fragment 36.
Socket blocked while sending fragment 37.
Socket blocked while sending fragment 38.
Socket blocked while sending fragment 39.
Socket blocked while sending fragment 40.
Socket blocked while sending fragment 41.
Socket blocked while sending fragment 42.
Socket blocked while sending fragment 43.
Socket blocked while sending fragment 44.
Socket blocked while sending fragment 45.

       4 |   32     3    |  275   187    |  1044     97    |  3065   2081    |  43     73       
   Cycle |  Sent  Recv   |  Sent  Recv   |   Sent   Recv   |   Sent   Recv   | Program System
         |  AUDIO (msg)  |  VIDEO (msg)  |  AUDIO (kbps)   |  VIDEO (kbps)   |     CPU (%) 
===========================================================================================
Socket blocked while sending fragment 110.
Socket blocked while sending fragment 111.
Socket blocked while sending fragment 112.
Socket blocked while sending fragment 113.
Socket blocked while sending fragment 114.
Socket blocked while sending fragment 115.
Socket blocked while sending fragment 116.
Socket blocked while sending fragment 117.
Socket blocked while sending fragment 118.
Socket blocked while sending fragment 119.
Socket blocked while sending fragment 120.
Socket blocked while sending fragment 121.
Socket blocked while sending fragment 122.
Socket blocked while sending fragment 123.
Socket blocked while sending fragment 124.
Socket blocked while sending fragment 125.
Socket blocked while sending fragment 126.
Socket blocked while sending fragment 127.
Socket blocked while sending fragment 128.
Socket blocked while sending fragment 129.
Socket blocked while sending fragment 130.
Socket blocked while sending fragment 131.
Socket blocked while sending fragment 132.
Socket blocked while sending fragment 133.
Socket blocked while sending fragment 134.
Socket blocked while sending fragment 135.
Socket blocked while sending fragment 136.
Socket blocked while sending fragment 137.
Socket blocked while sending fragment 138.
Socket blocked while sending fragment 139.
Socket blocked while sending fragment 140.
Socket blocked while sending fragment 141.
Socket blocked while sending fragment 142.
Socket blocked while sending fragment 143.
Socket blocked while sending fragment 144.
Socket blocked while sending fragment 145.
Socket blocked while sending fragment 146.
Socket blocked while sending fragment 147.
Socket blocked while sending fragment 148.
Socket blocked while sending fragment 149.
Socket blocked while sending fragment 150.
Socket blocked while sending fragment 151.
Socket blocked while sending fragment 152.
Socket blocked while sending fragment 153.
Socket blocked while sending fragment 154.
Socket blocked while sending fragment 155.
Socket blocked while sending fragment 156.
Socket blocked while sending fragment 157.
Socket blocked while sending fragment 158.
Socket blocked while sending fragment 159.
Socket blocked while sending fragment 160.
Socket blocked while sending fragment 161.
Socket blocked while sending fragment 162.
Socket blocked while sending fragment 163.
Socket blocked while sending fragment 164.

       5 |   29    21    |  274    17    |   946    685    |  3051    189    |  27     76       
   Cycle |  Sent  Recv   |  Sent  Recv   |   Sent   Recv   |   Sent   Recv   | Program System
         |  AUDIO (msg)  |  VIDEO (msg)  |  AUDIO (kbps)   |  VIDEO (kbps)   |     CPU (%) 
===========================================================================================
Socket blocked while sending fragment 42.
Socket blocked while sending fragment 43.
Socket blocked while sending fragment 44.
Socket blocked while sending fragment 45.
Socket blocked while sending fragment 46.
Socket blocked while sending fragment 47.
Socket blocked while sending fragment 48.
Socket blocked while sending fragment 49.
Socket blocked while sending fragment 50.
Socket blocked while sending fragment 51.
Socket blocked while sending fragment 52.
Socket blocked while sending fragment 53.
Socket blocked while sending fragment 54.
Socket blocked while sending fragment 55.
Socket blocked while sending fragment 56.
Socket blocked while sending fragment 57.
Socket blocked while sending fragment 58.
Socket blocked while sending fragment 59.
Socket blocked while sending fragment 60.
Socket blocked while sending fragment 61.
Socket blocked while sending fragment 62.
Socket blocked while sending fragment 63.
Socket blocked while sending fragment 64.
Socket blocked while sending fragment 65.
Socket blocked while sending fragment 66.
Socket blocked while sending fragment 67.
Socket blocked while sending fragment 68.
Socket blocked while sending fragment 69.
Socket blocked while sending fragment 70.
Socket blocked while sending fragment 71.
Socket blocked while sending fragment 72.
Socket blocked while sending fragment 73.
Socket blocked while sending fragment 74.
Socket blocked while sending fragment 75.
Socket blocked while sending fragment 76.
Socket blocked while sending fragment 77.
Socket blocked while sending fragment 78.
Socket blocked while sending fragment 79.
Socket blocked while sending fragment 80.
Socket blocked while sending fragment 81.
Socket blocked while sending fragment 82.
Socket blocked while sending fragment 83.
Socket blocked while sending fragment 84.
Socket blocked while sending fragment 85.
Socket blocked while sending fragment 86.
Socket blocked while sending fragment 87.

       6 |   24     7    |  276    76    |   785    228    |  3081    847    |  24     66       
   Cycle |  Sent  Recv   |  Sent  Recv   |   Sent   Recv   |   Sent   Recv   | Program System
         |  AUDIO (msg)  |  VIDEO (msg)  |  AUDIO (kbps)   |  VIDEO (kbps)   |     CPU (%) 
===========================================================================================
Video application stopped.

=== Global bandwidth statistics ===
Audio sent:       954.02 kbps
Audio received:   345.96 kbps
Video sent:       1771.67 kbps
Video received:   762.21 kbps
Total time:       6.3 s
=====================================
Program terminated.
QObject::killTimer: Timers cannot be stopped from another thread
QObject::~QObject: Timers cannot be stopped from another thread
\end{lstlisting}
\vspace{\baselineskip}

La imagen de la Figura \ref{fig:minimal_video_1mbps} es una captura del vídeo recibido en esta prueba.
\begin{center}
  \includegraphics[width = 0.7\textwidth]{images/VideoRecibido1.1.png}
  \captionof{figure}{Vídeo recibido en \textit{Minimal\_Video} para ancho de banda de 1 Mbps.}
  \label{fig:minimal_video_1mbps}
\end{center}

\newpage

\vspace{\baselineskip}

Ahora, ejecutaremos el módulo \textit{Minimal\_Video\_FPS} con el mismo ancho de banda, 1 Mbps. El comando usado ha sido el siguiente:

\begin{lstlisting}[language=bash, basicstyle=\ttfamily\scriptsize]
    python minimal_video_fps.py -a 192.168.0.58 --show_video --show_stats -z 12
\end{lstlisting}

Donde \verb|-a| es la dirección IP destino, \verb|--show_video| es para que se active la transmisión de vídeo, \verb|--show_stats| es para que se muestre la información de estadísticas y \verb|-z| es el numero de fotogramas por segundo (FPS) que se desea enviar. En este caso, se ha configurado para mostrar el vídeo a 12 FPS.

A continuación, se probará el módulo con un ancho de banda muy bajo, de 1 Mbps. El resultado obtenido ha sido el siguiente:
\vspace{\baselineskip}

\begin{lstlisting}[language=bash,basicstyle=\ttfamily\tiny]
         |  AUDIO (msg)  |  VIDEO (msg)  |  AUDIO (kbps)   |  VIDEO (kbps)   |     CPU (%) 
   Cycle |  Sent  Recv   |  Sent  Recv   |   Sent   Recv   |   Sent   Recv   | Program System
================================================================================================
       1 |   25    25    |  165    92    |   810    810    |  1826   1019    |  21    100       
       2 |   36     7    |    0     0    |  1141    221    |     0      0    |  43     76       
       3 |   37    25    |    0     0    |  1208    816    |     0      0    |  44     75       
       4 |   20     0    |  330   157    |   653      0    |  3681   1751    |  21     69       
   Cycle |  Sent  Recv   |  Sent  Recv   |   Sent   Recv   |   Sent   Recv   | Program System
         |  AUDIO (msg)  |  VIDEO (msg)  |  AUDIO (kbps)   |  VIDEO (kbps)   |     CPU (%) 
===========================================================================================
\end{lstlisting}

\begin{lstlisting}[language=bash,basicstyle=\ttfamily\tiny]
Socket blocked while sending fragment 43.
Socket blocked while sending fragment 44.
Socket blocked while sending fragment 45.
Socket blocked while sending fragment 49.
Socket blocked while sending fragment 50.
Socket blocked while sending fragment 52.
Socket blocked while sending fragment 53.
Socket blocked while sending fragment 54.
Socket blocked while sending fragment 55.
Socket blocked while sending fragment 56.
Socket blocked while sending fragment 58.
Socket blocked while sending fragment 61.
Socket blocked while sending fragment 62.
Socket blocked while sending fragment 63.
Socket blocked while sending fragment 64.
Socket blocked while sending fragment 65.
Socket blocked while sending fragment 68.
Socket blocked while sending fragment 69.
Socket blocked while sending fragment 71.
Socket blocked while sending fragment 74.
Socket blocked while sending fragment 75.
Socket blocked while sending fragment 79.
Socket blocked while sending fragment 80.
Socket blocked while sending fragment 81.
Socket blocked while sending fragment 82.
Socket blocked while sending fragment 83.
Socket blocked while sending fragment 84.
Socket blocked while sending fragment 85.
Socket blocked while sending fragment 86.
Socket blocked while sending fragment 89.
Socket blocked while sending fragment 90.
Socket blocked while sending fragment 91.
Socket blocked while sending fragment 93.
Socket blocked while sending fragment 94.
Socket blocked while sending fragment 95.
Socket blocked while sending fragment 97.
Socket blocked while sending fragment 98.
Socket blocked while sending fragment 99.

       5 |   31     6    |  323    87    |  1015    196    |  3613    971    |  33     68       
   Cycle |  Sent  Recv   |  Sent  Recv   |   Sent   Recv   |   Sent   Recv   | Program System
         |  AUDIO (msg)  |  VIDEO (msg)  |  AUDIO (kbps)   |  VIDEO (kbps)   |     CPU (%) 
===========================================================================================
Socket blocked while sending fragment 61.
Socket blocked while sending fragment 62.
Socket blocked while sending fragment 63.
Socket blocked while sending fragment 64.
Socket blocked while sending fragment 65.
Socket blocked while sending fragment 66.
Socket blocked while sending fragment 69.
Socket blocked while sending fragment 71.
Socket blocked while sending fragment 73.
Socket blocked while sending fragment 74.
Socket blocked while sending fragment 75.
Socket blocked while sending fragment 76.
Socket blocked while sending fragment 77.
Socket blocked while sending fragment 78.
Socket blocked while sending fragment 80.
Socket blocked while sending fragment 81.
Socket blocked while sending fragment 83.
Socket blocked while sending fragment 84.
Socket blocked while sending fragment 86.
Socket blocked while sending fragment 87.
Socket blocked while sending fragment 88.
Socket blocked while sending fragment 89.
Socket blocked while sending fragment 90.
Socket blocked while sending fragment 91.
Socket blocked while sending fragment 93.
Socket blocked while sending fragment 94.
Socket blocked while sending fragment 95.
Socket blocked while sending fragment 96.
Socket blocked while sending fragment 99.
Socket blocked while sending fragment 100.
Socket blocked while sending fragment 101.
Socket blocked while sending fragment 102.
Socket blocked while sending fragment 103.
Socket blocked while sending fragment 104.
Socket blocked while sending fragment 106.
Socket blocked while sending fragment 108.
Socket blocked while sending fragment 109.
Socket blocked while sending fragment 111.
Socket blocked while sending fragment 113.
Socket blocked while sending fragment 114.
Socket blocked while sending fragment 115.
Socket blocked while sending fragment 116.
Socket blocked while sending fragment 118.
Socket blocked while sending fragment 119.
Socket blocked while sending fragment 120.
Socket blocked while sending fragment 121.
Socket blocked while sending fragment 122.
Socket blocked while sending fragment 124.
Socket blocked while sending fragment 125.
Socket blocked while sending fragment 126.
Socket blocked while sending fragment 127.
Socket blocked while sending fragment 129.
Socket blocked while sending fragment 130.
Socket blocked while sending fragment 131.
Socket blocked while sending fragment 133.
Socket blocked while sending fragment 136.
Socket blocked while sending fragment 137.
Socket blocked while sending fragment 138.
Socket blocked while sending fragment 139.
Socket blocked while sending fragment 140.
Socket blocked while sending fragment 142.
Socket blocked while sending fragment 143.
Socket blocked while sending fragment 145.
Socket blocked while sending fragment 146.
Socket blocked while sending fragment 147.
Socket blocked while sending fragment 149.
Socket blocked while sending fragment 151.
Socket blocked while sending fragment 152.
Socket blocked while sending fragment 153.
Socket blocked while sending fragment 154.
Socket blocked while sending fragment 155.
Socket blocked while sending fragment 156.
Socket blocked while sending fragment 160.
Socket blocked while sending fragment 161.
Socket blocked while sending fragment 162.
Socket blocked while sending fragment 164.

       6 |   26    16    |  283    15    |   850    523    |  3160    167    |  22     69       
   Cycle |  Sent  Recv   |  Sent  Recv   |   Sent   Recv   |   Sent   Recv   | Program System
         |  AUDIO (msg)  |  VIDEO (msg)  |  AUDIO (kbps)   |  VIDEO (kbps)   |     CPU (%) 
===========================================================================================
Socket blocked while sending fragment 44.
Socket blocked while sending fragment 46.
Socket blocked while sending fragment 47.
Socket blocked while sending fragment 49.
Socket blocked while sending fragment 50.
Socket blocked while sending fragment 51.
Keyboard interrupt detected.

Video application stopped.

=== Global bandwidth statistics ===
Audio sent:       867.25 kbps
Audio received:   391.50 kbps
Video sent:       1862.96 kbps
Video received:   593.89 kbps
Total time:       6.6 s
=====================================

=== FPS Statistics ===
Target FPS:       12.0
Average real FPS: 1.0
FPS efficiency:   8.3%
======================
\end{lstlisting}

\newpage

La imagen de la Figura \ref{fig:minimal_video_fps_1mbps} es una captura del vídeo recibido en esta prueba.
\begin{center}
  \includegraphics[width = 0.7\textwidth]{images/VideoRecibido1.2.png}
  \captionof{figure}{Vídeo recibido en \textit{Minimal\_Video\_FPS} para ancho de banda de 1 Mbps.}
  \label{fig:minimal_video_fps_1mbps}
\end{center}

\newpage

Finalmente, se ejecutará el módulo \textit{Minimal\_Video\_Resolution} con el mismo ancho de banda, 1 Mbps. El comando usado ha sido el siguiente:
\begin{lstlisting}[language=bash,basicstyle=\ttfamily\scriptsize]
python minimal_video_resolution.py -a 192.168.0.58 --show_video --show_stats -z 12 \\
-w 350 -g 250
\end{lstlisting}
Donde \verb|-a| indica la dirección IP destino, \verb|--show_video| indica que se active la transmisión de vídeo, \verb|--show_stats| indica que se muestren las estadísticas de la transmisión, \verb|-z| indica el número de fotogramas por segundo (FPS), \verb|-w| indica el ancho del vídeo y \verb|-g| indica la altura del vídeo.
\vspace{\baselineskip}

\begin{lstlisting}[language=bash,basicstyle=\ttfamily\tiny]
         |  AUDIO (msg)  |  VIDEO (msg)  |  AUDIO (kbps)   |  VIDEO (kbps)   |     CPU (%) 
   Cycle |  Sent  Recv   |  Sent  Recv   |   Sent   Recv   |   Sent   Recv   | Program System
================================================================================================
       1 |   24    24    |  188   103    |   785    785    |  2101   1152    |  20      0       
       2 |   31     7    |  131    79    |  1006    227    |  1453    876    |  23     70       
   Cycle |  Sent  Recv   |  Sent  Recv   |   Sent   Recv   |   Sent   Recv   | Program System
         |  AUDIO (msg)  |  VIDEO (msg)  |  AUDIO (kbps)   |  VIDEO (kbps)   |     CPU (%) 
===========================================================================================
\end{lstlisting}

\begin{lstlisting}[language=bash,basicstyle=\ttfamily\tiny]
Socket blocked while sending fragment 2.
Socket blocked while sending fragment 3.
Socket blocked while sending fragment 4.
Socket blocked while sending fragment 5.
Socket blocked while sending fragment 6.
Socket blocked while sending fragment 7.
Socket blocked while sending fragment 8.
Socket blocked while sending fragment 9.
Socket blocked while sending fragment 10.
Socket blocked while sending fragment 11.
Socket blocked while sending fragment 12.
Socket blocked while sending fragment 13.
Socket blocked while sending fragment 14.
Socket blocked while sending fragment 15.
Socket blocked while sending fragment 16.
Socket blocked while sending fragment 17.
Socket blocked while sending fragment 18.
Socket blocked while sending fragment 19.
Socket blocked while sending fragment 20.
Socket blocked while sending fragment 21.
Socket blocked while sending fragment 22.
Socket blocked while sending fragment 23.
Socket blocked while sending fragment 24.
Socket blocked while sending fragment 25.
Socket blocked while sending fragment 26.
Socket blocked while sending fragment 27.
Socket blocked while sending fragment 28.
Socket blocked while sending fragment 29.
Socket blocked while sending fragment 30.
Socket blocked while sending fragment 31.
Socket blocked while sending fragment 32.
Socket blocked while sending fragment 33.
Socket blocked while sending fragment 34.
Socket blocked while sending fragment 35.
Socket blocked while sending fragment 36.
Socket blocked while sending fragment 37.
Socket blocked while sending fragment 38.
Socket blocked while sending fragment 39.
Socket blocked while sending fragment 40.
Socket blocked while sending fragment 41.
Socket blocked while sending fragment 42.
Socket blocked while sending fragment 43.
Socket blocked while sending fragment 44.
Socket blocked while sending fragment 45.
Socket blocked while sending fragment 46.
Socket blocked while sending fragment 47.
Socket blocked while sending fragment 48.
Socket blocked while sending fragment 49.
Socket blocked while sending fragment 50.
Socket blocked while sending fragment 51.
Socket blocked while sending fragment 52.
Socket blocked while sending fragment 53.
Socket blocked while sending fragment 54.
Socket blocked while sending fragment 55.
Socket blocked while sending fragment 56.
       3 |   27    20    |  273     6    |   871    645    |  3007     63    |  41     72       
   Cycle |  Sent  Recv   |  Sent  Recv   |   Sent   Recv   |   Sent   Recv   | Program System
         |  AUDIO (msg)  |  VIDEO (msg)  |  AUDIO (kbps)   |  VIDEO (kbps)   |     CPU (%) 
===========================================================================================
Socket blocked while sending fragment 28.
Socket blocked while sending fragment 29.
Socket blocked while sending fragment 30.
Socket blocked while sending fragment 31.
Socket blocked while sending fragment 33.
Socket blocked while sending fragment 34.
Socket blocked while sending fragment 36.
Socket blocked while sending fragment 37.
Socket blocked while sending fragment 38.
Socket blocked while sending fragment 39.
Socket blocked while sending fragment 41.
Socket blocked while sending fragment 42.
Socket blocked while sending fragment 43.
Socket blocked while sending fragment 44.
Socket blocked while sending fragment 45.
Socket blocked while sending fragment 47.
Socket blocked while sending fragment 48.
Socket blocked while sending fragment 49.
Socket blocked while sending fragment 50.
Socket blocked while sending fragment 51.

       4 |   28     9    |  336    86    |   891    286    |  3655    936    |  33     66       
   Cycle |  Sent  Recv   |  Sent  Recv   |   Sent   Recv   |   Sent   Recv   | Program System
         |  AUDIO (msg)  |  VIDEO (msg)  |  AUDIO (kbps)   |  VIDEO (kbps)   |     CPU (%) 
===========================================================================================
Socket blocked while sending fragment 177.
Socket blocked while sending fragment 178.
Socket blocked while sending fragment 179.
Socket blocked while sending fragment 181.
Socket blocked while sending fragment 182.
Socket blocked while sending fragment 183.
Socket blocked while sending fragment 184.
Socket blocked while sending fragment 185.
Socket blocked while sending fragment 187.
       5 |   13     0    |  200    50    |   425      0    |  2231    559    |  16     59       
   Cycle |  Sent  Recv   |  Sent  Recv   |   Sent   Recv   |   Sent   Recv   | Program System
         |  AUDIO (msg)  |  VIDEO (msg)  |  AUDIO (kbps)   |  VIDEO (kbps)   |     CPU (%) 
===========================================================================================
Keyboard interrupt detected.

Video application stopped.

=== Global bandwidth statistics ===
Audio sent:       738.20 kbps
Audio received:   360.10 kbps
Video sent:       2311.06 kbps
Video received:   664.16 kbps
Total time:       5.5 s
=====================================

=== FPS Statistics ===
Target FPS:       12.0
Average real FPS: 1.1
FPS efficiency:   8.9%
======================

=== Resolution Statistics ===
Target resolution: 350x250
Actual resolution: 352x288
Average rescaling time: 3.05 ms
Performance impact:     3.7%
=============================

=== Camera compatible resolutions ===
  1. 320x240
  2. 352x288 * SELECTED
  3. 640x360
  4. 640x480
  5. 800x600
  6. 1024x768
  7. 1280x720
  8. 1280x1024
  9. 1366x768
  10. 1600x900
  11. 1920x1080
  12. 2560x1440
  13. 3840x2160

Camera device: /dev/video0
======================================
Program terminated.
QObject::killTimer: Timers cannot be stopped from another thread
QObject::~QObject: Timers cannot be stopped from another thread
\end{lstlisting}
\vspace{\baselineskip}

\newpage

La imagen de la Figura \ref{fig:minimal_video_resolution_1mbps} es una captura del vídeo recibido en esta prueba.
\begin{center}
  \includegraphics[width = 0.7\textwidth]{images/VideoRecibido1.3.png}
  \captionof{figure}{Vídeo recibido en \textit{Minimal\_Video\_Resolution} para ancho de banda de 1 Mbps.}
  \label{fig:minimal_video_resolution_1mbps}
\end{center}

\newpage

Ahora, se procederá a extraer las conclusiones correspondientes a las pruebas realizadas con un ancho de banda muy bajo, 1 Mbps.

\vspace{\baselineskip}

\textbf{Análisis de \textit{Minimal\_Video} con 1 Mbps de tasa de transferencia:}
\vspace{\baselineskip}

Al observar la salida de \textit{Minimal\_Video} bajo esta limitación de red, se aprecian varios puntos clave. En los primeros ciclos de ejecución del módulo, se intenta enviar el vídeo (por ejemplo, en el ciclo 1, se envían 1838 kbps de vídeo y se reciben 1628 kbps, mostrando un intento inicial antes de la saturación). Luego, las estadísticas globales muestran que se intentó enviar 1771.67 kbps de vídeo, pero solo se recibieron 762.21 kbps. Para el audio, las cifras son: 954.02 kbps enviados y 345.96 kbps recibidos.
\vspace{\baselineskip}

La aparición constante de mensajes como ``Socket bloqueado al enviar fragmento X'' indica que la red no puede manejar el volumen de datos que el módulo intenta transmitir. El módulo está generando datos (especialmente de vídeo) a una tasa que supera significativamente el límite de 1 Mbps. Esto da como resultado que una gran cantidad de paquetes se pierdan o se retrasen, haciendo que la experiencia de usuario sea muy deficiente, con el vídeo viéndose durante bastante tiempo entrecortado y congelado, así como un audio de mala calidad. A pesar de todo, se puede comprobar que una parte del vídeo y el audio consigue llegar al destinatario, aunque la calidad es muy pobre.

\vspace{\baselineskip}

\textbf{Análisis de \textit{Minimal\_Video\_FPS} con 1 Mbps de tasa de transferencia:}
\vspace{\baselineskip}

Este módulo, al estar bajo la misma limitación de 1 Mbps, muestra resultados similares.
\vspace{\baselineskip}

Las estadísticas globales muestran un intento de envío de 1862.96 kbps de vídeo (con 593.89 kbps recibidos) y 867.25 kbps de audio (con 391.50 kbps recibidos). Sin embargo, lo realmente importante en este módulo es ver si se consigue llegar a los 12 FPS objetivo, pero el promedio real fue de solo 1.0 FPS, con una eficiencia del 8.3\%. Esto demuestra que, aunque el módulo intenta gestionar la tasa de fotogramas, la limitación del ancho de banda impide alcanzar el objetivo. El módulo no puede enviar los 12 fotogramas deseados por segundo porque no hay suficiente ancho de banda. Los mensajes de ``Socket bloqueado'' siguen apareciendo, confirmando la congestión. La cantidad total de vídeo enviado es alta, y aunque el control de FPS intenta reducir la carga por fotograma para cumplir el objetivo, la tasa de envío global sigue siendo excesiva para 1 Mbps.

\vspace{\baselineskip}

\textbf{Análisis de \textit{Minimal\_Video\_Resolution} con 1 Mbps de tasa de transferencia:}
\vspace{\baselineskip}

Este módulo trata de reescalar la resolución del vídeo (en este caso, a 350x250, aunque la cámara selecciona la más cercana compatible que es 352x288) además de mantener el objetivo de conseguir 12 FPS. Los resultados con la limitación de 1 Mbps de ancho de banda siguen siendo muy deficientes.
\vspace{\baselineskip}

Las estadísticas globales indican que se intentó enviar 2311.06 kbps de vídeo y 738.20 kbps de audio, de los cuales se recibieron 664.16 kbps de vídeo y 360.10 kbps de audio. Los FPS reales promedio fueron de 1.1, con una eficiencia del 8.9\%, muy parecido al módulo anterior. Aunque se recibió algo de vídeo y audio, la cantidad de datos que se intentan enviar sigue superando con creces la capacidad de la red. Los mensajes de ``Socket bloqueado'' son persistentes, indicando que el módulo está intentando enviar datos que la red no puede transmitir. La combinación de intentar conseguir unos FPS concretos y reescalar a la resolución solicitada, bajo una restricción tan extrema, sigue resultando en una comunicación de muy baja calidad.

\textbf{Conclusiones para limitación 1 Mbps de tasa de transferencia:}

En general, estas pruebas demuestran que este ancho de banda es insuficiente para una comunicación de vídeo y audio mínimamente aceptable.
\begin{itemize}
\item \textbf{Congestión severa y constante:} Todos los módulos intentan transmitir datos a tasas que sobrepasan significativamente el límite de 1 Mbps (vídeo enviado entre 1771.67 kbps y 2311.06 kbps). Esto resulta en la continua aparición de mensajes ``Socket bloqueado'' y una considerable pérdida de paquetes, que se puede observar por la diferencia entre los kbps enviados y recibidos (por ejemplo, para vídeo, se reciben entre 593.89 kbps y 762.21 kbps).
\item \textbf{Ineficacia de control de FPS y reescalado resolución:} Características como el control de FPS o el reescalado de resolución no consiguen ofrecer una experiencia normal y usable al usuario. Aunque los módulos \textit{Minimal\_Video\_FPS} y \textit{Minimal\_Video\_Resolution} intentan gestionar los recursos, los FPS reales obtenidos (1.0 FPS y 1.1 FPS respectivamente, con eficiencias del 8.3\% y 8.9\%) están muy por debajo del objetivo de 12 FPS. Aunque es cierto que se recibe una fracción del vídeo y audio, la calidad general es inaceptable debido a la saturación de la red.
\end{itemize}

\newpage

\textbf{Pruebas de limitación de la tasa de transferencia a 10 Mbps}
\vspace{\baselineskip}

Empezaremos por probar el módulo con un ancho de banda bajo, 10 Mbps. El comando usado para \textit{Minimal\_Video} ha sido el siguiente:

\begin{lstlisting}[language=bash]
python minimal_video.py -a 192.168.0.58 --show_video --show_stats
\end{lstlisting}
Donde \verb|-a| es la dirección IP del dispositivo con el que se va a comunicar el módulo, \verb|--show_video| es la opción para mostrar el vídeo en tiempo real y \verb|--show_stats| es la opción para mostrar las estadísticas de la red.
\vspace{\baselineskip}

\begin{lstlisting}[language=bash,basicstyle=\ttfamily\tiny]

         |  AUDIO (msg)  |  VIDEO (msg)  |  AUDIO (kbps)   |  VIDEO (kbps)   |     CPU (%) 
   Cycle |  Sent  Recv   |  Sent  Recv   |   Sent   Recv   |   Sent   Recv   | Program System
================================================================================================
       1 |   31    31    |  165   155    |   977    977    |  1776   1670    |  20      0       
       2 |   26    26    |    0     0    |   825    825    |     0      0    |  41     71       
       3 |   16    16    |  482   433    |   520    520    |  5350   4803    |  28     82       
       4 |   38    38    | 1035   969    |  1174   1174    | 10923  10228    |  23     70       
   Cycle |  Sent  Recv   |  Sent  Recv   |   Sent   Recv   |   Sent   Recv   | Program System
         |  AUDIO (msg)  |  VIDEO (msg)  |  AUDIO (kbps)   |  VIDEO (kbps)   |     CPU (%) 
===========================================================================================
Socket blocked while sending fragment 85.
Socket blocked while sending fragment 86.
Socket blocked while sending fragment 42.
Socket blocked while sending fragment 43.
Socket blocked while sending fragment 57.
Socket blocked while sending fragment 56.
       5 |   32    32    |  791   742    |  1027   1027    |  8668   8130    |  37     74       
   Cycle |  Sent  Recv   |  Sent  Recv   |   Sent   Recv   |   Sent   Recv   | Program System
         |  AUDIO (msg)  |  VIDEO (msg)  |  AUDIO (kbps)   |  VIDEO (kbps)   |     CPU (%) 
===========================================================================================
Socket blocked while sending fragment 6.
Socket blocked while sending fragment 7.
Socket blocked while sending fragment 8.
Socket blocked while sending fragment 20.
Socket blocked while sending fragment 21.
Socket blocked while sending fragment 23.
Socket blocked while sending fragment 26.
Socket blocked while sending fragment 27.
Socket blocked while sending fragment 28.
Socket blocked while sending fragment 36.
Socket blocked while sending fragment 37.
Socket blocked while sending fragment 38.
Socket blocked while sending fragment 39.
Socket blocked while sending fragment 40.
Socket blocked while sending fragment 41.
Socket blocked while sending fragment 42.
Socket blocked while sending fragment 43.
Socket blocked while sending fragment 8.
Socket blocked while sending fragment 9.
Socket blocked while sending fragment 17.
Socket blocked while sending fragment 18.
Socket blocked while sending fragment 19.
Socket blocked while sending fragment 137.
Socket blocked while sending fragment 138.
Socket blocked while sending fragment 139.
Socket blocked while sending fragment 140.
Socket blocked while sending fragment 141.
Socket blocked while sending fragment 142.
Socket blocked while sending fragment 143.
Socket blocked while sending fragment 144.
Socket blocked while sending fragment 145.
Socket blocked while sending fragment 55.
Socket blocked while sending fragment 57.
Socket blocked while sending fragment 58.
Socket blocked while sending fragment 59.
Socket blocked while sending fragment 60.
Socket blocked while sending fragment 61.
       6 |   36    36    |  682   570    |  1172   1172    |  7582   6338    |  33     70       
       7 |   23    23    |  366   275    |   740    740    |  4024   3025    |  33     50       
       8 |   19     3    |  602   562    |   621     98    |  6726   6280    |  13     72       
   Cycle |  Sent  Recv   |  Sent  Recv   |   Sent   Recv   |   Sent   Recv   | Program System
         |  AUDIO (msg)  |  VIDEO (msg)  |  AUDIO (kbps)   |  VIDEO (kbps)   |     CPU (%) 
===========================================================================================
Video application stopped.

=== Global bandwidth statistics ===
Audio sent:       860.65 kbps
Audio received:   798.34 kbps
Video sent:       5481.62 kbps
Video received:   4927.59 kbps
Total time:       8.4 s
=====================================
Program terminated.
QObject::killTimer: Timers cannot be stopped from another thread
QObject::~QObject: Timers cannot be stopped from another thread
\end{lstlisting}
\vspace{\baselineskip}

La imagen de la Figura \ref{fig:minimal_video_10mbps} es una captura del vídeo recibido en esta prueba.
\begin{center}
  \includegraphics[width = 0.7\textwidth]{images/VideoRecibido2.1.png}
  \captionof{figure}{Vídeo recibido en \textit{Minimal\_Video} para ancho de banda de 10 Mbps.}
  \label{fig:minimal_video_10mbps}
\end{center}

\newpage

Ahora, ejecutaremos el módulo \textit{Minimal\_Video\_FPS} con el mismo ancho de banda, 10 Mbps. El comando usado ha sido el siguiente:
\begin{lstlisting}[language=bash, basicstyle=\ttfamily\scriptsize]
    python minimal_video_fps.py -a 192.168.0.58 --show_video --show_stats -z 12
\end{lstlisting}
Donde \verb|-a| es la dirección IP destino, \verb|--show_video| es para que se active la transmisión de vídeo, \verb|--show_stats| es para que se muestre la información de estadísticas y \verb|-z| es el numero de fotogramas por segundo (FPS) que se desea enviar. En este caso, se ha configurado para mostrar el vídeo a 12 FPS.

A continuación, se probará el módulo con un ancho de banda bajo, de 10 Mbps. El resultado obtenido ha sido el siguiente:
\vspace{\baselineskip}

\begin{lstlisting}[language=bash,basicstyle=\ttfamily\tiny]
         |  AUDIO (msg)  |  VIDEO (msg)  |  AUDIO (kbps)   |  VIDEO (kbps)   |     CPU (%) 
   Cycle |  Sent  Recv   |  Sent  Recv   |   Sent   Recv   |   Sent   Recv   | Program System
================================================================================================
       1 |   25    25    |  147     0    |   802    802    |  1613      0    |  22      0       
       2 |   33    10    |   18     0    |  1077    326    |   198      0    |  37     77       
       3 |   33    33    |    0     0    |  1080   1080    |     0      0    |  34     80       
       4 |   33    33    |  267   240    |  1077   1077    |  2977   2679    |  37     67       
       5 |   39    39    |  288   259    |  1232   1232    |  3106   2795    |  55     69       
       6 |   36    36    |  276   248    |  1172   1172    |  3068   2761    |  23     65
       7 |   32    32    |  243   219    |  1042   1042    |  2702   2432    |  30     69       
       8 |   33    33    |  220   198    |  1049   1049    |  2389   2150    |  35     58       
       9 |   35    35    |  198   178    |  1124   1124    |  2170   1953    |  46     67       
      10 |   31    31    |  215   194    |  1007   1007    |  2386   2147    |  26     69       
      11 |   30    30    |  266   239    |   963    963    |  2914   2623    |  35     71       
      12 |   35    35    |  284   256    |  1145   1145    |  3175   2858    |  37     80       
      13 |   32    32    |  266   239    |  1044   1044    |  2963   2667    |  31     74       
      14 |   29    29    |  186   167    |   941    941    |  2061   1855    |  30     78       
      15 |   33    33    |  184   166    |  1077   1077    |  2052   1847    |  37     80       
      16 |   34    34    |  231   208    |  1110   1110    |  2577   2319    |  26     77       
      17 |   29    29    |  201   181    |   926    926    |  2191   1972    |  22     73       
      18 |    1     1    |  132   119    |    32     32    |  1450   1305    |   4     60       
   Cycle |  Sent  Recv   |  Sent  Recv   |   Sent   Recv   |   Sent   Recv   | Program System
         |  AUDIO (msg)  |  VIDEO (msg)  |  AUDIO (kbps)   |  VIDEO (kbps)   |     CPU (%) 
===========================================================================================
Socket blocked while sending fragment 41.
Socket blocked while sending fragment 42.
Socket blocked while sending fragment 43.
Socket blocked while sending fragment 8.
Socket blocked while sending fragment 9.
Socket blocked while sending fragment 17.
Socket blocked while sending fragment 18.
Socket blocked while sending fragment 19.
Socket blocked while sending fragment 137.
Socket blocked while sending fragment 138.
Socket blocked while sending fragment 139.
Socket blocked while sending fragment 140.
Socket blocked while sending fragment 141.
Socket blocked while sending fragment 142.
Socket blocked while sending fragment 143.
Socket blocked while sending fragment 144.
Video application stopped.

=== Global bandwidth statistics ===
Audio sent:       980.18 kbps
Audio received:   939.41 kbps
Video sent:       2191.73 kbps
Video received:   1973.56 kbps
Total time:       18.5 s
=====================================

=== FPS Statistics ===
Target FPS:       12.0
Average real FPS: 1.2
FPS efficiency:   9.9%
======================
Program terminated.
QObject::killTimer: Timers cannot be stopped from another thread
QObject::~QObject: Timers cannot be stopped from another thread
\end{lstlisting}
\vspace{\baselineskip}

\newpage

La imagen de la Figura \ref{fig:minimal_video_fps_10mbps} es una captura del vídeo recibido en esta prueba.
\begin{center}
  \includegraphics[width = 0.7\textwidth]{images/VideoRecibido2.2.png}
  \captionof{figure}{Vídeo recibido en \textit{Minimal\_Video\_FPS} para ancho de banda de 10 Mbps.}
  \label{fig:minimal_video_fps_10mbps}
\end{center}

\newpage

Finalmente, se ejecutará el módulo \textit{Minimal\_Video\_Resolution} con el mismo ancho de banda, 10 Mbps. El comando usado ha sido el siguiente:
\begin{lstlisting}[language=bash,basicstyle=\ttfamily\scriptsize]
python minimal_video_resolution.py -a 192.168.0.58 --show_video --show_stats -z 12 \\
-w 350 -g 250
\end{lstlisting}
Donde \verb|-a| indica la dirección IP destino, \verb|--show_video| indica que se active la transmisión de vídeo, \verb|--show_stats| indica que se muestren las estadísticas de la transmisión, \verb|-z| indica el número de fotogramas por segundo (FPS), \verb|-w| indica el ancho del vídeo y \verb|-g| indica la altura del vídeo.
\vspace{\baselineskip}

\begin{lstlisting}[language=bash,basicstyle=\ttfamily\tiny]
         |  AUDIO (msg)  |  VIDEO (msg)  |  AUDIO (kbps)   |  VIDEO (kbps)   |     CPU (%) 
   Cycle |  Sent  Recv   |  Sent  Recv   |   Sent   Recv   |   Sent   Recv   | Program System
================================================================================================
       1 |   27    27    |  188   188    |   882    882    |  2098   2098    |  17      0       
       2 |   37    37    |    0     0    |  1150   1150    |     0      0    |  35     76       
       3 |   29    29    |  188   188    |   939    939    |  2079   2081    |  31     78       
       4 |   31    31    |  937   882    |  1014   1014    | 10470   9853    |  22     73       
   Cycle |  Sent  Recv   |  Sent  Recv   |   Sent   Recv   |   Sent   Recv   | Program System
         |  AUDIO (msg)  |  VIDEO (msg)  |  AUDIO (kbps)   |  VIDEO (kbps)   |     CPU (%) 
===========================================================================================
Socket blocked while sending fragment 64.
Socket blocked while sending fragment 65.
Socket blocked while sending fragment 66.
Socket blocked while sending fragment 67.
Socket blocked while sending fragment 71.
Socket blocked while sending fragment 72.
Socket blocked while sending fragment 73.
Socket blocked while sending fragment 84.
Socket blocked while sending fragment 85.
Socket blocked while sending fragment 136.
Socket blocked while sending fragment 137.
Socket blocked while sending fragment 138.
Socket blocked while sending fragment 139.
Socket blocked while sending fragment 140.
Socket blocked while sending fragment 141.
Socket blocked while sending fragment 142.
Socket blocked while sending fragment 143.
Socket blocked while sending fragment 144.
Socket blocked while sending fragment 145.
Socket blocked while sending fragment 146.
Socket blocked while sending fragment 147.
Socket blocked while sending fragment 148.
Socket blocked while sending fragment 149.
Socket blocked while sending fragment 150.
Socket blocked while sending fragment 151.
Socket blocked while sending fragment 152.
Socket blocked while sending fragment 153.
Socket blocked while sending fragment 154.
Socket blocked while sending fragment 155.
       5 |   35    35    |  848   786    |  1144   1144    |  9466   8773    |  24     73       
   Cycle |  Sent  Recv   |  Sent  Recv   |   Sent   Recv   |   Sent   Recv   | Program System
         |  AUDIO (msg)  |  VIDEO (msg)  |  AUDIO (kbps)   |  VIDEO (kbps)   |     CPU (%) 
===========================================================================================
Socket blocked while sending fragment 185.
Socket blocked while sending fragment 93.
Socket blocked while sending fragment 94.
Socket blocked while sending fragment 95.
Socket blocked while sending fragment 96.
Socket blocked while sending fragment 97.
Socket blocked while sending fragment 118.
Socket blocked while sending fragment 119.
Socket blocked while sending fragment 120.
Socket blocked while sending fragment 121.
Socket blocked while sending fragment 122.
Socket blocked while sending fragment 123.
Socket blocked while sending fragment 124.
Socket blocked while sending fragment 125.
Socket blocked while sending fragment 12.
Socket blocked while sending fragment 73.
Socket blocked while sending fragment 75.
Socket blocked while sending fragment 76.
Socket blocked while sending fragment 79.
Socket blocked while sending fragment 81.
Socket blocked while sending fragment 82.
Socket blocked while sending fragment 83.
Socket blocked while sending fragment 88.
Socket blocked while sending fragment 71.
Socket blocked while sending fragment 97.
Socket blocked while sending fragment 115.
Socket blocked while sending fragment 116.
Socket blocked while sending fragment 117.
Socket blocked while sending fragment 118.
       6 |   34    34    |  801   732    |  1067   1067    |  8582   7842    |  24     73       
   Cycle |  Sent  Recv   |  Sent  Recv   |   Sent   Recv   |   Sent   Recv   | Program System
         |  AUDIO (msg)  |  VIDEO (msg)  |  AUDIO (kbps)   |  VIDEO (kbps)   |     CPU (%) 
===========================================================================================
Socket blocked while sending fragment 137.
Socket blocked while sending fragment 138.
Socket blocked while sending fragment 139.
Socket blocked while sending fragment 140.
Socket blocked while sending fragment 141.
Socket blocked while sending fragment 142.
Socket blocked while sending fragment 143.
Socket blocked while sending fragment 144.
Socket blocked while sending fragment 178.
Socket blocked while sending fragment 187.
Socket blocked while sending fragment 29.
Socket blocked while sending fragment 30.
Socket blocked while sending fragment 31.
Socket blocked while sending fragment 32.
Socket blocked while sending fragment 33.
Socket blocked while sending fragment 34.
Socket blocked while sending fragment 35.
Socket blocked while sending fragment 36.
Socket blocked while sending fragment 37.
Socket blocked while sending fragment 38.
Socket blocked while sending fragment 39.
Socket blocked while sending fragment 175.
Socket blocked while sending fragment 184.
Socket blocked while sending fragment 185.
Socket blocked while sending fragment 24.
Socket blocked while sending fragment 25.
Socket blocked while sending fragment 26.
Socket blocked while sending fragment 27.
Socket blocked while sending fragment 28.
Socket blocked while sending fragment 29.
       7 |   32    32    |  868   801    |  1003   1003    |  9295   8577    |  25     72       
   Cycle |  Sent  Recv   |  Sent  Recv   |   Sent   Recv   |   Sent   Recv   | Program System
         |  AUDIO (msg)  |  VIDEO (msg)  |  AUDIO (kbps)   |  VIDEO (kbps)   |     CPU (%) 
===========================================================================================
Socket blocked while sending fragment 89.
Socket blocked while sending fragment 90.
Socket blocked while sending fragment 180.
Socket blocked while sending fragment 181.
Socket blocked while sending fragment 182.
Socket blocked while sending fragment 183.
Socket blocked while sending fragment 184.
Socket blocked while sending fragment 185.
Socket blocked while sending fragment 186.
Socket blocked while sending fragment 187.
Socket blocked while sending fragment 29.
^CSocket blocked while sending fragment 47.
Socket blocked while sending fragment 49.
Socket blocked while sending fragment 62.
Socket blocked while sending fragment 63.
Socket blocked while sending fragment 73.
Socket blocked while sending fragment 74.
Socket blocked while sending fragment 46.
Socket blocked while sending fragment 47.
Socket blocked while sending fragment 48.
Socket blocked while sending fragment 49.
Socket blocked while sending fragment 50.
       8 |    7     7    |  491   442    |   228    228    |  5469   4925    |   8     61       
   Cycle |  Sent  Recv   |  Sent  Recv   |   Sent   Recv   |   Sent   Recv   | Program System
         |  AUDIO (msg)  |  VIDEO (msg)  |  AUDIO (kbps)   |  VIDEO (kbps)   |     CPU (%) 
===========================================================================================
Video application stopped.

=== Global bandwidth statistics ===
Audio sent:       546.99 kbps
Audio received:   546.99 kbps
Video sent:       3477.81 kbps
Video received:   3234.63 kbps
Total time:       13.9 s
=====================================

=== FPS Statistics ===
Target FPS:       12.0
Average real FPS: 1.6
FPS efficiency:   13.7%
======================

=== Resolution Statistics ===
Target resolution: 350x250
Actual resolution: 352x288
Average rescaling time: 6.47 ms
Performance impact:     7.8%
=============================

=== Camera compatible resolutions ===
  1. 320x240
  2. 352x288 * SELECTED
  3. 640x360
  4. 640x480
  5. 800x600
  6. 1024x768
  7. 1280x720
  8. 1280x1024
  9. 1366x768
  10. 1600x900
  11. 1920x1080
  12. 2560x1440
  13. 3840x2160

Camera device: /dev/video0
======================================
Program terminated.
QObject::killTimer: Timers cannot be stopped from another thread
QObject::~QObject: Timers cannot be stopped from another thread
\end{lstlisting}
\vspace{\baselineskip}

La imagen de la Figura \ref{fig:minimal_video_resolution_10mbps} es una captura del vídeo recibido en esta prueba.
\begin{center}
  \includegraphics[width = 0.7\textwidth]{images/VideoRecibido2.3.png}
  \captionof{figure}{Vídeo recibido en \textit{Minimal\_Video\_Resolution} para ancho de banda de 10 Mbps.}
  \label{fig:minimal_video_resolution_10mbps}
\end{center}
\newpage

Ahora, se procederá a extraer las conclusiones correspondientes a las pruebas realizadas con un ancho de banda bajo, 10 Mbps.
\vspace{\baselineskip}

\textbf{Análisis de \textit{Minimal\_Video} con 10 Mbps de tasa de transferencia:}
\vspace{\baselineskip}

El comportamiento de \textit{Minimal\_Video} bajo el límite del ancho de banda de 10 Mbps mejora notablemente respecto a la prueba de 1 Mbps. Según las estadísticas globales, el módulo intentó enviar 5481.62 kbps de vídeo y 860.65 kbps de audio, logrando recibir 4927.59 kbps de vídeo y 798.34 kbps de audio. Esto indica que la mayor parte de los datos logra transmitirse correctamente, y la experiencia del usuario es considerablemente mejor: el vídeo y el audio son reconocibles y, en general, la comunicación es funcional.
\vspace{\baselineskip}

Sin embargo, la presencia de mensajes como ``Socket bloqueado al enviar fragmento X'' (por ejemplo, entre el ciclo 4 y 5, y entre el ciclo 5 y 6) indica que el módulo sigue generando picos de tráfico o un flujo sostenido que, en ocasiones, sobrepasa la capacidad de la red. En dichos ciclos, el socket bloquea temporalmente el envío de nuevos paquetes hasta aliviar la congestión en la cola de red. Es cierto que los fotogramas individuales que logran transmitirse pueden mantener una buena calidad visual, como se observa en la Figura \ref{fig:minimal_video_fps_10mbps}, estos bloqueos pueden resultar en una experiencia visual con interrupciones momentáneas en la fluidez o la aparición de artefactos leves que no son evidentes en cada fotograma individual capturado. El sistema funciona y la comunicación es reconocible, pero la fluidez no es perfectamente constante y aún existen limitaciones para un uso completamente estable.

\vspace{\baselineskip}

\textbf{Análisis de \textit{Minimal\_Video\_FPS} con 10 Mbps de tasa de transferencia:}
\vspace{\baselineskip}

Al aumentar el ancho de banda a 10 Mbps, este módulo, que intenta fijar la tasa de FPS en 12, muestra un resultado significativamente mejorado. Las estadísticas globales indican un envío de 980.18 kbps de audio (con 939.41 kbps recibidos), lo cual es positivo para el audio. Para el vídeo, se intentaron enviar 2191.73 kbps y se recibieron 1973.56 kbps, representando una eficiencia de transmisión del 90\%. Siguen apareciendo mensajes de ``Socket bloqueado'', lo que indica que el módulo sigue enfrentando congestión en la red, pero en menor medida que con 1 Mbps.
\vspace{\baselineskip}

La sección de ``Estadísticas de FPS'' revela que, pese a tener de objetivo 12 FPS, solo se logra un promedio real de 1.2 FPS (eficiencia del 9.9\%). Sin embargo, a diferencia de la situación anterior, ahora existe una recepción exitosa del 90\% del vídeo transmitido. Esta aparente contradicción entre el bajo FPS real (1.2) y la alta eficiencia de transmisión de vídeo (90\%) sugiere que el sistema logra transmitir los fotogramas de vídeo de manera efectiva cuando los genera, pero la limitación principal radica en la capacidad de generación/captura de fotogramas por parte del módulo. La visualización del vídeo es funcional en este caso, aunque con una tasa de fotogramas considerablemente inferior a la deseada, lo que indica que el cuello de botella se encuentra en el procesamiento local más que en la capacidad de la red con 10 Mbps.

\vspace{\baselineskip}

\textbf{Análisis de \textit{Minimal\_Video\_Resolution} con 10 Mbps de tasa de transferencia:}
\vspace{\baselineskip}

En este caso, el módulo intenta transmitir vídeo a 12 FPS y una resolución objetivo de 350x250 píxeles (ajustando realmente a 352x288, que es la resolución compatible según la cámara). Los datos globales muestran que se enviaron 3477.81 kbps de vídeo (con 3234.63 kbps recibidos) y 546.99 kbps de audio (recibidos en su totalidad). La recepción de vídeo y audio es considerable, lo que es una mejora respecto al módulo \textit{Minimal\_Video\_FPS}.
\vspace{\baselineskip}

Las ``Estadísticas de FPS'' indican un promedio real de 1.6 FPS frente al objetivo de 12 (eficiencia del 13.7\%). Aunque se recibe vídeo y los fotogramas individuales pueden mostrar una calidad visual aceptable como se aprecia en la Figura \ref{fig:minimal_video_resolution_10mbps}, la eficiencia de FPS es baja. Esto resulta en una experiencia de vídeo con una fluidez muy limitada y una percepción de movimiento entrecortada, ya que la tasa de actualización de la imagen es significativamente inferior a la deseada.  Los mensajes de ``Socket bloqueado'' (visibles entre los ciclos 4 y 5, 5 y 6, etc.) confirman esta congestión. El tiempo de reescalado promedio es de 6.47 ms, con un impacto en CPU del 7.8\%, lo cual es manejable. El sistema transmite vídeo y audio, pero la fluidez es muy pobre debido la limitación de la red para sostener 12 FPS.
\vspace{\baselineskip}

\textbf{Conclusiones para limitación 10 Mbps de tasa de transferencia:}

La ejecución de los módulos con un ancho de banda de 10 Mbps muestra resultados mixtos, con una mejora general respecto a 1 Mbps, pero con problemas significativos en algunos escenarios.
\begin{itemize}
\item \textbf{Transmisión variable según el módulo:} \textit{Minimal\_Video} y \textit{Minimal\_Video\_Resolution} logran transmitir una cantidad sustancial de datos de vídeo y audio (vídeo recibido: 4927.59 kbps y 3234.63 kbps respectivamente), permitiendo una comunicación reconocible. \textit{Minimal\_Video\_FPS} logra una transmisión de vídeo parcialmente efectiva (1973.56 kbps recibidos de 2191.73 kbps enviados, 90\% de eficiencia) pero con limitaciones en la generación de fotogramas.
\item \textbf{Congestión:} A pesar del aumento de ancho de banda, los mensajes de ``Socket bloqueado'' siguen apareciendo en todos los módulos. La baja eficiencia de FPS en los módulos que lo gestionan (9.9\% para \textit{Minimal\_Video\_FPS} y 13.7\% para \textit{Minimal\_Video\_Resolution}, con FPS reales de 1.2 y 1.6 respectivamente) indica que el cuello de botella principal no radica en la capacidad de red sino en el procesamiento local y la generación de fotogramas. Para \textit{Minimal\_Video\_FPS}, la red de 10 Mbps es capaz de transmitir eficientemente el vídeo generado, pero la baja tasa de generación limita la experiencia.
\item \textbf{Impacto de funcionalidades avanzadas:} El intento de fijar FPS en \textit{Minimal\_Video\_FPS} revela una separación clara entre la capacidad de transmisión de red (exitosa al 90\%) y la capacidad de generación de fotogramas (limitada al 9.9\% de eficiencia). \textit{Minimal\_Video\_Resolution}, aunque con FPS bajos, gestiona mejor tanto la generación como la transmisión. El coste de CPU del reescalado en \textit{Minimal\_Video\_Resolution} (7.8\%) es moderado.
\item \textbf{Mejora pequeña en rendimiento:} Mientras que \textit{Minimal\_Video} ofrece una experiencia básica funcional, los módulos que intentan controlar los FPS muestran que el principal desafío no es el ancho de banda de 10 Mbps sino la optimización del procesamiento local. \textit{Minimal\_Video\_FPS} demuestra que la red puede transmitir efectivamente el vídeo cuando está disponible, pero la generación de fotogramas a 12 FPS requiere optimizaciones en el procesamiento de vídeo local más que un mayor ancho de banda.
\end{itemize}

\newpage

\textbf{Pruebas de limitación de la tasa de transferencia a 50 Mbps}
\vspace{\baselineskip}

Empezaremos por probar el módulo con un ancho de banda alto, 50 Mbps. El comando usado para \textit{Minimal\_Video} ha sido el siguiente:

\begin{lstlisting}[language=bash]
python minimal_video.py -a 192.168.0.58 --show_video --show_stats
\end{lstlisting}
Donde \verb|-a| es la dirección IP del dispositivo con el que se va a comunicar el módulo, \verb|--show_video| es la opción para mostrar el vídeo en tiempo real y \verb|--show_stats| es la opción para mostrar las estadísticas de la red.
\vspace{\baselineskip}

\begin{lstlisting}[language=bash,basicstyle=\ttfamily\tiny]
         |  AUDIO (msg)  |  VIDEO (msg)  |  AUDIO (kbps)   |  VIDEO (kbps)   |     CPU (%) 
   Cycle |  Sent  Recv   |  Sent  Recv   |   Sent   Recv   |   Sent   Recv   | Program System
================================================================================================
       1 |   29    29    |  165   153    |   949    949    |  1845   1713    |  16      0       
       2 |   35    35    |    0     0    |  1121   1121    |     0      0    |  45     78       
       3 |   31    31    |  464   463    |  1008   1008    |  5152   5137    |  40     71       
       4 |   29    29    |  975   912    |   946    946    | 10861  10161    |  23     76       
       5 |   26    26    |  865   747    |   845    845    |  9600   8291    |  36     66       
       6 |   33    33    | 1231  1143    |  1080   1080    | 13759  12772    |  48     74       
       7 |   39    39    |  444   313    |  1241   1241    |  4823   3402    |  49     72       
       8 |   37    37    | 1129  1094    |  1164   1164    | 12125  11749    |  41     75       
       9 |   29    29    | 1250  1186    |   948    948    | 13957  13240    |  31     78       
      10 |   42    42    |  924   790    |  1371   1371    | 10296   8803    |  36     67       
      11 |   37    22    |  717   582    |  1203    715    |  7961   6462    |  43     71       
      12 |   29    29    | 1068  1004    |   948    948    | 11922  11209    |  31     69       
      13 |   28    28    |  990   980    |   916    916    | 11058  10948    |  33     73       
      14 |   15    15    |  818   787    |   489    489    |  9117   8769    |  23     72       
   Cycle |  Sent  Recv   |  Sent  Recv   |   Sent   Recv   |   Sent   Recv   | Program System
         |  AUDIO (msg)  |  VIDEO (msg)  |  AUDIO (kbps)   |  VIDEO (kbps)   |     CPU (%) 
===========================================================================================
Video application stopped.

=== Global bandwidth statistics ===
Audio sent:       1000.25 kbps
Audio received:   966.08 kbps
Video sent:       8587.79 kbps
Video received:   7898.40 kbps
Total time:       14.4 s
=====================================
Program terminated.
QObject::killTimer: Timers cannot be stopped from another thread
QObject::~QObject: Timers cannot be stopped from another thread
\end{lstlisting}
\vspace{\baselineskip}

\newpage

La imagen de la Figura \ref{fig:minimal_video_50mbps} es una captura del vídeo recibido en esta prueba.
\begin{center}
  \includegraphics[width = 0.7\textwidth]{images/VideoRecibido3.1.png}
  \captionof{figure}{Vídeo recibido en \textit{Minimal\_Video} para ancho de banda de 50 Mbps.}
  \label{fig:minimal_video_50mbps}
\end{center}

\newpage
Ahora, ejecutaremos el módulo \textit{Minimal\_Video\_FPS} con el mismo ancho de banda, 50 Mbps. El comando usado ha sido el siguiente:
\begin{lstlisting}[language=bash, basicstyle=\ttfamily\scriptsize]
    python minimal_video_fps.py -a 192.168.0.58 --show_video --show_stats -z 12
\end{lstlisting}
Donde \verb|-a| es la dirección IP destino, \verb|--show_video| es para que se active la transmisión de vídeo, \verb|--show_stats| es para que se muestre la información de estadísticas y \verb|-z| es el numero de fotogramas por segundo (FPS) que se desea enviar. En este caso, se ha configurado para mostrar el vídeo a 12 FPS.
\vspace{\baselineskip}

\begin{lstlisting}[language=bash,basicstyle=\ttfamily\tiny]
         |  AUDIO (msg)  |  VIDEO (msg)  |  AUDIO (kbps)   |  VIDEO (kbps)   |     CPU (%) 
   Cycle |  Sent  Recv   |  Sent  Recv   |   Sent   Recv   |   Sent   Recv   | Program System
================================================================================================
       1 |   20    20    |  165   165    |   651    651    |  1836   1836    |  24      0       
       2 |   34    34    |  332   330    |   957    957    |  3191   3172    |  30     76       
       3 |   26    26    |  946   901    |   850    850    | 10564  10059    |  24     72       
       4 |   39    36    |  557   330    |  1243   1147    |  6059   3590    |  38     63       
       5 |   28    28    | 1077   997    |   915    915    | 12016  11123    |  28     71       
       6 |   31    31    | 1142  1112    |  1014   1014    | 12752  12419    |  45     74       
       7 |   35    35    |  940   922    |  1140   1140    | 10458  10257    |  34     73       
       8 |   38    25    |  391   199    |  1243    817    |  4367   2220    |  45     66       
       9 |   30    24    |  718   641    |   980    784    |  8006   7151    |  43     75       
      10 |   25    25    |  825   825    |   818    818    |  9221   9221    |  38     78       
      11 |   37    37    | 1155  1141    |  1207   1207    | 12871  12719    |  43     75       
      12 |   31    31    |  825   810    |  1014   1014    |  9219   9051    |  32     76       
      13 |   28    28    |  825   825    |   912    912    |  9181   9181    |  30     70       
   Cycle |  Sent  Recv   |  Sent  Recv   |   Sent   Recv   |   Sent   Recv   | Program System
         |  AUDIO (msg)  |  VIDEO (msg)  |  AUDIO (kbps)   |  VIDEO (kbps)   |     CPU (%) 
===========================================================================================
Video application stopped.

=== Global bandwidth statistics ===
Audio sent:       976.67 kbps
Audio received:   923.22 kbps
Video sent:       8209.75 kbps
Video received:   7629.43 kbps
Total time:       13.5 s
=====================================

=== FPS Statistics ===
Target FPS:       12.0
Average real FPS: 5.0
FPS efficiency:   41.5%
======================
Program terminated.
QObject::killTimer: Timers cannot be stopped from another thread
QObject::~QObject: Timers cannot be stopped from another thread
\end{lstlisting}
\vspace{\baselineskip}

\newpage

La imagen de la Figura \ref{fig:minimal_video_fps_50mbps} es una captura del vídeo recibido en esta prueba.
\begin{center}
  \includegraphics[width = 0.7\textwidth]{images/VideoRecibido3.2.png}
  \captionof{figure}{Vídeo recibido en \textit{Minimal\_Video\_FPS} para ancho de banda de 50 Mbps.}
  \label{fig:minimal_video_fps_50mbps}
\end{center}

\newpage
Finalmente, se ejecutará el módulo \textit{Minimal\_Video\_Resolution} con el mismo ancho de banda, 50 Mbps. El comando usado ha sido el siguiente:

\begin{lstlisting}[language=bash,basicstyle=\ttfamily\scriptsize]
python minimal_video_resolution.py -a 192.168.0.58 --show_video --show_stats -z 12 \\
-w 350 -g 250
\end{lstlisting}
Donde \verb|-a| indica la dirección IP destino, \verb|--show_video| indica que se active la transmisión de vídeo, \verb|--show_stats| indica que se muestren las estadísticas de la transmisión, \verb|-z| indica el número de fotogramas por segundo (FPS), \verb|-w| indica el ancho del vídeo y \verb|-g| indica la altura del vídeo.
\vspace{\baselineskip}

\begin{lstlisting}[language=bash,basicstyle=\ttfamily\tiny]
         |  AUDIO (msg)  |  VIDEO (msg)  |  AUDIO (kbps)   |  VIDEO (kbps)   |     CPU (%) 
   Cycle |  Sent  Recv   |  Sent  Recv   |   Sent   Recv   |   Sent   Recv   | Program System
================================================================================================
       1 |   32    32    |  188   186    |  1042   1042    |  2091   2071    |  26      0       
       2 |   36    36    |  259   211    |  1166   1166    |  2865   2331    |  34     70       
       3 |   29    29    | 1055  1032    |   946    946    | 11757  11503    |  34     72       
       4 |   31    31    |  967   924    |  1003   1003    | 10686  10212    |  30     71       
       5 |   25    25    | 1190  1184    |   798    798    | 12976  12907    |  35     77       
       6 |   28    28    | 1052  1044    |   916    916    | 11755  11666    |  34     71       
       7 |   35    35    |  827   688    |  1144   1144    |  9235   7681    |  38     71       
       8 |   21    21    |  784   727    |   684    684    |  8720   8087    |  14     78       
       9 |   34    34    |  996   951    |  1111   1111    | 11113  10607    |  43     69       
      10 |   30    30    | 1018  1019    |   981    981    | 11363  11377    |  31     71       
      11 |   29    29    | 1085  1010    |   949    949    | 12127  11288    |  36     73       
      12 |   30    30    |  878   862    |   980    980    |  9801   9620    |  24     73       
      13 |   31    31    |  929   866    |  1004   1004    | 10280   9584    |  32     74       
      14 |   31    11    |  228     0    |  1014    360    |  2548      0    |  42     65       
      15 |   30     0    |  200     0    |   982      0    |  2233      0    |  45     63       
      16 |   31     0    |  201     0    |  1004      0    |  2222      0    |  45     67       
      17 |    0     0    |  175     0    |     0      0    |  1932      0    |   2     12       
   Cycle |  Sent  Recv   |  Sent  Recv   |   Sent   Recv   |   Sent   Recv   | Program System
         |  AUDIO (msg)  |  VIDEO (msg)  |  AUDIO (kbps)   |  VIDEO (kbps)   |     CPU (%) 
===========================================================================================
Video application stopped.

=== Global bandwidth statistics ===
Audio sent:       904.14 kbps
Audio received:   752.51 kbps
Video sent:       7688.82 kbps
Video received:   6840.18 kbps
Total time:       17.5 s
=====================================

=== FPS Statistics ===
Target FPS:       12.0
Average real FPS: 3.6
FPS efficiency:   29.8%
======================

=== Resolution Statistics ===
Target resolution: 350x250
Actual resolution: 352x288
Average rescaling time: 5.53 ms
Performance impact:     6.6%
=============================

=== Camera compatible resolutions ===
  1. 320x240
  2. 352x288 * SELECTED
  3. 640x360
  4. 640x480
  5. 800x600
  6. 1024x768
  7. 1280x720
  8. 1280x1024
  9. 1366x768
  10. 1600x900
  11. 1920x1080
  12. 2560x1440
  13. 3840x2160

Camera device: /dev/video0
======================================
Program terminated.
QObject::killTimer: Timers cannot be stopped from another thread
QObject::~QObject: Timers cannot be stopped from another thread
\end{lstlisting}
\vspace{\baselineskip}

\newpage
La imagen de la Figura \ref{fig:minimal_video_resolution_50mbps} es una captura del vídeo recibido en esta prueba.
\begin{center}
  \includegraphics[width = 0.7\textwidth]{images/VideoRecibido3.3.png}
  \captionof{figure}{Vídeo recibido en \textit{Minimal\_Video\_Resolution} para ancho de banda de 50 Mbps.}
  \label{fig:minimal_video_resolution_50mbps}
\end{center}

\newpage

Ahora, se procederá a extraer las conclusiones correspondientes a las pruebas realizadas con un ancho de banda alto, 50 Mbps.
\vspace{\baselineskip}

\textbf{Análisis de \textit{Minimal\_Video} con 50 Mbps de tasa de transferencia:}
\vspace{\baselineskip}

Al examinar el comportamiento de \textit{Minimal\_Video} bajo esta condición de red de 50 Mbps, se observa una mejora muy significativa respecto a las pruebas con anchos de banda inferiores. Las estadísticas globales muestran que el módulo intentó enviar 8587.79 kbps de vídeo y 1000.25 kbps de audio, logrando recibir 7898.40 kbps de vídeo y 966.08 kbps de audio. Estos valores representan un porcentaje de recepción muy alto (aproximadamente el 92.0\% para vídeo y 96.6\% para audio), lo que indica que el ancho de banda disponible es bien aprovechado y suficiente para la carga generada.
\vspace{\baselineskip}

Es importante destacar que en la salida proporcionada no se observan mensajes de ``Socket bloqueado'', lo que confirma que la red puede manejar adecuadamente el flujo de datos generado por el módulo sin llegar a la congestión. La experiencia del usuario mejora notablemente: el vídeo se muestra fluido y el audio se escucha claro y continuo. Dado que este módulo no implementa un control específico de FPS (por defecto estará establecido a 30 FPS), la tasa de fotogramas variará según la carga del sistema y la capacidad de la cámara, pero la transmisión de los fotogramas generados es eficiente.

\vspace{\baselineskip}

\textbf{Análisis de \textit{Minimal\_Video\_FPS} con 50 Mbps de tasa de transferencia:}
\vspace{\baselineskip}

Este módulo, que intenta obtener una tasa de fotogramas de 12 FPS, muestra un comportamiento considerablemente mejorado con 50 Mbps. Los datos globales indican un envío de 8209.75 kbps de vídeo (con 7629.43 kbps recibidos) y 976.67 kbps de audio (con 923.22 kbps recibidos). El porcentaje de recepción es alto (aproximadamente 92.9\% para vídeo y 94.5\% para audio), lo que sugiere que el ancho de banda es en su mayor parte adecuado.
\vspace{\baselineskip}

La sección de ``Estadísticas de FPS'' revela que el módulo alcanza un promedio real de 5.0 FPS frente al objetivo de 12 FPS, lo que representa una eficiencia del 41.5\%. Aunque esto supone una mejora sustancial respecto a las pruebas con anchos de banda más limitados y permite una experiencia visual más fluida, todavía no se alcanza el objetivo. Este rendimiento sugiere que, incluso con un ancho de banda amplio, pueden existir otros factores limitantes (posiblemente relacionados con el procesamiento necesario para capturar, codificar y enviar a 12 FPS) que impiden alcanzar la tasa objetivo de forma consistente. Sin embargo, la experiencia del usuario es significativamente mejor, con un vídeo más estable.

\vspace{\baselineskip}

\textbf{Análisis de \textit{Minimal\_Video\_Resolution} con 50 Mbps de tasa de transferencia:}
\vspace{\baselineskip}

Este módulo, que combina el control de FPS a 12 con una resolución objetivo de 350x250 (ajustada a la compatible 352x288), muestra un rendimiento aceptable con 50 Mbps, aunque con algunas particularidades. Las estadísticas globales indican un envío de 7688.82 kbps de vídeo y 904.14 kbps de audio. De estos, se recibieron 6840.18 kbps de vídeo y 752.51 kbps de audio. Es importante señalar que los últimos ciclos de la prueba (del ciclo 14 al 17) muestran una recepción de 0 kbps tanto para audio como para vídeo, lo que indica una interrupción o fallo en la comunicación hacia el final de la prueba. Si excluimos estos ciclos finales problemáticos, la recepción en los ciclos anteriores fue muy buena.

Las ``Estadísticas de FPS'' muestran que se logran 3.6 FPS reales de los 12 FPS objetivo (eficiencia del 29.8\%). Este valor es inferior al del módulo \textit{Minimal\_Video\_FPS}, y también se ve afectado por la interrupción final. El ``Tiempo de reescalado promedio'' es de 5.53 ms, con un impacto en el rendimiento (CPU) del 6.6\%. Este coste de procesamiento es moderado. La resolución real utilizada es 352x288. La interrupción en la recepción al final de la prueba afecta negativamente la percepción de fiabilidad, aunque durante la mayor parte de la ejecución la transmisión fue buena.
\vspace{\baselineskip}

\textbf{Conclusiones para limitación 50 Mbps:}

La ejecución de los módulos con un ancho de banda de 50 Mbps muestra un rendimiento significativamente superior, acercándose a una experiencia de comunicación de vídeo y audio de buena calidad, aunque persisten algunas limitaciones.
\begin{itemize}
\item \textbf{Transmisión mayoritariamente exitosa y sin congestión de red:} Para los módulos, concretamente, \textit{Minimal\_Video} y \textit{Minimal\_Video\_FPS}, la gran mayoría de los datos enviados son recibidos correctamente (tasas de recepción de vídeo superiores al 92\%). Los mensajes de ``Socket bloqueado'' no aparecen, indicando que la red de 50 Mbps maneja adecuadamente el flujo de datos generado. \textit{Minimal\_Video\_Resolution} también muestra buena recepción durante la mayor parte de la prueba, pero sufre una interrupción en la recepción de datos al final.
\item \textbf{Mejora en la eficacia del control de FPS, pero sin alcanzar el objetivo:} El módulo \textit{Minimal\_Video\_FPS} alcanza una eficiencia del 41.5\% (5.0 FPS reales), y \textit{Minimal\_Video\_Resolution} un 29.8\% (3.6 FPS reales). Aunque es una mejora considerable que se traduce en un vídeo más fluido, el objetivo de 12 FPS no se cumple, sugiriendo cuellos de botella más allá del ancho de banda de red (posiblemente procesamiento o limitaciones de la aplicación).
\item \textbf{Viabilidad del reescalado de resolución con impacto moderado:} Con 50 Mbps, el reescalado a 352x288 es viable, con un impacto en CPU del 6.6\% para \textit{Minimal\_Video\_Resolution}, lo cual es aceptable.
\item \textbf{Estabilidad y factores limitantes adicionales:} \textit{Minimal\_Video} es el más estable en términos de transmisión pura. Los módulos con control de FPS es cierto que mejoran per no logran el máximo rendimiento, lo que apunta a la necesidad de optimizar la aplicación o considerar las limitaciones de hardware/procesamiento para alcanzar tasas de fotogramas más altas. 
\end{itemize}

Esta prueba demuestra que un ancho de banda de 50 Mbps es suficiente para una experiencia de videollamada de buena calidad, especialmente con el módulo base. Para los módulos con control de FPS, aunque la calidad mejora sustancialmente, el no alcanzar el objetivo de 12 FPS sugiere que otros factores además del ancho de banda de red entran en juego.
\newpage

\subsubsection{Experimentos con jitter}

En esta sección, se analizarán los resultados obtenidos al aplicar una limitación de latencia (delay) a los distintos módulos.

\begin{itemize}
    \item \textbf{Pruebas con jitter)}
\end{itemize}

\textbf{Pruebas con jitter mínimo (Latencia ideal de 0 ms)}
\vspace{\baselineskip}

Comenzamos probando con una latencia ideal de 0 ms, lo que implica un jitter mínimo o despreciable, inherente a la red local.

- El comando usado para \textit{Minimal\_Video} ha sido el siguiente:
\vspace{\baselineskip}

\begin{lstlisting}[language=bash]
python minimal_video.py -a 192.168.0.58 --show_video --show_stats
\end{lstlisting}
Donde \verb|-a| es la dirección IP del dispositivo con el que se va a comunicar el módulo, \verb|--show_video| es la opción para mostrar el vídeo en tiempo real y \verb|--show_stats| es la opción para mostrar las estadísticas de la red.
\vspace{\baselineskip}

\begin{lstlisting}[language=bash,basicstyle=\ttfamily\tiny]
         |  AUDIO (msg)  |  VIDEO (msg)  |  AUDIO (kbps)   |  VIDEO (kbps)   |     CPU (%) 
   Cycle |  Sent  Recv   |  Sent  Recv   |   Sent   Recv   |   Sent   Recv   | Program System
================================================================================================
       1 |   29     0    |  165   136    |   947      0    |  1841   1519    |  26      0       
       2 |   38    33    |    0     0    |  1244   1080    |     0      0    |  44     79       
       3 |   30    30    |  237   210    |   979    979    |  2642   2339    |  39     73       
       4 |   33    33    |  688   665    |  1064   1064    |  7578   7325    |  36     77       
       5 |   26    26    | 1198  1141    |   847    847    | 13332  12699    |  36     74       
       6 |   23    23    |  601   592    |   744    744    |  6643   6541    |  22     77       
       7 |   33    33    |  613   572    |  1078   1078    |  6837   6379    |  41     75       
       8 |   26    26    | 1097  1085    |   849    849    | 12235  12101    |  27     77       
       9 |   28    28    |  917   910    |   914    914    | 10220  10142    |  24     73       
      10 |   26    26    | 1101  1075    |   813    813    | 11757  11479    |  30     73       
      11 |   32    32    |  607   557    |   993    993    |  6435   5903    |  27     77       
      12 |   28    28    |  369   368    |   895    895    |  4028   4020    |  32     78       
      13 |    9     9    |  301   302    |   293    293    |  3355   3368    |  15     68       
   Cycle |  Sent  Recv   |  Sent  Recv   |   Sent   Recv   |   Sent   Recv   | Program System
         |  AUDIO (msg)  |  VIDEO (msg)  |  AUDIO (kbps)   |  VIDEO (kbps)   |     CPU (%) 
===========================================================================================
Video application stopped.

=== Global bandwidth statistics ===
Audio sent:       882.48 kbps
Audio received:   799.36 kbps
Video sent:       6587.96 kbps
Video received:   6353.34 kbps
Total time:       13.4 s
=====================================
Program terminated.
QObject::killTimer: Timers cannot be stopped from another thread
QObject::~QObject: Timers cannot be stopped from another thread
\end{lstlisting}
\vspace{\baselineskip}

\newpage
La imagen de la Figura \ref{fig:minimal_video_0ms} es una captura del vídeo recibido en esta prueba.
\begin{center}
  \includegraphics[width = 0.7\textwidth]{images/VideoRecibido4.1.png}
  \captionof{figure}{Vídeo recibido en \textit{Minimal\_Video} con latencia base de 0 ms.}
  \label{fig:minimal_video}
\end{center}

\newpage

Ahora, ejecutaremos el módulo \textit{Minimal\_Video\_FPS} con la misma latencia base de 0 ms y, por tanto, jitter mínimo. El comando usado ha sido el siguiente:

\begin{lstlisting}[language=bash, basicstyle=\ttfamily\scriptsize]
    python minimal_video_fps.py -a 192.168.0.58 --show_video --show_stats -z 12
\end{lstlisting}
Donde \verb|-a| es la dirección IP destino, \verb|--show_video| es para que se active la transmisión de vídeo, \verb|--show_stats| es para que se muestre la información de estadísticas y \verb|-z| es el numero de fotogramas por segundo (FPS) que se desea enviar. En este caso, se ha configurado para mostrar el vídeo a 12 FPS.
\vspace{\baselineskip}


\begin{lstlisting}[language=bash,basicstyle=\ttfamily\tiny]
         |  AUDIO (msg)  |  VIDEO (msg)  |  AUDIO (kbps)   |  VIDEO (kbps)   |     CPU (%) 
   Cycle |  Sent  Recv   |  Sent  Recv   |   Sent   Recv   |   Sent   Recv   | Program System
================================================================================================
       1 |   26    26    |  165   156    |   850    850    |  1841   1743    |  29      0       
       2 |   42    42    |    0     0    |  1335   1335    |     0      0    |  46     77       
       3 |   33    33    |  415   415    |  1031   1031    |  4430   4428    |  31     85       
       4 |   36    35    |  673   637    |  1060   1030    |  6767   6406    |  18     76       
       5 |   37    38    | 1101  1087    |  1197   1229    | 12163  12011    |  26     74       
       6 |   29    29    |  938   928    |   944    944    | 10435  10319    |  38     76       
       7 |   33    33    |  948   871    |  1080   1080    | 10595   9737    |  32     72       
       8 |   39    39    |  939   890    |  1274   1274    | 10478   9931    |  53     77       
       9 |   37    19    |  351   129    |  1210    621    |  3919   1440    |  48     72       
      10 |   24    24    | 1027  1000    |   785    785    | 11475  11176    |  29     77       
      11 |   15    15    | 1074  1029    |   489    489    | 11974  11474    |  33     74       
      12 |   34    34    | 1385  1344    |  1074   1074    | 14941  14497    |  33     75       
      13 |    8     8    |  426   427    |   260    260    |  4737   4751    |  10     71       
   Cycle |  Sent  Recv   |  Sent  Recv   |   Sent   Recv   |   Sent   Recv   | Program System
         |  AUDIO (msg)  |  VIDEO (msg)  |  AUDIO (kbps)   |  VIDEO (kbps)   |     CPU (%) 
===========================================================================================
Video application stopped.

=== Global bandwidth statistics ===
Audio sent:       955.12 kbps
Audio received:   911.37 kbps
Video sent:       7834.12 kbps
Video received:   7395.49 kbps
Total time:       13.5 s
=====================================

=== FPS Statistics ===
Target FPS:       12.0
Average real FPS: 5.4
FPS efficiency:   45.2%
======================
Program terminated.
QObject::killTimer: Timers cannot be stopped from another thread
QObject::~QObject: Timers cannot be stopped from another thread
\end{lstlisting}

\newpage
La imagen de la Figura \ref{fig:minimal_video_fps_0ms} es una captura del vídeo recibido en esta prueba.
\begin{center}
  \includegraphics[width = 0.7\textwidth]{images/VideoRecibido4.2.png}
  \captionof{figure}{Vídeo recibido en \textit{Minimal\_Video\_FPS} con latencia base de 0 ms.}
  \label{fig:minimal_video_fps_0ms}
\end{center}
\newpage

Finalmente, se ejecutará el módulo \textit{Minimal\_Video\_Resolution} con la misma latencia base de 0 ms y jitter mínimo. El comando usado ha sido el siguiente:

\begin{lstlisting}[language=bash,basicstyle=\ttfamily\scriptsize]
python minimal_video_resolution.py -a 192.168.0.58 --show_video --show_stats -z 12 \\
-w 350 -g 250
\end{lstlisting}
Donde \verb|-a| indica la dirección IP destino, \verb|--show_video| indica que se active la transmisión de vídeo, \verb|--show_stats| indica que se muestren las estadísticas de la transmisión, \verb|-z| indica el número de fotogramas por segundo (FPS), \verb|-w| indica el ancho del vídeo y \verb|-g| indica la altura del vídeo.
\vspace{\baselineskip}

\begin{lstlisting}[language=bash,basicstyle=\ttfamily\tiny]
         |  AUDIO (msg)  |  VIDEO (msg)  |  AUDIO (kbps)   |  VIDEO (kbps)   |     CPU (%) 
   Cycle |  Sent  Recv   |  Sent  Recv   |   Sent   Recv   |   Sent   Recv   | Program System
================================================================================================
       1 |   27    27    |  188   183    |   879    879    |  2091   2038    |  21      0       
       2 |   37    37    |    0     0    |  1209   1209    |     0      0    |  43     78       
       3 |   27    27    |  443   435    |   835    835    |  4681   4597    |  42     80       
       4 |   29    29    |  840   835    |   944    944    |  9336   9281    |  35     76       
       5 |   32    32    | 1160  1112    |  1016   1016    | 12572  12054    |  45     68       
       6 |   39    39    | 1128  1121    |  1274   1274    | 12583  12508    |  48     75       
       7 |   36    36    |  910   871    |  1168   1168    | 10087   9652    |  47     73       
       8 |   34    34    |  979   930    |  1099   1099    | 10805  10266    |  39     73       
       9 |   31    31    | 1067  1021    |  1014   1014    | 11919  11404    |  32     73       
      10 |   35    35    |  905   836    |  1112   1112    |  9817   9067    |  28     79       
      11 |   31    31    | 1101  1082    |  1009   1009    | 12233  12024    |  39     71       
      12 |    0     0    |  113   108    |     0      0    |  1259   1206    |   1     34       
   Cycle |  Sent  Recv   |  Sent  Recv   |   Sent   Recv   |   Sent   Recv   | Program System
         |  AUDIO (msg)  |  VIDEO (msg)  |  AUDIO (kbps)   |  VIDEO (kbps)   |     CPU (%) 
===========================================================================================
Video application stopped.

=== Global bandwidth statistics ===
Audio sent:       933.82 kbps
Audio received:   933.82 kbps
Video sent:       7866.30 kbps
Video received:   7599.87 kbps
Total time:       12.6 s
=====================================

=== FPS Statistics ===
Target FPS:       12.0
Average real FPS: 5.2
FPS efficiency:   43.0%
======================

=== Resolution Statistics ===
Target resolution: 350x250
Actual resolution: 352x288
Average rescaling time: 3.06 ms
Performance impact:     3.7%
=============================

=== Camera compatible resolutions ===
  1. 320x240
  2. 352x288 * SELECTED
  3. 640x360
  4. 640x480
  5. 800x600
  6. 1024x768
  7. 1280x720
  8. 1280x1024
  9. 1366x768
  10. 1600x900
  11. 1920x1080
  12. 2560x1440
  13. 3840x2160

Camera device: /dev/video0
======================================
Program terminated.
QObject::killTimer: Timers cannot be stopped from another thread
QObject::~QObject: Timers cannot be stopped from another thread
\end{lstlisting}
\vspace{\baselineskip}

\newpage
La imagen de la Figura \ref{fig:minimal_video_resolution_0ms} es una captura del vídeo recibido en esta prueba.
\begin{center}
  \includegraphics[width = 0.7\textwidth]{images/VideoRecibido4.3.png}
  \captionof{figure}{Vídeo recibido en \textit{Minimal\_Video\_Resolution} con latencia base de 0 ms.}
  \label{fig:minimal_video_resolution_0ms}
\end{center}

\newpage

Ahora, se procederá a extraer las conclusiones correspondientes a las pruebas realizadas con una latencia base de 0 ms y jitter mínimo.
\vspace{\baselineskip}

\textbf{Análisis de \textit{Minimal\_Video} con jitter mínimo (latencia 0ms):}
\vspace{\baselineskip}

Al observar el comportamiento de \textit{Minimal\_Video} bajo condiciones de latencia base cero y jitter mínimo, se aprecia una transmisión estable y fluida. Las estadísticas globales muestran que el módulo envía audio a una tasa de 882.48 kbps (recibiendo 799.36 kbps) y vídeo a 6587.96 kbps (recibiendo 6353.34 kbps). Esto significa un porcentaje de recepción aproximado del 90.6\% para el audio y del 96.4\% para el vídeo, indicando una transmisión muy eficiente con pocas pérdidas, que se podrían atribuir a factores inherentes al sistema y no al jitter de la red en este escenario ideal.
\vspace{\baselineskip}

La ausencia de una latencia de red grande y tener un jitter mínimo, permite que los ciclos de envío y recepción estén bien sincronizados, como se observa en las estadísticas por ciclo, donde se mantiene un flujo constante de mensajes y una alta tasa de recepción en la mayoría de los ciclos. El uso de CPU del programa oscila generalmente entre un 20-40\% (con picos y valles), lo que indica un procesamiento manejable. La experiencia de usuario resultante es notablemente buena, con una comunicación clara y continua.

\vspace{\baselineskip}

\textbf{Análisis de \textit{Minimal\_Video\_FPS} con jitter mínimo (latencia 0ms):}
\vspace{\baselineskip}

Para este módulo, bajo condiciones de red ideales, las estadísticas globales muestran una transmisión de audio de 955.12 kbps (recibiendo 911.37 kbps) y de vídeo de 7834.12 kbps (recibiendo 7395.49 kbps). Estos valores representan un porcentaje de recepción del 95.4\% para el audio y del 94.4\% para el vídeo.


Lo más relevante son las ``Estadísticas de FPS'', que muestran un promedio real de 5.4 FPS frente al objetivo de 12 FPS, resultando en una eficiencia del 45.2\%. Este dato es significativo, pues incluso en condiciones de red ideales (jitter mínimo), el rendimiento de FPS no alcanza el objetivo, aunque es el mejor valor de eficiencia obtenido hasta ahora para este módulo. Esto sugiere que, aunque la ausencia de jitter mejora el rendimiento, las limitaciones principales para alcanzar los 12 FPS no radican únicamente en la variabilidad de la latencia de red, sino también en otros factores como el procesamiento del vídeo o la captura de la cámara.
\vspace{\baselineskip}

\textbf{Análisis de \textit{Minimal\_Video\_Resolution} con jitter mínimo (latencia 0ms):}
\vspace{\baselineskip}

Este módulo, combinando control de FPS y resolución reescalada a 352x288, muestra un rendimiento muy bueno bajo condiciones de jitter mínimo. Los datos globales indican una transmisión de audio de 933.82 kbps (recibiendo 933.82 kbps, es decir, el 100\%) y de vídeo de 7866.30 kbps (recibiendo 7599.87 kbps, un 96.6\%), siendo este el porcentaje de recepción de vídeo más alto entre los tres módulos en esta condición.
\vspace{\baselineskip}

Las ``Estadísticas de FPS'' revelan un promedio de 5.2 FPS reales frente al objetivo de 12 FPS, alcanzando una eficiencia del 43.0\%. Si bien sigue estando por debajo del objetivo, es un valor de eficiencia alto y comparable al del módulo \textit{Minimal\_Video\_FPS}. El ``Tiempo de reescalado promedio'' es muy bajo, 3.06 ms, con un impacto en rendimiento del 3.7\%. Estos valores indican que, sin las perturbaciones del jitter, el sistema opera eficientemente, aunque aún existen algunos cuellos de botella para alcanzar los 12 FPS.
\vspace{\baselineskip}

\textbf{Conclusiones para jitter mínimo (latencia 0ms):}

Las pruebas con latencia base cero y jitter mínimo nos muestran claramente el rendimiento de los módulos sin interferencias significativas de la red:

\begin{itemize}
\item \textbf{Transmisión altamente eficiente:} Los tres módulos logran tasas de transmisión y recepción muy altas, con porcentajes de recepción de vídeo que varían entre el 94.4\% y el 96.6\%, y para audio entre el 90.6\% y el 100\%. \textit{Minimal\_Video\_Resolution} destaca con la mejor recepción global de datos.
\item \textbf{Limitaciones para FPS persisten sin jitter de red:} A pesar de las condiciones de red ideales, los módulos con control de FPS (\textit{Minimal\_Video\_FPS} y \textit{Minimal\_Video\_Resolution}) no logran el objetivo de 12 FPS, alcanzando eficiencias del 45.2\% (5.4 FPS) y 43.0\% (5.2 FPS) respectivamente. Esto confirma que existen limitaciones fundamentales en otros componentes del sistema (capacidad de procesamiento de la cámara, eficiencia de la codificación/captura a esa tasa de fotogramas, o la implementación del control de FPS) que no están relacionadas con la variabilidad de la latencia de la red.
\item \textbf{Rendimiento óptimo de los módulos:} En ausencia de jitter, el rendimiento de los módulos es el mejor observado. \textit{Minimal\_Video\_Resolution} demuestra que el reescalado de resolución puede ser muy eficiente (3.7\% de impacto en CPU) y no compromete significativamente la tasa de FPS en comparación con \textit{Minimal\_Video\_FPS} en estas condiciones.
\end{itemize}

Estas pruebas establecen muestran un rendimiento óptimo. Demuestran que, si bien eliminar el jitter mejora la estabilidad y la eficiencia de la transmisión, alcanzar tasas de fotogramas más altas requiere optimizaciones adicionales en el procesamiento del vídeo, la captura de la cámara, y posiblemente en la propia lógica de control de FPS de los módulos, independientemente de las condiciones de la red.
\newpage

\textbf{Pruebas con jitter moderado (Latencia de 100 ms):}
\vspace{\baselineskip}

Ahora, se procederá a realizar las pruebas introduciendo una latencia base de 100 ms, lo cual, en la práctica, también implica una mayor variación en los tiempos de llegada de los paquetes (jitter).
\vspace{\baselineskip}

El comando usado para \textit{Minimal\_Video} ha sido el siguiente:
\begin{lstlisting}[language=bash]
python minimal_video.py -a 192.168.0.58 --show_video --show_stats
\end{lstlisting}
Donde \verb|-a| es la dirección IP del dispositivo con el que se va a comunicar el módulo, \verb|--show_video| es la opción para mostrar el vídeo en tiempo real y \verb|--show_stats| es la opción para mostrar las estadísticas de la red.
\vspace{\baselineskip}

\begin{lstlisting}[language=bash,basicstyle=\ttfamily\tiny]
         |  AUDIO (msg)  |  VIDEO (msg)  |  AUDIO (kbps)   |  VIDEO (kbps)   |     CPU (%) 
   Cycle |  Sent  Recv   |  Sent  Recv   |   Sent   Recv   |   Sent   Recv   | Program System
================================================================================================
       1 |   29    29    |  165   119    |   948    948    |  1842   1330    |  21      0       
       2 |   33    33    |    0     0    |  1079   1079    |     0      0    |  38     76       
       3 |   35    26    |  445   372    |  1099    817    |  4775   3991    |  38     78       
       4 |   27    27    | 1515  1403    |   826    826    | 15822  14654    |  31     65       
       5 |   35    35    | 1076   868    |  1146   1146    | 12031   9705    |  40     70       
       6 |   32    32    | 1882  1780    |  1001   1001    | 20104  19018    |  40     71       
       7 |   33    33    | 1140   976    |  1054   1054    | 12433  10644    |  30     68       
       8 |   31    24    |  678   458    |  1011    783    |  7555   5105    |  43     70       
       9 |   31    31    | 1725  1584    |   999    999    | 18989  17437    |  28     67       
      10 |   37    37    | 2019  1924    |  1209   1209    | 22526  21467    |  46     74       
      11 |   35    35    | 1539  1387    |  1144   1144    | 17184  15485    |  45     70       
      12 |   29    29    | 1101   983    |   947    947    | 12274  10959    |  34     70       
      13 |    9     9    |  495   458    |   294    294    |  5521   5110    |   6     72       
   Cycle |  Sent  Recv   |  Sent  Recv   |   Sent   Recv   |   Sent   Recv   | Program System
         |  AUDIO (msg)  |  VIDEO (msg)  |  AUDIO (kbps)   |  VIDEO (kbps)   |     CPU (%) 
===========================================================================================
Video application stopped.

=== Global bandwidth statistics ===
Audio sent:       962.34 kbps
Audio received:   923.46 kbps
Video sent:       11432.49 kbps
Video received:   10215.23 kbps
Total time:       13.5 s
=====================================
Program terminated.
QObject::killTimer: Timers cannot be stopped from another thread
QObject::~QObject: Timers cannot be stopped from another thread
\end{lstlisting}
\vspace{\baselineskip}

\newpage
La imagen de la Figura \ref{fig:minimal_video_100ms} es una captura del vídeo recibido en esta prueba.
\begin{center}
  \includegraphics[width = 0.7\textwidth]{images/VideoRecibido5.1.png}
  \captionof{figure}{Vídeo recibido en \textit{Minimal\_Video} con latencia base de 100 ms.}
  \label{fig:minimal_video_100ms}
\end{center}

\newpage

Ahora, ejecutaremos el módulo \textit{Minimal\_Video\_FPS} con una latencia base de 100 ms y el jitter asociado. El comando usado ha sido el siguiente:

\begin{lstlisting}[language=bash, basicstyle=\ttfamily\scriptsize]
    python minimal_video_fps.py -a 192.168.0.58 --show_video --show_stats -z 12
\end{lstlisting}
Donde \verb|-a| es la dirección IP destino, \verb|--show_video| es para que se active la transmisión de vídeo, \verb|--show_stats| es para que se muestre la información de estadísticas y \verb|-z| es el numero de fotogramas por segundo (FPS) que se desea enviar. En este caso, se ha configurado para mostrar el vídeo a 12 FPS.
\vspace{\baselineskip}

\begin{lstlisting}[language=bash,basicstyle=\ttfamily\tiny]
        |  AUDIO (msg)  |  VIDEO (msg)  |  AUDIO (kbps)   |  VIDEO (kbps)   |     CPU (%) 
   Cycle |  Sent  Recv   |  Sent  Recv   |   Sent   Recv   |   Sent   Recv   | Program System
================================================================================================
       1 |   29    29    |  165   145    |   946    946    |  1839   1618    |  28      0       
       2 |   33    33    |    0     0    |  1061   1061    |     0      0    |  50     76       
       3 |   22    22    |  330   278    |   713    713    |  3652   3076    |  40     77       
       4 |   20    20    | 1316  1250    |   631    631    | 14183  13473    |  24     77       
       5 |   33    33    | 1252  1151    |  1079   1079    | 13986  12859    |  36     73       
       6 |   35    35    | 1799  1754    |  1143   1143    | 20070  19567    |  41     74       
       7 |   29    29    | 1903  1843    |   947    947    | 21215  20550    |  40     69       
       8 |   34    34    | 1650  1630    |  1080   1080    | 17906  17695    |  31     72       
       9 |   40    40    | 1951  1945    |  1295   1295    | 21578  21512    |  46     68       
      10 |   31    31    | 1815  1800    |  1010   1010    | 20207  20042    |  46     73       
      11 |   33    33    | 1980  1921    |  1073   1073    | 21985  21331    |  44     70       
      12 |   15    15    |  990   983    |   479    479    | 10803  10728    |  17     51       
   Cycle |  Sent  Recv   |  Sent  Recv   |   Sent   Recv   |   Sent   Recv   | Program System
         |  AUDIO (msg)  |  VIDEO (msg)  |  AUDIO (kbps)   |  VIDEO (kbps)   |     CPU (%) 
===========================================================================================
Video application stopped.

=== Global bandwidth statistics ===
Audio sent:       936.37 kbps
Audio received:   936.37 kbps
Video sent:       13682.08 kbps
Video received:   13276.25 kbps
Total time:       12.4 s
=====================================

=== FPS Statistics ===
Target FPS:       12.0
Average real FPS: 11.4
FPS efficiency:   95.0%
======================
Program terminated.
QObject::killTimer: Timers cannot be stopped from another thread
QObject::~QObject: Timers cannot be stopped from another thread
\end{lstlisting}
\vspace{\baselineskip}

\newpage
La imagen de la Figura \ref{fig:minimal_video_fps_100ms} es una captura del vídeo recibido en esta prueba.
\begin{center}
  \includegraphics[width = 0.7\textwidth]{images/VideoRecibido5.2.png}
  \captionof{figure}{Vídeo recibido en \textit{Minimal\_Video\_FPS} con latencia base de 100 ms.}
  \label{fig:minimal_video_fps_100ms}
\end{center}

\newpage

Finalmente, se ejecutará el módulo \textit{Minimal\_Video\_Resolution} con una latencia base de 100 ms y el jitter asociado. El comando usado ha sido el siguiente:

\begin{lstlisting}[language=bash,basicstyle=\ttfamily\scriptsize]
python minimal_video_resolution.py -a 192.168.0.58 --show_video --show_stats -z 12 \\
-w 350 -g 250
\end{lstlisting}
Donde \verb|-a| indica la dirección IP destino, \verb|--show_video| indica que se active la transmisión de vídeo, \verb|--show_stats| indica que se muestren las estadísticas de la transmisión, \verb|-z| indica el número de fotogramas por segundo (FPS), \verb|-w| indica el ancho del vídeo y \verb|-g| indica la altura del vídeo.
\vspace{\baselineskip}

\begin{lstlisting}[language=bash,basicstyle=\ttfamily\tiny]
         |  AUDIO (msg)  |  VIDEO (msg)  |  AUDIO (kbps)   |  VIDEO (kbps)   |     CPU (%) 
   Cycle |  Sent  Recv   |  Sent  Recv   |   Sent   Recv   |   Sent   Recv   | Program System
================================================================================================
       1 |   27    27    |  188    86    |   883    883    |  2100    962    |  23      0       
       2 |   27    27    |  250   249    |   829    829    |  2623   2612    |  30     78       
       3 |   22    22    |  877   787    |   707    707    |  9626   8642    |  26     75       
       4 |   29    29    | 1504  1405    |   946    946    | 16751  15650    |  31     68       
       5 |   26    26    | 1645  1580    |   817    817    | 17650  16952    |  32     68       
       6 |   33    33    | 1914  1866    |  1055   1055    | 20902  20377    |  28     68       
       7 |   36    36    | 1840  1775    |  1176   1176    | 20533  19807    |  36     71       
       8 |   30    30    | 1874  1811    |   971    971    | 20721  20024    |  38     73       
       9 |   35    35    |  763   561    |  1141   1141    |  8494   6250    |  30     72       
      10 |   31    31    | 1880  1806    |  1012   1012    | 20971  20144    |  37     71       
      11 |   37    31    |  992   813    |  1211   1015    | 11089   9088    |  45     69       
      12 |   11     3    |  324   158    |   359     98    |  3614   1762    |  17     56       
   Cycle |  Sent  Recv   |  Sent  Recv   |   Sent   Recv   |   Sent   Recv   | Program System
         |  AUDIO (msg)  |  VIDEO (msg)  |  AUDIO (kbps)   |  VIDEO (kbps)   |     CPU (%) 
===========================================================================================
Video application stopped.

=== Global bandwidth statistics ===
Audio sent:       896.90 kbps
Audio received:   860.39 kbps
Video sent:       12506.10 kbps
Video received:   11479.56 kbps
Total time:       12.6 s
=====================================

=== FPS Statistics ===
Target FPS:       12.0
Average real FPS: 6.6
FPS efficiency:   55.4%
======================

=== Resolution Statistics ===
Target resolution: 350x250
Actual resolution: 352x288
Average rescaling time: 5.67 ms
Performance impact:     6.8%
=============================

=== Camera compatible resolutions ===
  1. 320x240
  2. 352x288 * SELECTED
  3. 640x360
  4. 640x480
  5. 800x600
  6. 1024x768
  7. 1280x720
  8. 1280x1024
  9. 1366x768
  10. 1600x900
  11. 1920x1080
  12. 2560x1440
  13. 3840x2160

Camera device: /dev/video0
======================================
Program terminated.
QObject::killTimer: Timers cannot be stopped from another thread
QObject::~QObject: Timers cannot be stopped from another thread
\end{lstlisting}

\newpage
La imagen de la Figura \ref{fig:minimal_video_resolution_100ms} es una captura del vídeo recibido en esta prueba.
\begin{center}
  \includegraphics[width = 0.7\textwidth]{images/VideoRecibido5.3.png}
  \captionof{figure}{Vídeo recibido en \textit{Minimal\_Video\_Resolution} con latencia base de 100 ms.}
  \label{fig:minimal_video_resolution_100ms}
\end{center}

\newpage

Ahora, se procederá a extraer las conclusiones correspondientes a las pruebas realizadas con una latencia base de 100 ms y el jitter asociado.
\vspace{\baselineskip}

\textbf{Análisis de \textit{Minimal\_Video} con jitter moderado y latencia de 100ms:}
\vspace{\baselineskip}

Al observar el comportamiento de \textit{Minimal\_Video} bajo una latencia base de 100ms y el jitter relativo a esta condición de red, se muestra un rendimiento robusto en términos de volumen de datos, pero con la variabilidad esperada debido al jitter. Las estadísticas globales muestran que el módulo envía audio a 962.34 kbps recibiendo 923.46 kbps (un 95.9\%) y para vídeo envía 11432.49 kbps recibiendo 10215.23 kbps (un 89.3\%). Esta diferencia entre envío y recepción no es grave, lo que sugiere que la mayoría de los datos llegan, pero la latencia y el jitter afecta levemente a la fluidez.
\vspace{\baselineskip}

Analizando los ciclos, se puede observar que hay ciertas fluctuaciones en las tasas de recepción. Por ejemplo, en el ciclo 3, la recepción de audio es menor que la enviada, y en el ciclo 8, tanto audio como vídeo muestran una caída en la recepción respecto al envío. Esto es causado por el jitter, donde los paquetes pueden llegar con retrasos variables, afectando la contabilidad en cada ciclo. No obstante, la cantidad total de datos transmitidos es considerable, y la comunicación, aunque con menor fluidez y con el retraso de 100ms, se mantiene. La experiencia del usuario se ve ligeramente afectada por el retraso y la posible falta de sincronización perfecta debido a la llegada irregular de los paquetes.

\vspace{\baselineskip}

\textbf{Análisis de \textit{Minimal\_Video\_FPS} con jitter moderado y latencia de 100ms:}
\vspace{\baselineskip}

Para este módulo, que intenta mantener una tasa constante de 12 FPS, la latencia base de 100ms y el jitter asociado sorprendentemente muestran un rendimiento de FPS muy alto. Las estadísticas globales muestran un envío de audio de 936.37 kbps (recibiendo el 100\%) y de vídeo de 13682.08 kbps (recibiendo 13276.25 kbps, un 97.0\%). Estos porcentajes de recepción son excelentes, indicando muy pocas pérdidas.
\vspace{\baselineskip}

La sección de ``Estadísticas de FPS'' es notable: el módulo consigue un promedio de 11.4 FPS reales frente al objetivo de 12 FPS, alcanzando una eficiencia del 95.0\%. Este es el mejor rendimiento de FPS observado en todas las pruebas hasta ahora. Este resultado sugiere que, en este caso particular, la latencia de 100ms podría estar actuando como un buffer que ayuda al sistema a gestionar y enviar los fotogramas de manera más consistente hacia el objetivo de 12 FPS, o que el mecanismo de control de FPS es particularmente efectivo bajo estas condiciones. La llegada de paquetes, aunque con jitter, no impide que el módulo se acerque mucho a la tasa objetivo. La experiencia del usuario es muy fluida en términos de fotogramas por segundo, aunque con el retraso perceptible de 100ms.

\vspace{\baselineskip}

\textbf{Análisis de \textit{Minimal\_Video\_Resolution} con jitter moderado y latencia de 100ms:}
\vspace{\baselineskip}

Este módulo, combinando control de FPS y resolución reescalada, también muestra un buen rendimiento bajo una latencia base de 100ms y el jitter asociado. Los datos globales indican una transmisión de audio de 896.90 kbps (recibiendo 860.39 kbps, un 95.9\%) y de vídeo de 12506.10 kbps (recibiendo 11479.56 kbps, un 91.8\%). Estos porcentajes de recepción son altos, indicando una transmisión de datos eficiente.
\vspace{\baselineskip}

Las ``Estadísticas de FPS'' revelan un promedio de 6.6 FPS reales frente al objetivo de 12 FPS, con una eficiencia del 55.4\%. Aunque no tan espectacular como \textit{Minimal\_Video\_FPS}, sigue siendo un buen resultado, superior al obtenido en condiciones de jitter mínimo. El ``Tiempo de reescalado promedio'' es de 5.67 ms, con un impacto en rendimiento del 6.8\%, valores manejables. El módulo logra una buena transmisión de datos y una tasa de fotogramas decente, lo que sugiere una buena adaptación a las condiciones de jitter y latencia.

\vspace{\baselineskip}

\textbf{Conclusiones para latencia de 100ms con jitter moderado:}

La introducción de una latencia base de 100ms y un jitter moderado revela un comportamiento sorprendentemente positivo, especialmente para el módulo \textit{Minimal\_Video\_FPS}:

\begin{itemize}
\item \textbf{Altas tasas de recepción de datos a pesar del jitter:} Todos los módulos mantienen altas tasas de recepción de datos (audio >95\%, vídeo >89\%), lo que indica que la pérdida de paquetes no es el principal problema bajo estas condiciones, a pesar del jitter.
\item \textbf{Rendimiento excepcional de FPS para \textit{Minimal\_Video\_FPS}:} Este módulo alcanza una eficiencia de FPS del 95.0\% (11.4 FPS reales), el mejor resultado de todas las pruebas. Esto podría indicar que la latencia introducida actúa como un buffer para su mecanismo de control de FPS.
\item \textbf{Buen rendimiento general de \textit{Minimal\_Video\_Resolution}:} Este módulo también se beneficia, alcanzando una eficiencia de FPS del 55.4\% (6.6 FPS reales) y manteniendo una alta integridad de datos.
\item \textbf{El jitter no resulta en pérdidas masivas:} Aunque el jitter está presente, su principal impacto parece ser la variabilidad en los tiempos de llegada y las fluctuaciones en las tasas de kbps por ciclo, más que una pérdida catastrófica de paquetes. La latencia de 100ms es el factor dominante en el retraso de la comunicación.
\item \textbf{Posible efecto de "buffering" por la latencia:} La latencia constante de 100ms podría estar permitiendo que los mecanismos internos de los módulos (especialmente el control de FPS) manejen mejor la secuenciación y el envío de paquetes, incluso con la variabilidad del jitter, al proporcionar más tiempo para la preparación y el envío de los fotogramas.
\end{itemize}

Estas conclusiones sugieren que una latencia moderada, incluso con jitter, no es relativamente perjudicial para todas las métricas de rendimiento, y en el caso de \textit{Minimal\_Video\_FPS}, parece incluso mejorar la capacidad de alcanzar la tasa de fotogramas objetivo. No obstante, la experiencia del usuario siempre estará marcada por el retraso de 100ms en la interacción.

\newpage

\textbf{Pruebas con jitter grave y latencia de 250ms}
\vspace{\baselineskip}

Ahora, se procederá a realizar las pruebas con una latencia base de 250 ms, lo que implica un nivel de jitter aún más pronunciado.
\vspace{\baselineskip}

El comando usado para \textit{Minimal\_Video} ha sido el siguiente:

\begin{lstlisting}[language=bash]
python minimal_video.py -a 192.168.0.58 --show_video --show_stats
\end{lstlisting}
Donde \verb|-a| es la dirección IP del dispositivo con el que se va a comunicar el módulo, \verb|--show_video| es la opción para mostrar el vídeo en tiempo real y \verb|--show_stats| es la opción para mostrar las estadísticas de la red.
\vspace{\baselineskip}

\begin{lstlisting}[language=bash,basicstyle=\ttfamily\tiny]
         |  AUDIO (msg)  |  VIDEO (msg)  |  AUDIO (kbps)   |  VIDEO (kbps)   |     CPU (%) 
   Cycle |  Sent  Recv   |  Sent  Recv   |   Sent   Recv   |   Sent   Recv   | Program System
================================================================================================
       1 |   25    14    |  165     8    |   815    456    |  1836     89    |  33      0       
       2 |   35    35    |    0     0    |  1140   1140    |     0      0    |  36     74       
       3 |   33    33    |  368   307    |  1072   1072    |  4083   3405    |  35     73       
       4 |   33    33    | 1207  1116    |  1078   1078    | 13468  12452    |  33     72       
       5 |   35    35    | 1417  1300    |  1136   1136    | 15703  14407    |  35     71       
       6 |   35    35    | 1793  1689    |  1143   1143    | 20000  18843    |  35     72       
       7 |   31    31    | 1269  1123    |  1012   1012    | 14157  12527    |  39     70       
       8 |   33    33    | 1086   941    |  1079   1079    | 12127  10510    |  25     73       
       9 |   32    32    | 1621  1500    |  1045   1045    | 18073  16725    |  36     69       
      10 |   35    35    | 1342  1223    |  1112   1112    | 14568  13276    |  28     72       
      11 |   25    25    | 1226  1065    |   814    814    | 13635  11844    |  38     73       
      12 |   32    32    | 1244  1135    |  1004   1004    | 13329  12162    |  34     69       
      13 |    7     7    |  757   696    |   227    227    |  8409   7730    |  11     55       
   Cycle |  Sent  Recv   |  Sent  Recv   |   Sent   Recv   |   Sent   Recv   | Program System
         |  AUDIO (msg)  |  VIDEO (msg)  |  AUDIO (kbps)   |  VIDEO (kbps)   |     CPU (%) 
===========================================================================================
Video application stopped.

=== Global bandwidth statistics ===
Audio sent:       956.82 kbps
Audio received:   929.90 kbps
Video sent:       11274.19 kbps
Video received:   10111.51 kbps
Total time:       13.4 s
=====================================
Program terminated.
QObject::killTimer: Timers cannot be stopped from another thread
QObject::~QObject: Timers cannot be stopped from another thread
\end{lstlisting}
\vspace{\baselineskip}

\newpage
La imagen de la Figura \ref{fig:minimal_video_250ms} es una captura del vídeo recibido en esta prueba.
\begin{center}
  \includegraphics[width = 0.7\textwidth]{images/VideoRecibido6.1.png}
  \captionof{figure}{Vídeo recibido en \textit{Minimal\_Video} con latencia base de 250 ms.}
  \label{fig:minimal_video_250ms}
\end{center}

\newpage


Ahora, se ejecutará el módulo \textit{Minimal\_Video\_FPS} con una latencia base de 250 ms y el jitter severo asociado. El comando usado ha sido el siguiente:

\begin{lstlisting}[language=bash, basicstyle=\ttfamily\scriptsize]
    python minimal_video_fps.py -a 192.168.0.58 --show_video --show_stats -z 12
\end{lstlisting}
Donde \verb|-a| es la dirección IP destino, \verb|--show_video| es para que se active la transmisión de vídeo, \verb|--show_stats| es para que se muestre la información de estadísticas y \verb|-z| es el numero de fotogramas por segundo (FPS) que se desea enviar. En este caso, se ha configurado para mostrar el vídeo a 12 FPS.
\vspace{\baselineskip}

\begin{lstlisting}[language=bash,basicstyle=\ttfamily\tiny]
         |  AUDIO (msg)  |  VIDEO (msg)  |  AUDIO (kbps)   |  VIDEO (kbps)   |     CPU (%) 
   Cycle |  Sent  Recv   |  Sent  Recv   |   Sent   Recv   |   Sent   Recv   | Program System
================================================================================================
       1 |   31    28    |  165    93    |  1009    911    |  1833   1034    |  23      0       
       2 |   38    38    |    0     0    |  1232   1232    |     0      0    |  44     76       
       3 |   35    35    |  255   240    |  1143   1143    |  2846   2676    |  40     79       
       4 |   26    26    |  900   779    |   850    850    | 10050   8701    |  35     70       
       5 |   33    33    | 1341  1263    |  1065   1065    | 14783  13922    |  34     72       
       6 |   31    31    | 1464  1362    |   979    979    | 15787  14686    |  36     62       
       7 |   28    28    | 1143  1132    |   915    915    | 12762  12640    |  36     51       
       8 |   16    16    |  507   493    |   520    520    |  5625   5471    |  30     29       
       9 |   39    39    | 1247  1063    |  1277   1277    | 13941  11882    |  43     73       
      10 |   37    37    | 1192  1075    |  1211   1211    | 13325  12020    |  53     66       
      11 |   34    34    | 1519  1399    |  1068   1068    | 16297  15010    |  40     72       
      12 |   34    34    | 1761  1690    |  1086   1086    | 19212  18436    |  48     67       
      13 |    4     4    |  377   378    |   130    130    |  4200   4213    |   9     41       
   Cycle |  Sent  Recv   |  Sent  Recv   |   Sent   Recv   |   Sent   Recv   | Program System
         |  AUDIO (msg)  |  VIDEO (msg)  |  AUDIO (kbps)   |  VIDEO (kbps)   |     CPU (%) 
===========================================================================================
Video application stopped.

=== Global bandwidth statistics ===
Audio sent:       945.70 kbps
Audio received:   938.35 kbps
Video sent:       9929.23 kbps
Video received:   9173.34 kbps
Total time:       13.4 s
=====================================

=== FPS Statistics ===
Target FPS:       12.0
Average real FPS: 8.3
FPS efficiency:   69.1%
======================
Program terminated.
QObject::killTimer: Timers cannot be stopped from another thread
QObject::~QObject: Timers cannot be stopped from another thread
\end{lstlisting}

\newpage
La imagen de la Figura \ref{fig:minimal_video_fps_250ms} es una captura del vídeo recibido en esta prueba.
\begin{center}
  \includegraphics[width = 0.7\textwidth]{images/VideoRecibido6.2.png}
  \captionof{figure}{Vídeo recibido en \textit{Minimal\_Video\_FPS} con latencia base de 250 ms.}
  \label{fig:minimal_video_fps_250ms}
\end{center}

\newpage


Ahora, se procederá a ejecutar el módulo \textit{Minimal\_Video\_Resolution} con una latencia base de 250 ms y el jitter severo asociado. El comando usado ha sido el siguiente:

\begin{lstlisting}[language=bash,basicstyle=\ttfamily\scriptsize]
python minimal_video_resolution.py -a 192.168.0.58 --show_video --show_stats -z 12 \\
-w 350 -g 250
\end{lstlisting}
Donde \verb|-a| indica la dirección IP destino, \verb|--show_video| indica que se active la transmisión de vídeo, \verb|--show_stats| indica que se muestren las estadísticas de la transmisión, \verb|-z| indica el número de fotogramas por segundo (FPS), \verb|-w| indica el ancho del vídeo y \verb|-g| indica la altura del vídeo.
\vspace{\baselineskip}

\begin{lstlisting}[language=bash,basicstyle=\ttfamily\tiny]
         |  AUDIO (msg)  |  VIDEO (msg)  |  AUDIO (kbps)   |  VIDEO (kbps)   |     CPU (%) 
   Cycle |  Sent  Recv   |  Sent  Recv   |   Sent   Recv   |   Sent   Recv   | Program System
================================================================================================
       1 |   27    27    |  188   188    |   883    883    |  2100   2100    |  22    100       
       2 |   31    26    |    0     0    |  1000    839    |     0      0    |  41     77       
       3 |   28    24    |  376   283    |   911    781    |  4178   3146    |  35     74       
       4 |   37    37    | 1245  1074    |  1190   1190    | 13670  11791    |  40     70       
       5 |   35    35    | 1032   816    |  1144   1144    | 11520   9110    |  43     66       
       6 |   33    33    | 1416  1291    |  1072   1072    | 15711  14323    |  40     71       
       7 |   29    29    | 1513  1430    |   948    948    | 16888  15963    |  31     68       
       8 |   35    35    | 1880  1791    |  1145   1145    | 21006  20013    |  46     71       
       9 |   17    17    | 1505  1463    |   551    551    | 16668  16205    |  39     70       
      10 |   38    38    |  713   453    |  1240   1240    |  7948   5049    |  46     61       
      11 |   28    28    |  965   850    |   915    915    | 10767   9485    |  36     67       
      12 |   33    33    | 1149  1006    |  1064   1064    | 12648  11072    |  36     72       
      13 |   13    13    |  919   858    |   424    424    | 10248   9567    |  17     70       
   Cycle |  Sent  Recv   |  Sent  Recv   |   Sent   Recv   |   Sent   Recv   | Program System
         |  AUDIO (msg)  |  VIDEO (msg)  |  AUDIO (kbps)   |  VIDEO (kbps)   |     CPU (%) 
===========================================================================================
Video application stopped.

=== Global bandwidth statistics ===
Audio sent:       932.17 kbps
Audio received:   910.32 kbps
Video sent:       10691.15 kbps
Video received:   9532.74 kbps
Total time:       13.5 s
=====================================

=== FPS Statistics ===
Target FPS:       12.0
Average real FPS: 6.0
FPS efficiency:   50.3%
======================

=== Resolution Statistics ===
Target resolution: 350x250
Actual resolution: 352x288
Average rescaling time: 6.43 ms
Performance impact:     7.7%
=============================

=== Camera compatible resolutions ===
  1. 320x240
  2. 352x288 * SELECTED
  3. 640x360
  4. 640x480
  5. 800x600
  6. 1024x768
  7. 1280x720
  8. 1280x1024
  9. 1366x768
  10. 1600x900
  11. 1920x1080
  12. 2560x1440
  13. 3840x2160

Camera device: /dev/video0
======================================
Program terminated.
QObject::killTimer: Timers cannot be stopped from another thread
QObject::~QObject: Timers cannot be stopped from another thread
\end{lstlisting}
\vspace{\baselineskip}

\newpage
La imagen de la Figura \ref{fig:minimal_video_resolution_250ms} es una captura del vídeo recibido en esta prueba.
\begin{center}
  \includegraphics[width = 0.7\textwidth]{images/VideoRecibido6.3.png}
  \captionof{figure}{Vídeo recibido en \textit{Minimal\_Video\_Resolution} con latencia base de 250 ms.}
  \label{fig:minimal_video_resolution_250ms}
\end{center}
\newpage


Ahora, se procederá a analizar los resultados obtenidos en las pruebas realizadas con una latencia base de 250 ms y el jitter severo asociado.
\vspace{\baselineskip}

\textbf{Análisis de \textit{Minimal\_Video} con jitter alto y latencia de 250ms:}
\vspace{\baselineskip}

Al analizar el comportamiento de \textit{Minimal\_Video} bajo una latencia base de 250ms y el jitter severo, observamos un impacto notable pero no un colapso total. Las estadísticas globales muestran que el módulo envía audio a 956.82 kbps (recibiendo 929.90 kbps, un 97.2\%) y vídeo a 11274.19 kbps (recibiendo 10111.51 kbps, un 89.7\%). Estos porcentajes de recepción son bastante altos, indicando que la mayoría de los datos llegan al destino.
\vspace{\baselineskip}

Sin embargo, el análisis de los ciclos, especialmente el primero, muestra una recepción inicial de vídeo muy baja (89 kbps recibidos de 1836 kbps enviados), lo que refleja el impacto combinado de la alta latencia inicial y el jitter. Aunque los ciclos posteriores muestran una mejor recuperación en la cantidad de datos, la variabilidad asociada a un jitter severo y la alta latencia constante, perjudican significativamente la experiencia del usuario. La comunicación es percibida con un retraso considerable y una falta de fluidez debido a la llegada irregular de los paquetes, manifestándose en "tirones" y desincronizaciones.

\vspace{\baselineskip}

\textbf{Análisis de \textit{Minimal\_Video\_FPS} con jitter alto y latencia de 250ms:}
\vspace{\baselineskip}

Este módulo muestra una adaptación robusta al jitter severo y la alta latencia. Las estadísticas globales indican una transmisión de audio de 945.70 kbps (recibiendo 938.35 kbps, un 99.2\%) y de vídeo de 9929.23 kbps (recibiendo 9173.34 kbps, un 92.4\%). Estos porcentajes de recepción son excelentes, sugiriendo que el control de FPS ayuda a mitigar la pérdida de paquetes incluso en estas condiciones adversas.
\vspace{\baselineskip}

En cunato a las ``Estadísticas de FPS'' el módulo consigue mantener 8.3 FPS reales frente al objetivo de 12 FPS, alcanzando una eficiencia del 69.1\%. Para una latencia base tan alta y el jitter severo asociado, este es un rendimiento de FPS promedio muy bueno. Indica que el control de FPS se ajusta para enviar fotogramas a un ritmo que, aunque no ideal, es sostenible bajo estas condiciones extremas. Sin embargo, es importante entender que una media de 8.3 FPS bajo alto jitter no garantiza una visualización perfectamente fluida. La llegada de esos fotogramas será irregular, aunque la cantidad total transmitida sea alta. La experiencia del usuario estará marcada por el retardo de 250ms y una fluidez visual por la variabilidad en la entrega.

\vspace{\baselineskip}

\textbf{Análisis de \textit{Minimal\_Video\_Resolution} con jitter alto y latencia de 250ms:}
\vspace{\baselineskip}

Los resultados de este módulo bajo una latencia base de 250ms y jitter severo también son bastante positivos en términos de transmisión de datos y FPS promedio. Los datos globales muestran una transmisión de audio de 932.17 kbps (recibiendo 910.32 kbps, un 97.6\%) y de vídeo de 10691.15 kbps (recibiendo 9532.74 kbps, un 89.2\%).
\vspace{\baselineskip}

Las ``Estadísticas de FPS'' muestran un promedio de 6.0 FPS reales frente al objetivo de 12 FPS, alcanzando una eficiencia del 50.3\%. Aunque inferior al módulo \textit{Minimal\_Video\_FPS}, sigue siendo un resultado suficiente dadas las condiciones de red. El ``Tiempo de reescalado promedio'' de 6.43 ms (impacto del 7.7\%) es manejable. El módulo logra transmitir una cantidad significativa de datos y mantener una tasa de fotogramas promedio funcional. La combinación de control de FPS y ajuste de resolución parece ofrecer un buen compromiso para mantener la comunicación activa, aunque la fluidez está afectada por el alto jitter y la latencia.

\vspace{\baselineskip}

\textbf{Conclusiones para latencia base de 250ms con jitter alto:}

Una latencia base de 250ms, con el jitter severo asociado, establece condiciones muy difíciles, pero los módulos aguantan notablemente estas condiciones, especialmente en la cantidad de datos transmitidos y los FPS promedio:

\begin{itemize}
\item \textbf{Alta integridad de datos a pesar del jitter extremo:} Sorprendentemente, los tres módulos mantienen porcentajes de recepción de datos globales muy altos (audio >97\%, vídeo >89\%), lo que indica que la pérdida de paquetes no es el factor dominante, sino la irregularidad en su llegada.
\item \textbf{Rendimiento de FPS notable bajo condiciones adversas:} \textit{Minimal\_Video\_FPS} destaca con una eficiencia del 69.1\% (8.3 FPS reales), seguido de \textit{Minimal\_Video\_Resolution} con 50.3\% (6.0 FPS reales). Estos valores sugieren que los mecanismos de control de FPS se adaptan para mantener una tasa de envío viable incluso con un jitter muy elevado.
\item \textbf{Impacto de la latencia y el jitter en la experiencia del usuario:} Aunque se transmiten muchos datos y se logran FPS promedio considerables, la experiencia del usuario estará severamente degradada por el retraso de 250ms y la llegada errática de los fotogramas, causando una sensación de desconexión y falta de fluidez.
\item \textbf{Adaptabilidad de los módulos con control de FPS:} Los módulos \textit{Minimal\_Video\_FPS} y \textit{Minimal\_Video\_Resolution} muestran una capacidad significativa para ajustarse a un entorno de red muy adverso, priorizando mantener un flujo de fotogramas.
\item \textbf{El problema de la irregularidad:} Bajo estas condiciones, el principal problema para la calidad de la videollamada no es tanto la pérdida de paquetes (que es baja), sino la enorme variabilidad en los tiempos de llegada, lo que hace imposible una reproducción fluida y sincronizada sin mecanismos de buffering muy sofisticados y tolerantes a grandes retrasos.
\end{itemize}

Estos resultados demuestran que, si bien una latencia base alta y un jitter severo perjudican gravemente la calidad percibida de una videollamada, los módulos pueden seguir transmitiendo una cantidad considerable de datos y mantener una tasa de fotogramas promedio funcional. La clave está en la irregularidad de la entrega, que es el principal obstáculo para una experiencia de usuario satisfactoria.
\newpage

\subsubsection{Experimentos con pérdida de paquetes}

Ahora, se procederá a analizar los resultados obtenidos en las pruebas realizadas con pérdida de paquetes. 
\vspace{\baselineskip}
\begin{itemize}
  \item \textbf{Pruebas frente a pérdida de paquetes}
\end{itemize}

\textbf{Pruebas frente a pérdida de paquetes del 5\%:}
\vspace{\baselineskip}

Comenzaremos con una pérdida del 5\% de paquetes. El comando usado en \textit{Minimal\_Video} ha sido el siguiente:

\begin{lstlisting}[language=bash]
python minimal_video.py -a 192.168.0.58 --show_video --show_stats
\end{lstlisting}
Donde \verb|-a| es la dirección IP del dispositivo con el que se va a comunicar el módulo, \verb|--show_video| es la opción para mostrar el vídeo en tiempo real y \verb|--show_stats| es la opción para mostrar las estadísticas de la red.
\vspace{\baselineskip}

\begin{lstlisting}[language=bash,basicstyle=\ttfamily\tiny]
         |  AUDIO (msg)  |  VIDEO (msg)  |  AUDIO (kbps)   |  VIDEO (kbps)   |     CPU (%) 
   Cycle |  Sent  Recv   |  Sent  Recv   |   Sent   Recv   |   Sent   Recv   | Program System
================================================================================================
       1 |   23    23    |  165   150    |   748    748    |  1834   1669    |  30      0       
       2 |   37    37    |    0     0    |  1210   1210    |     0      0    |  52     78       
       3 |   33    33    |    0     0    |  1067   1067    |     0      0    |  51     76       
       4 |   25    25    |  252   252    |   815    815    |  2805   2801    |  29     77       
       5 |   30    30    |  971   933    |   980    980    | 10839  10416    |  25     76       
       6 |   30    30    | 1086  1067    |   981    981    | 12125  11914    |  36     78       
       7 |   35    27    |  472   220    |  1134    875    |  5225   2433    |  32     69       
       8 |   27    18    |  953   868    |   875    583    | 10544   9604    |  31     73       
       9 |   33    33    |  763   741    |  1078   1078    |  8510   8264    |  36     73       
      10 |   32    32    |  883   846    |  1036   1036    |  9764   9354    |  35     70       
      11 |   40    40    |  916   873    |  1307   1307    | 10224   9743    |  32     76       
      12 |   35    35    |  814   788    |  1138   1138    |  9039   8750    |  35     75       
      13 |   29    29    | 1226  1169    |   947    947    | 13673  13037    |  30     74       
      14 |    0     0    |   74    75    |     0      0    |   825    836    |   0     42       
   Cycle |  Sent  Recv   |  Sent  Recv   |   Sent   Recv   |   Sent   Recv   | Program System
         |  AUDIO (msg)  |  VIDEO (msg)  |  AUDIO (kbps)   |  VIDEO (kbps)   |     CPU (%) 
===========================================================================================
Video application stopped.

=== Global bandwidth statistics ===
Audio sent:       937.22 kbps
Audio received:   898.27 kbps
Video sent:       6708.33 kbps
Video received:   6244.18 kbps
Total time:       14.3 s
=====================================
Program terminated.
QObject::killTimer: Timers cannot be stopped from another thread
QObject::~QObject: Timers cannot be stopped from another thread
\end{lstlisting}
\vspace{\baselineskip}

\newpage

La imagen de la Figura \ref{fig:minimal_video_5percent} es una captura del vídeo recibido en esta prueba.
\begin{center}
  \includegraphics[width = 0.7\textwidth]{images/VideoRecibido7.1.png}
  \captionof{figure}{Vídeo recibido en \textit{Minimal\_Video} con una pérdida del 5\% de paquetes.}
  \label{fig:minimal_video_5percent}
\end{center}
\newpage


Ahora, se ejecutará el módulo \textit{Minimal\_Video\_FPS} con una pérdida del 5\% de paquetes. El comando usado ha sido el siguiente:

\begin{lstlisting}[language=bash, basicstyle=\ttfamily\scriptsize]
    python minimal_video_fps.py -a 192.168.0.58 --show_video --show_stats -z 12
\end{lstlisting}
Donde \verb|-a| es la dirección IP destino, \verb|--show_video| es para que se active la transmisión de vídeo, \verb|--show_stats| es para que se muestre la información de estadísticas y \verb|-z| es el numero de fotogramas por segundo (FPS) que se desea enviar. En este caso, se ha configurado para mostrar el vídeo a 12 FPS.
\vspace{\baselineskip}

\begin{lstlisting}[language=bash,basicstyle=\ttfamily\tiny]
         |  AUDIO (msg)  |  VIDEO (msg)  |  AUDIO (kbps)   |  VIDEO (kbps)   |     CPU (%) 
   Cycle |  Sent  Recv   |  Sent  Recv   |   Sent   Recv   |   Sent   Recv   | Program System
================================================================================================
       1 |   25    25    |  165   157    |   814    814    |  1834   1747    |  30      0       
       2 |   39    39    |    0     0    |  1252   1252    |     0      0    |  31     80       
       3 |    7     7    |   17    17    |   208    208    |   173    173    |  30     28       
       4 |   22    22    |  706   697    |   705    705    |  7731   7634    |  27     77       
       5 |   35    35    | 1296  1254    |  1141   1141    | 14430  13964    |  55     72       
       6 |   24    24    |  948   898    |   784    784    | 10580  10023    |  31     77       
       7 |   28    28    |  953   915    |   913    913    | 10612  10186    |  33     80       
       8 |   29    29    |  995   942    |   949    949    | 11116  10523    |  43     75       
       9 |   26    26    | 1316  1250    |   851    851    | 14708  13969    |  31     76       
      10 |   33    33    | 1051   968    |  1050   1050    | 11415  10515    |  38     75       
      11 |   37    37    | 1088  1034    |  1164   1164    | 11691  11110    |  59     80       
      12 |   25    25    |  803   791    |   789    789    |  8655   8526    |  37     81       
      13 |   39    39    |  703   638    |  1269   1269    |  7813   7092    |  36     75       
      14 |    0     0    |  165   163    |     0      0    |  1843   1821    |   2     45       
   Cycle |  Sent  Recv   |  Sent  Recv   |   Sent   Recv   |   Sent   Recv   | Program System
         |  AUDIO (msg)  |  VIDEO (msg)  |  AUDIO (kbps)   |  VIDEO (kbps)   |     CPU (%) 
===========================================================================================
Video application stopped.

=== Global bandwidth statistics ===
Audio sent:       833.44 kbps
Audio received:   833.44 kbps
Video sent:       7869.79 kbps
Video received:   7498.43 kbps
Total time:       14.5 s
=====================================

=== FPS Statistics ===
Target FPS:       12.0
Average real FPS: 5.7
FPS efficiency:   47.3%
======================
Program terminated.
QObject::killTimer: Timers cannot be stopped from another thread
QObject::~QObject: Timers cannot be stopped from another thread
\end{lstlisting}
\vspace{\baselineskip}

\newpage
La imagen de la Figura \ref{fig:minimal_video_fps_5percent} es una captura del vídeo recibido en esta prueba.
\begin{center}
  \includegraphics[width = 0.7\textwidth]{images/VideoRecibido7.2.png}
  \captionof{figure}{Vídeo recibido en \textit{Minimal\_Video\_FPS} con una pérdida del 5\% de paquetes.}
  \label{fig:minimal_video_fps_5percent}
\end{center}
\newpage


Ahora, se procederá a ejecutar el módulo \textit{Minimal\_Video\_Resolution} con una pérdida del 5\% de paquetes. El comando usado ha sido el siguiente:

\begin{lstlisting}[language=bash,basicstyle=\ttfamily\scriptsize]
python minimal_video_resolution.py -a 192.168.0.58 --show_video --show_stats -z 12 \\
-w 350 -g 250
\end{lstlisting}
Donde \verb|-a| indica la dirección IP destino, \verb|--show_video| indica que se active la transmisión de vídeo, \verb|--show_stats| indica que se muestren las estadísticas de la transmisión, \verb|-z| indica el número de fotogramas por segundo (FPS), \verb|-w| indica el ancho del vídeo y \verb|-g| indica la altura del vídeo.
\vspace{\baselineskip}

\begin{lstlisting}[language=bash,basicstyle=\ttfamily\tiny]
         |  AUDIO (msg)  |  VIDEO (msg)  |  AUDIO (kbps)   |  VIDEO (kbps)   |     CPU (%) 
   Cycle |  Sent  Recv   |  Sent  Recv   |   Sent   Recv   |   Sent   Recv   | Program System
================================================================================================
       1 |   22    22    |  188   180    |   719    719    |  2100   2010    |  30      0       
       2 |   34    34    |    0     0    |  1107   1107    |     0      0    |  35     76       
       3 |   30    30    |    0     0    |   969    969    |     0      0    |  32     78       
       4 |   18    18    |  309   309    |   587    587    |  3442   3442    |  21     81       
       5 |   20    20    |  730   719    |   654    654    |  8152   8032    |  20     75       
       6 |   28    28    | 1307  1264    |   909    909    | 14485  14008    |  34     76       
       7 |   23    23    |  511   505    |   745    745    |  5657   5593    |  15     77       
       8 |   27    27    |  813   803    |   881    881    |  9060   8946    |  35     72       
       9 |   29    29    | 1010   998    |   949    949    | 11286  11151    |  19     76       
      10 |   35    35    |  604   594    |  1118   1118    |  6590   6483    |  43     75       
      11 |   37    37    | 1082  1038    |  1201   1201    | 12000  11511    |  30     73       
      12 |   32    32    | 1101  1059    |  1033   1033    | 12136  11673    |  34     73       
      13 |   33    33    |  933   890    |  1077   1077    | 10396   9917    |  32     75       
      14 |   20    20    |  808   788    |   653    653    |  9015   8794    |  24     69       
   Cycle |  Sent  Recv   |  Sent  Recv   |   Sent   Recv   |   Sent   Recv   | Program System
         |  AUDIO (msg)  |  VIDEO (msg)  |  AUDIO (kbps)   |  VIDEO (kbps)   |     CPU (%) 
===========================================================================================
Video application stopped.

=== Global bandwidth statistics ===
Audio sent:       877.27 kbps
Audio received:   877.27 kbps
Video sent:       7252.29 kbps
Video received:   7060.40 kbps
Total time:       14.5 s
=====================================

=== FPS Statistics ===
Target FPS:       12.0
Average real FPS: 5.0
FPS efficiency:   42.1%
======================

=== Resolution Statistics ===
Target resolution: 350x250
Actual resolution: 352x288
Average rescaling time: 5.30 ms
Performance impact:     6.4%
=============================

=== Camera compatible resolutions ===
  1. 320x240
  2. 352x288 * SELECTED
  3. 640x360
  4. 640x480
  5. 800x600
  6. 1024x768
  7. 1280x720
  8. 1280x1024
  9. 1366x768
  10. 1600x900
  11. 1920x1080
  12. 2560x1440
  13. 3840x2160

Camera device: /dev/video0
======================================
Program terminated.
QObject::killTimer: Timers cannot be stopped from another thread
QObject::~QObject: Timers cannot be stopped from another thread
\end{lstlisting}

\newpage
La imagen de la Figura \ref{fig:minimal_video_resolution_5percent} es una captura del vídeo recibido en esta prueba.
\begin{center}
  \includegraphics[width = 0.7\textwidth]{images/VideoRecibido7.3.png}
  \captionof{figure}{Vídeo recibido en \textit{Minimal\_Video\_Resolution} con una pérdida del 5\% de paquetes.}
  \label{fig:minimal_video_resolution_5percent}
\end{center}

\newpage


Ahora, se procederá a analizar los resultados obtenidos en las pruebas realizadas con una pérdida del 5\% de paquetes.
\vspace{\baselineskip}

\textbf{Análisis de \textit{Minimal\_Video} con 5\% de pérdida de paquetes:}
\vspace{\baselineskip}

Al observar \textit{Minimal\_Video} bajo una pérdida del 5\% de paquetes, notamos un impacto directo en la cantidad de datos recibidos, como era de esperar. Las estadísticas globales muestran que el módulo envía audio a 937.22 kbps (recibiendo 898.27 kbps, lo que implica una pérdida efectiva en audio del 4.2\%) y vídeo a 6708.33 kbps (recibiendo 6244.18 kbps, una pérdida efectiva en vídeo del 6.9\%). Estos valores están alineados con la tasa de pérdida, aunque el vídeo parece ligeramente más afectado.
\vspace{\baselineskip}

El análisis por ciclos muestra fluctuaciones en la recepción, por ejemplo, en los ciclos 7 y 8, donde la recepción de audio y vídeo es notablemente inferior a la enviada. Esta variabilidad es característica de la pérdida de paquetes, que afecta de manera aleatoria pero continua a la transmisión. La experiencia del usuario se manifiesta en artefactos visuales (como bloques o congelaciones momentáneas cuando se pierden partes clave de un fotograma) y pequeños cortes en el audio.

\vspace{\baselineskip}

\textbf{Análisis de \textit{Minimal\_Video\_FPS} con 5\% de pérdida de paquetes:}
\vspace{\baselineskip}

Este módulo muestra una buena capacidad para mantener la integridad del audio frente a la pérdida de paquetes. Las estadísticas globales indican una transmisión de audio de 833.44 kbps (recibiendo el 100\% de este flujo), lo que es un resultado excelente para el audio. Para el vídeo, se enviaron 7869.79 kbps y se recibieron 7498.43 kbps, lo que representa una pérdida efectiva en vídeo del 4.7\%, muy cercana a la tasa de pérdida.
\vspace{\baselineskip}

Las ``Estadísticas de FPS'' muestran que el módulo logra 5.7 FPS reales frente al objetivo de 12 FPS, alcanzando una eficiencia del 47.3\%. Esta cifra indica que, aunque el módulo parece recibir bien el audio y la mayoría de los datos de vídeo que llegan, la pérdida de paquetes impacta la capacidad de mantener la tasa de fotogramas objetivo. El sistema probablemente reduce la tasa de envío o descarta fotogramas que no pueden ser reconstruidos completamente debido a la pérdida de paquetes para intentar mantener una calidad visual aceptable en los fotogramas que sí se muestran.

\vspace{\baselineskip}

\textbf{Análisis de \textit{Minimal\_Video\_Resolution} con 5\% de pérdida de paquetes:}
\vspace{\baselineskip}

Los datos globales de este módulo en esta prueba muestran una transmisión de audio de 877.27 kbps (recibiendo el 100\% de este flujo), lo que es, de nuevo, un resultado perfecto para el audio. Para el vídeo, se enviaron 7252.29 kbps y se recibieron 7060.40 kbps, lo que implica una pérdida efectiva en vídeo del 2.6\%. Este es el mejor resultado en términos de porcentaje de vídeo recibido entre los tres módulos bajo esta condición de pérdida.
\vspace{\baselineskip}

Las ``Estadísticas de FPS'' indican que el módulo alcanza 5.0 FPS reales frente al objetivo de 12 FPS, resultando en una eficiencia del 42.1\%. Aunque la pérdida de datos de vídeo es la más baja, la eficiencia de FPS es ligeramente inferior a la de \textit{Minimal\_Video\_FPS}. Esto podría sugerir que el módulo es más conservador al intentar reconstruir o mostrar fotogramas si falta alguna parte, priorizando la integridad del fotograma sobre la tasa. El ``Tiempo de reescalado promedio'' es de 5.30 ms, con un impacto en rendimiento del 6.4\%, valores razonables que indican que la limitación principal sigue siendo la pérdida de paquetes.

\vspace{\baselineskip}

\textbf{Conclusiones para pérdida de paquetes del 5\%:}

La prueba con una pérdida de paquetes moderada del 5\% revela cómo los diferentes módulos gestionan esta condición adversa:

\begin{itemize}
\item \textbf{Impacto directo en la recepción de vídeo:} \textit{Minimal\_Video} muestra una pérdida de datos tanto en audio (4.2\%) como en vídeo (6.9\%), en línea con lo esperado. Sin embargo, \textit{Minimal\_Video\_FPS} y \textit{Minimal\_Video\_Resolution} logran una recepción de audio del 100\% de lo enviado, y pérdidas de vídeo del 4.7\% y 2.6\% respectivamente. Esto sugiere que los mecanismos de control de FPS o la gestión de la transmisión en estos módulos podrían estar priorizando mejor el flujo de audio.
\item \textbf{Degradación de la eficiencia de FPS:} La pérdida de paquetes afecta significativamente la capacidad de alcanzar los 12 FPS objetivo. \textit{Minimal\_Video\_FPS} alcanza un 47.3\% de eficiencia (5.7 FPS reales) y \textit{Minimal\_Video\_Resolution} un 42.1\% (5.0 FPS reales). Aunque se transmiten muchos datos, la fluidez visual se ve comprometida.
\item \textbf{Variabilidad en la calidad de la transmisión:} El análisis por ciclos en \textit{Minimal\_Video} (ciclos 7 y 8) y la naturaleza de la pérdida de paquetes implican que la experiencia del usuario será inconsistente, con artefactos visuales e interrupciones de audio que aparecen de forma intermitente.
\item \textbf{Mecanismos de adaptación:} Los módulos con control de FPS parecen adaptarse a las pérdidas reduciendo la tasa de fotogramas efectiva para intentar asegurar que los fotogramas que se envían y reciben estén lo más completos posible, especialmente \textit{Minimal\_Video\_Resolution} que logra la menor tasa de pérdida de vídeo.
\end{itemize}

Estos resultados indican que una pérdida de paquetes del 5\% es un pequeño obstáculo para la comunicación en tiempo real. Aunque los módulos con control de FPS aguantan en la transmisión del audio y una pérdida de vídeo contenida, la fluidez se ve considerablemente afectada.
\newpage

\textbf{Pruebas frente a pérdida de paquetes del 25\%:}
\vspace{\baselineskip}

Ahora, realizaremos las pruebas con una pérdida del 25\% de paquetes. El comando usado en \textit{Minimal\_Video} ha sido el siguiente:

\begin{lstlisting}[language=bash]
python minimal_video.py -a 192.168.0.58 --show_video --show_stats
\end{lstlisting}
Donde \verb|-a| es la dirección IP del dispositivo con el que se va a comunicar el módulo, \verb|--show_video| es la opción para mostrar el vídeo en tiempo real y \verb|--show_stats| es la opción para mostrar las estadísticas de la red.
\vspace{\baselineskip}

\begin{lstlisting}[language=bash,basicstyle=\ttfamily\tiny]
         |  AUDIO (msg)  |  VIDEO (msg)  |  AUDIO (kbps)   |  VIDEO (kbps)   |     CPU (%) 
   Cycle |  Sent  Recv   |  Sent  Recv   |   Sent   Recv   |   Sent   Recv   | Program System
================================================================================================
       1 |   31     0    |  165    92    |  1003      0    |  1824   1018    |  36     75       
       2 |   40     3    |    0     0    |  1285     96    |     0      0    |  47     76       
       3 |   37    18    |  158    35    |  1203    585    |  1756    387    |  51     77       
       4 |   42    21    |  293     0    |  1355    677    |  3228      0    |  42     68       
       5 |   35    16    |  640   417    |  1143    522    |  7137   4654    |  37     68       
       6 |   31    17    |  477   329    |  1001    549    |  5258   3627    |  34     67       
       7 |   37    11    |  466   326    |  1202    357    |  5170   3617    |  51     71       
       8 |   27    16    |  372   262    |   877    520    |  4129   2907    |  26     69       
       9 |   23    20    |  425   314    |   738    642    |  4658   3444    |  35     72       
      10 |   37    13    |  499   305    |  1209    424    |  5568   3403    |  38     72       
      11 |   36    19    |  497   327    |  1177    621    |  5549   3651    |  33     72       
      12 |   26    20    |  483   356    |   850    654    |  5395   3979    |  32     69       
      13 |    9     6    |  303   212    |   288    192    |  3311   2317    |  14     42       
   Cycle |  Sent  Recv   |  Sent  Recv   |   Sent   Recv   |   Sent   Recv   | Program System
         |  AUDIO (msg)  |  VIDEO (msg)  |  AUDIO (kbps)   |  VIDEO (kbps)   |     CPU (%) 
===========================================================================================
Video application stopped.

=== Global bandwidth statistics ===
Audio sent:       1002.06 kbps
Audio received:   438.86 kbps
Video sent:       3977.00 kbps
Video received:   2476.75 kbps
Total time:       13.4 s
=====================================
Program terminated.
QObject::killTimer: Timers cannot be stopped from another thread
QObject::~QObject: Timers cannot be stopped from another thread
\end{lstlisting}
\vspace{\baselineskip}

\newpage
La imagen de la Figura \ref{fig:minimal_video_25percent} es una captura del vídeo recibido en esta prueba.
\begin{center}
  \includegraphics[width = 0.7\textwidth]{images/VideoRecibido8.1.png}
  \captionof{figure}{Vídeo recibido en \textit{Minimal\_Video} con una pérdida del 25\% de paquetes.}
  \label{fig:minimal_video_25percent}
\end{center}

\newpage


Ahora, se ejecutará el módulo \textit{Minimal\_Video\_FPS} con una pérdida del 25\% de paquetes. El comando usado ha sido el siguiente:

\begin{lstlisting}[language=bash, basicstyle=\ttfamily\scriptsize]
    python minimal_video_fps.py -a 192.168.0.58 --show_video --show_stats -z 12
\end{lstlisting}
Donde \verb|-a| es la dirección IP destino, \verb|--show_video| es para que se active la transmisión de vídeo, \verb|--show_stats| es para que se muestre la información de estadísticas y \verb|-z| es el numero de fotogramas por segundo (FPS) que se desea enviar. En este caso, se ha configurado para mostrar el vídeo a 12 FPS.
\vspace{\baselineskip}

\begin{lstlisting}[language=bash,basicstyle=\ttfamily\tiny]
         |  AUDIO (msg)  |  VIDEO (msg)  |  AUDIO (kbps)   |  VIDEO (kbps)   |     CPU (%) 
   Cycle |  Sent  Recv   |  Sent  Recv   |   Sent   Recv   |   Sent   Recv   | Program System
================================================================================================
       1 |   21    12    |  160   127    |   388    222    |  1012    802    |  42      0       
       2 |   21    21    |    5     4    |   591    591    |    46     38    |  10     84       
       3 |   38    31    |   66    65    |  1173    957    |   696    686    |  30     78       
       4 |   28    19    |  625   606    |   903    613    |  6885   6676    |  22     76       
       5 |   35    25    |  528   384    |  1133    809    |  5840   4247    |  37     70       
       6 |   24    18    |  431   319    |   784    588    |  4811   3561    |  36     71       
       7 |   31    13    |  347   254    |  1002    420    |  3832   2804    |  36     73       
       8 |   34    12    |  444   305    |  1094    386    |  4879   3350    |  50     75       
       9 |   35    19    |  492   353    |  1141    619    |  5479   3929    |  28     74       
      10 |   33     8    |  431   224    |  1080    261    |  4818   2503    |  25     71       
      11 |   38     5    |  466   351    |  1242    163    |  5199   3917    |  43     74       
      12 |   32     0    |  647   572    |  1043      0    |  7200   6364    |  29     73       
      13 |   13     0    |  460   374    |   424      0    |  5128   4168    |  10     65       
   Cycle |  Sent  Recv   |  Sent  Recv   |   Sent   Recv   |   Sent   Recv   | Program System
         |  AUDIO (msg)  |  VIDEO (msg)  |  AUDIO (kbps)   |  VIDEO (kbps)   |     CPU (%) 
===========================================================================================
Video application stopped.

=== Global bandwidth statistics ===
Audio sent:       877.05 kbps
Audio received:   419.06 kbps
Video sent:       3988.65 kbps
Video received:   3078.19 kbps
Total time:       14.3 s
=====================================

=== FPS Statistics ===
Target FPS:       12.0
Average real FPS: 3.0
FPS efficiency:   25.2%
======================
Program terminated.
QObject::killTimer: Timers cannot be stopped from another thread
QObject::~QObject: Timers cannot be stopped from another thread
\end{lstlisting}
\vspace{\baselineskip}

\newpage
La imagen de la Figura \ref{fig:minimal_video_fps_25percent} es una captura del vídeo recibido en esta prueba.
\begin{center}
  \includegraphics[width = 0.7\textwidth]{images/VideoRecibido8.2.png}
  \captionof{figure}{Vídeo recibido en \textit{Minimal\_Video\_FPS} con una pérdida del 25\% de paquetes.}
  \label{fig:minimal_video_fps_25percent}
\end{center}

\newpage


Ahora, se procederá a ejecutar el módulo \textit{Minimal\_Video\_Resolution} con una pérdida del 25\% de paquetes. El comando usado ha sido el siguiente:

\begin{lstlisting}[language=bash,basicstyle=\ttfamily\scriptsize]
python minimal_video_resolution.py -a 192.168.0.58 --show_video --show_stats -z 12 \\
-w 350 -g 250
\end{lstlisting}
Donde \verb|-a| indica la dirección IP destino, \verb|--show_video| indica que se active la transmisión de vídeo, \verb|--show_stats| indica que se muestren las estadísticas de la transmisión, \verb|-z| indica el número de fotogramas por segundo (FPS), \verb|-w| indica el ancho del vídeo y \verb|-g| indica la altura del vídeo.
\vspace{\baselineskip}

\begin{lstlisting}[language=bash,basicstyle=\ttfamily\tiny]
         |  AUDIO (msg)  |  VIDEO (msg)  |  AUDIO (kbps)   |  VIDEO (kbps)   |     CPU (%) 
   Cycle |  Sent  Recv   |  Sent  Recv   |   Sent   Recv   |   Sent   Recv   | Program System
================================================================================================
       1 |   23    20    |  188   187    |   751    653    |  2097   2086    |  22      0       
       2 |   37    19    |    0     0    |  1210    621    |     0      0    |  34     80       
       3 |   22     8    |  259   259    |   711    258    |  2861   2856    |  33     79       
       4 |   27    16    |  601   521    |   858    508    |  6522   5653    |  23     76       
       5 |   32    25    |  409   320    |  1047    818    |  4570   3574    |  33     72       
       6 |   37    12    |  592   415    |  1206    391    |  6591   4623    |  39     71       
       7 |   36    18    |  529   373    |  1169    584    |  5867   4137    |  41     71       
       8 |   34    20    |  528   390    |  1098    646    |  5824   4300    |  41     71       
       9 |   33    20    |  532   377    |  1059    641    |  5827   4132    |  43     71       
      10 |   38    23    |  487   345    |  1241    751    |  5433   3845    |  37     77       
      11 |   39    12    |  513   336    |  1275    392    |  5727   3756    |  36     72       
      12 |   33     8    |  440   265    |  1078    261    |  4906   2954    |  32     72       
      13 |    0     0    |  178   130    |     0      0    |  1988   1451    |   3     46       
   Cycle |  Sent  Recv   |  Sent  Recv   |   Sent   Recv   |   Sent   Recv   | Program System
         |  AUDIO (msg)  |  VIDEO (msg)  |  AUDIO (kbps)   |  VIDEO (kbps)   |     CPU (%) 
===========================================================================================
Video application stopped.

=== Global bandwidth statistics ===
Audio sent:       948.40 kbps
Audio received:   487.54 kbps
Video sent:       4352.12 kbps
Video received:   3244.00 kbps
Total time:       13.5 s
=====================================

=== FPS Statistics ===
Target FPS:       12.0
Average real FPS: 2.1
FPS efficiency:   17.3%
======================

=== Resolution Statistics ===
Target resolution: 350x250
Actual resolution: 352x288
Average rescaling time: 3.98 ms
Performance impact:     4.8%
=============================

=== Camera compatible resolutions ===
  1. 320x240
  2. 352x288 * SELECTED
  3. 640x360
  4. 640x480
  5. 800x600
  6. 1024x768
  7. 1280x720
  8. 1280x1024
  9. 1366x768
  10. 1600x900
  11. 1920x1080
  12. 2560x1440
  13. 3840x2160

Camera device: /dev/video0
======================================
Program terminated.
QObject::killTimer: Timers cannot be stopped from another thread
QObject::~QObject: Timers cannot be stopped from another thread
\end{lstlisting}

\newpage

La imagen de la Figura \ref{fig:minimal_video_resolution_25percent} es una captura del vídeo recibido en esta prueba.
\begin{center}
  \includegraphics[width = 0.7\textwidth]{images/VideoRecibido8.3.png}
  \captionof{figure}{Vídeo recibido en \textit{Minimal\_Video\_Resolution} con una pérdida del 25\% de paquetes.}
  \label{fig:minimal_video_resolution_25percent}
\end{center}

\newpage

Ahora, se procederá a analizar los resultados obtenidos en las pruebas realizadas con una pérdida del 25\% de paquetes.
\vspace{\baselineskip}

\textbf{Análisis de \textit{Minimal\_Video} con 25\% de pérdida de paquetes:}
\vspace{\baselineskip}

El impacto de una pérdida de paquetes del 25\% en \textit{Minimal\_Video} es severo y hace que la comunicación sea muy deficiente. Las estadísticas globales revelan que el módulo envía audio a 1002.06 kbps pero recibe solo 438.86 kbps (una pérdida efectiva de audio del 56.2\%), mientras que para vídeo envía 3977.00 kbps y recibe 2476.75 kbps (una pérdida efectiva de vídeo del 37.7\%). La pérdida real observada, especialmente en el audio, es significativamente mayor que el 25\%, lo que sugiere que la pérdida de ciertos paquetes tiene un efecto en cascada, comprometiendo la recepción de otros paquetes dependientes o fragmentos de mensajes.
\vspace{\baselineskip}

El análisis por ciclos muestra un comportamiento errático. Por ejemplo, en el ciclo 1, la recepción de audio es nula, y en el ciclo 4, la recepción de vídeo es nula. A lo largo de la prueba, la cantidad de mensajes de audio y vídeo recibidos es baja en comparación con los enviados. Esta inconsistencia provoca una experiencia de usuario fragmentada, con constantes y largas interrupciones tanto en el audio como en el vídeo. La comunicación se vuelve prácticamente ininteligible y la visualización es realmente pobre.

\vspace{\baselineskip}

\textbf{Análisis de \textit{Minimal\_Video\_FPS} con 25\% de pérdida de paquetes:}
\vspace{\baselineskip}

Este módulo sufre igualmente un impacto devastador bajo una pérdida de paquetes tan elevada. Las estadísticas globales muestran una transmisión de audio de 877.05 kbps (recibiendo 419.06 kbps, una pérdida efectiva de audio del 52.2\%) y de vídeo de 3988.65 kbps (recibiendo 3078.19 kbps, una pérdida efectiva de vídeo del 22.8\%). Es interesante notar que, en este caso, la pérdida efectiva de vídeo está más cerca del 25\%, mientras que el audio sigue sufriendo una pérdida mucho mayor.
\vspace{\baselineskip}

Las ``Estadísticas de FPS'' muestran un dato crítico: el módulo solo alcanza 3.0 FPS reales frente al objetivo de 12 FPS, resultando en una eficiencia del 25.2\%. Esta tasa de fotogramas es extremadamente baja para una comunicación visual efectiva. El algoritmo de control de FPS, al detectar las pérdidas masivas, reduce drásticamente la frecuencia de envío o no puede reconstruir suficientes fotogramas. El resultado es una experiencia visual severamente degradada, con actualizaciones de imagen tan esporádicas que la percepción de movimiento es casi inexistente. Los últimos ciclos (12 y 13) muestran una recepción de audio nula.

\vspace{\baselineskip}

\textbf{Análisis de \textit{Minimal\_Video\_Resolution} con 25\% de pérdida de paquetes:}
\vspace{\baselineskip}

Este módulo, que combina control de FPS con resolución adaptada, también presenta un rendimiento muy pobre bajo condiciones de pérdida del 25\%. Los datos globales indican una transmisión de audio de 948.40 kbps (recibiendo 487.54 kbps, una pérdida efectiva de audio del 48.6\%) y de vídeo de 4352.12 kbps (recibiendo 3244.00 kbps, una pérdida efectiva de vídeo del 25.5\%). De nuevo, el audio es el más afectado.
\vspace{\baselineskip}

Las ``Estadísticas de FPS'' revelan un resultado crítico: apenas 2.1 FPS reales frente al objetivo de 12 FPS, con una eficiencia catastrófica del 17.3\%. En términos prácticos, esto significa una actualización de imagen cada casi medio segundo, lo que hace que la comunicación visual sea completamente inviable. El ``Tiempo de reescalado promedio'' es de 3.98 ms, con un impacto en rendimiento del 4.8\%; estos valores son bajos porque hay muy pocos fotogramas que procesar.
\vspace{\baselineskip}

\textbf{Conclusiones para pérdida de paquetes del 25\%:}

Una pérdida de paquetes del 25\% muestra un escenario extremo que degrada severamente la calidad de la comunicación, haciendo que los tres módulos sean prácticamente inutilizables para una videoconferencia:

\begin{itemize}
\item \textbf{Pérdida masiva en audio:} Los tres módulos experimentan pérdidas de audio efectivas entre el 48.6\% y el 56.2\%, muy superiores al 25\% de pérdida inducida. Esto hace que el audio sea intermitente e ininteligible.
\item \textbf{Tasas de FPS inutilizables:} Con FPS reales entre 2.1 y 3.0, la parte visual de la comunicación es una sucesión de imágenes estáticas con muy poca conexión entre ellas. La eficiencia de FPS oscila en niveles de entre 17.3\% y 25.2\%.
\item \textbf{Comunicación completamente fragmentada:} El análisis por ciclos muestra que la recepción de datos es errática y muy baja en muchos momentos, lo que se traduce en una experiencia de usuario con constantes cortes, congelaciones y pérdida de información.
\item \textbf{Impacto en la recepción de vídeo:} La pérdida efectiva de vídeo se sitúa entre el 22.8\% y el 37.7\%, lo que, combinado con los bajos FPS, resulta en una calidad visual pésima.
\item \textbf{Ineficacia de las funcionalidades avanzadas:} Las características como el control de FPS o el reescalado de resolución no pueden compensar un nivel de pérdida de paquetes tan alto. De hecho, \textit{Minimal\_Video\_Resolution} presenta la peor eficiencia de FPS.
\end{itemize}

Estos resultados muestran una pérdida de paquetes del 25\% que sobrepasa la capacidad de los módulos para ofrecer una comunicación mínimamente funcional. En estas condiciones de pérdida, se requerirían protocolos de transmisión mucho más robustos y tolerantes a errores o cambiar a modos de comunicación que demanden mucho menos ancho de banda y sean menos sensibles a la pérdida (como solo audio con códecs muy robustos). Para la videoconferencia tal como está implementada en estos módulos, este nivel de pérdida es insostenible.
\newpage

\textbf{Pruebas frente a pérdida de paquetes del 50\%}
\vspace{\baselineskip}

Finalmente, realizaremos la última prueba por una pérdida del 50\% de paquetes. El comando usado en \textit{Minimal\_Video} ha sido el siguiente:

\begin{lstlisting}[language=bash]
python minimal_video.py -a 192.168.0.58 --show_video --show_stats
\end{lstlisting}
Donde \verb|-a| es la dirección IP del dispositivo con el que se va a comunicar el módulo, \verb|--show_video| es la opción para mostrar el vídeo en tiempo real y \verb|--show_stats| es la opción para mostrar las estadísticas de la red.
\vspace{\baselineskip}

\begin{lstlisting}[language=bash,basicstyle=\ttfamily\tiny]
         |  AUDIO (msg)  |  VIDEO (msg)  |  AUDIO (kbps)   |  VIDEO (kbps)   |     CPU (%) 
   Cycle |  Sent  Recv   |  Sent  Recv   |   Sent   Recv   |   Sent   Recv   | Program System
================================================================================================
       1 |   29     8    |  165    78    |   929    256    |  1805    854    |  31    100       
       2 |   34     5    |    0     0    |  1056    155    |     0      0    |  47     74       
       3 |   35     5    |  262   182    |  1097    156    |  2804   1949    |  31     75       
       4 |   29     6    |  381   202    |   949    196    |  4260   2256    |  28     73       
       5 |   37     7    |  427   203    |  1210    228    |  4767   2267    |  25     70       
       6 |   39     4    |  412   219    |  1261    129    |  4547   2418    |  35     70       
       7 |   34     3    |  375   210    |  1107     97    |  4171   2334    |  39     67       
       8 |   29     7    |  383   215    |   940    226    |  4239   2377    |  35     67       
       9 |   35     4    |  416   180    |  1145    130    |  4645   2012    |  37     71       
      10 |   35     5    |  353   166    |  1145    163    |  3944   1854    |  34     72       
      11 |   25     8    |  393   193    |   818    261    |  4391   2158    |  35     66       
      12 |   11     4    |  223   114    |   360    130    |  2491   1275    |  16     42       
   Cycle |  Sent  Recv   |  Sent  Recv   |   Sent   Recv   |   Sent   Recv   | Program System
         |  AUDIO (msg)  |  VIDEO (msg)  |  AUDIO (kbps)   |  VIDEO (kbps)   |     CPU (%) 
===========================================================================================
Video application stopped.

=== Global bandwidth statistics ===
Audio sent:       984.90 kbps
Audio received:   174.74 kbps
Video sent:       3425.69 kbps
Video received:   1773.91 kbps
Total time:       12.4 s
=====================================
Program terminated.
QObject::killTimer: Timers cannot be stopped from another thread
QObject::~QObject: Timers cannot be stopped from another thread
\end{lstlisting}

\newpage
La imagen de la Figura \ref{fig:minimal_video_50percent} es una captura del vídeo recibido en esta prueba.
\begin{center}
  \includegraphics[width = 0.7\textwidth]{images/VideoRecibido9.1.png}
  \captionof{figure}{Vídeo recibido en \textit{Minimal\_Video} con una pérdida del 50\% de paquetes.}
  \label{fig:minimal_video_50percent}
\end{center}

\newpage


Ahora, se procederá a ejecutar el módulo \textit{Minimal\_Video\_FPS} con una pérdida del 50\% de paquetes. El comando usado ha sido el siguiente:

\begin{lstlisting}[language=bash, basicstyle=\ttfamily\scriptsize]
    python minimal_video_fps.py -a 192.168.0.58 --show_video --show_stats -z 12
\end{lstlisting}
Donde \verb|-a| es la dirección IP destino, \verb|--show_video| es para que se active la transmisión de vídeo, \verb|--show_stats| es para que se muestre la información de estadísticas y \verb|-z| es el numero de fotogramas por segundo (FPS) que se desea enviar. En este caso, se ha configurado para mostrar el vídeo a 12 FPS.
\vspace{\baselineskip}

\begin{lstlisting}[language=bash,basicstyle=\ttfamily\tiny]
         |  AUDIO (msg)  |  VIDEO (msg)  |  AUDIO (kbps)   |  VIDEO (kbps)   |     CPU (%) 
   Cycle |  Sent  Recv   |  Sent  Recv   |   Sent   Recv   |   Sent   Recv   | Program System
================================================================================================
       1 |   31     5    |  165    84    |  1007    162    |  1830    932    |  27      0       
       2 |   40     1    |    0     0    |  1304     32    |     0      0    |  46     75       
       3 |   35     3    |  148   125    |  1036     88    |  1498   1265    |  35     75       
       4 |   29     8    |  350   168    |   948    261    |  3904   1872    |  27     69       
       5 |   37     7    |  374   189    |  1190    225    |  4109   2074    |  45     65       
       6 |   37     7    |  338   155    |  1210    228    |  3774   1728    |  38     71       
       7 |   37     8    |  372   197    |  1197    258    |  4111   2179    |  45     74       
       8 |   38     7    |  423   207    |  1239    228    |  4707   2304    |  41     67       
       9 |   35     3    |  367   156    |  1145     98    |  4101   1741    |  43     67       
      10 |   30     8    |  324   181    |   978    261    |  3609   2018    |  48     66       
      11 |   36     0    |  342    30    |  1177      0    |  3818    335    |  53     72       
      12 |   24     8    |  396   216    |   782    260    |  4407   2406    |  37     69       
   Cycle |  Sent  Recv   |  Sent  Recv   |   Sent   Recv   |   Sent   Recv   | Program System
         |  AUDIO (msg)  |  VIDEO (msg)  |  AUDIO (kbps)   |  VIDEO (kbps)   |     CPU (%) 
===========================================================================================
Video application stopped.

=== Global bandwidth statistics ===
Audio sent:       1070.38 kbps
Audio received:   170.11 kbps
Video sent:       3215.71 kbps
Video received:   1526.27 kbps
Total time:       12.5 s
=====================================

=== FPS Statistics ===
Target FPS:       12.0
Average real FPS: 1.7
FPS efficiency:   14.0%
======================
Program terminated.
QObject::killTimer: Timers cannot be stopped from another thread
QObject::~QObject: Timers cannot be stopped from another thread
\end{lstlisting}

\newpage
La imagen de la Figura \ref{fig:minimal_video_fps_50percent} es una captura del vídeo recibido en esta prueba.
\begin{center}
  \includegraphics[width = 0.7\textwidth]{images/VideoRecibido9.2.png}
  \captionof{figure}{Vídeo recibido en \textit{Minimal\_Video\_FPS} con una pérdida del 50\% de paquetes.}
  \label{fig:minimal_video_fps_50percent}
\end{center}

\newpage


Ahora, se procederá a ejecutar el módulo \textit{Minimal\_Video\_Resolution} con una pérdida del 50\% de paquetes. El comando usado ha sido el siguiente:
\begin{lstlisting}[language=bash,basicstyle=\ttfamily\scriptsize]
python minimal_video_resolution.py -a 192.168.0.58 --show_video --show_stats -z 12 \\
-w 350 -g 250
\end{lstlisting}
Donde \verb|-a| indica la dirección IP destino, \verb|--show_video| indica que se active la transmisión de vídeo, \verb|--show_stats| indica que se muestren las estadísticas de la transmisión, \verb|-z| indica el número de fotogramas por segundo (FPS), \verb|-w| indica el ancho del vídeo y \verb|-g| indica la altura del vídeo.
\vspace{\baselineskip}

\begin{lstlisting}[language=bash,basicstyle=\ttfamily\tiny]
         |  AUDIO (msg)  |  VIDEO (msg)  |  AUDIO (kbps)   |  VIDEO (kbps)   |     CPU (%) 
   Cycle |  Sent  Recv   |  Sent  Recv   |   Sent   Recv   |   Sent   Recv   | Program System
================================================================================================
       1 |   20     4    |  165   104    |   646    129    |  1823   1146    |  20      0       
       2 |   35     4    |   23     0    |  1140    130    |   253      0    |  48     74       
       3 |   33     9    |    0     0    |  1079    294    |     0      0    |  63     74       
       4 |   31     4    |  346   321    |  1015    130    |  3870   3592    |  38     75       
       5 |   24     5    |  310   148    |   783    163    |  3453   1648    |  44     69       
       6 |   28     4    |  401   207    |   915    130    |  4475   2308    |  45     70       
       7 |   38     3    |  392   199    |  1242     98    |  4376   2224    |  57     72       
       8 |   33     2    |  420   176    |  1057     64    |  4593   1921    |  26     67       
       9 |   31     7    |  385   193    |  1013    228    |  4292   2155    |  24     70       
      10 |   35     5    |  400   184    |  1117    159    |  4360   2005    |  16     61       
      11 |   32     5    |  431   209    |  1035    161    |  4763   2312    |  15     63       
      12 |   35     3    |  399   189    |  1142     97    |  4447   2106    |  20     64       
      13 |    9     1    |  274   139    |   293     32    |  3051   1548    |  10     58       
   Cycle |  Sent  Recv   |  Sent  Recv   |   Sent   Recv   |   Sent   Recv   | Program System
         |  AUDIO (msg)  |  VIDEO (msg)  |  AUDIO (kbps)   |  VIDEO (kbps)   |     CPU (%) 
===========================================================================================
Video application stopped.

=== Global bandwidth statistics ===
Audio sent:       931.34 kbps
Audio received:   135.82 kbps
Video sent:       3267.15 kbps
Video received:   1713.44 kbps
Total time:       13.5 s
=====================================

=== FPS Statistics ===
Target FPS:       12.0
Average real FPS: 1.5
FPS efficiency:   12.7%
======================

=== Resolution Statistics ===
Target resolution: 350x250
Actual resolution: 352x288
Average rescaling time: 2.81 ms
Performance impact:     3.4%
=============================

=== Camera compatible resolutions ===
  1. 320x240
  2. 352x288 * SELECTED
  3. 640x360
  4. 640x480
  5. 800x600
  6. 1024x768
  7. 1280x720
  8. 1280x1024
  9. 1366x768
  10. 1600x900
  11. 1920x1080
  12. 2560x1440
  13. 3840x2160

Camera device: /dev/video0
======================================
Program terminated.
QObject::killTimer: Timers cannot be stopped from another thread
QObject::~QObject: Timers cannot be stopped from another thread
\end{lstlisting}

\newpage

La imagen de la Figura \ref{fig:minimal_video_resolution_50percent} es una captura del vídeo recibido en esta prueba.
\begin{center}
  \includegraphics[width = 0.7\textwidth]{images/VideoRecibido9.3.png}
  \captionof{figure}{Vídeo recibido en \textit{Minimal\_Video\_Resolution} con una pérdida del 50\% de paquetes.}
  \label{fig:minimal_video_resolution_50percent}
\end{center}

\newpage

Ahora, se procederá a analizar los resultados obtenidos en las pruebas realizadas con una pérdida del 50\% de paquetes.
\vspace{\baselineskip}

\textbf{Análisis de \textit{Minimal\_Video} con 50\% de pérdida de paquetes:}
\vspace{\baselineskip}

El impacto de una pérdida de paquetes del 50\% en \textit{Minimal\_Video} es, como era de esperar, devastador para la comunicación. Las estadísticas globales muestran que el módulo envía audio a 984.90 kbps pero recibe solo 174.74 kbps (una pérdida efectiva de audio del 82.2\%), mientras que para vídeo envía 3425.69 kbps y recibe 1773.91 kbps (una pérdida efectiva de vídeo del 48.2\%). La pérdida real observada en el audio es drásticamente superior al 50\%, lo que indica que la pérdida de la mitad de los paquetes de audio hace que la mayor parte del flujo sea irrecuperable o ininteligible. La pérdida de vídeo está más cerca de la tasa de pérdida, pero sigue siendo masiva.
\vspace{\baselineskip}

El análisis por ciclos es muy grave, la cantidad de mensajes de audio recibidos es consistentemente una pequeña fracción de los enviados (por ejemplo, 4 de 39 en el ciclo 6, 3 de 34 en el ciclo 7). Aunque se recibe algo de vídeo, la cantidad es muy baja y la calidad es pésima. La comunicación es completamente inviable, con audio prácticamente ausente y un vídeo extremadamente fragmentado e incomprensible.

\vspace{\baselineskip}

\textbf{Análisis de \textit{Minimal\_Video\_FPS} con 50\% de pérdida de paquetes:}
\vspace{\baselineskip}

Este módulo también colapsa bajo una pérdida de paquetes del 50\%. Las estadísticas globales indican una transmisión de audio de 1070.38 kbps (recibiendo solo 170.11 kbps, una pérdida efectiva de audio del 84.1\%) y de vídeo de 3215.71 kbps (recibiendo 1526.27 kbps, una pérdida efectiva de vídeo del 52.5\%).
\vspace{\baselineskip}

Las ``Estadísticas de FPS'' muestran que el módulo apenas alcanza 1.7 FPS reales frente al objetivo de 12 FPS, resultando en una eficiencia del 14.0\%. Esto significa que se muestra menos de una imagen cada medio segundo, lo que hace imposible cualquier tipo de seguimiento visual. El control de FPS es incapaz de gestionar una pérdida tan masiva de información, y la comunicación visual es nula en la práctica. La recepción de audio también es mínima, como en el caso anterior.

\vspace{\baselineskip}

\textbf{Análisis de \textit{Minimal\_Video\_Resolution} con 50\% de pérdida de paquetes:}
\vspace{\baselineskip}

Finalmente, este módulo muestra resultados igualmente catastróficos, si no peores en el caso del audio. Los datos globales revelan una transmisión de audio de 931.34 kbps (recibiendo solo 135.82 kbps, una pérdida efectiva de audio del 85.4\%) y de vídeo de 3267.15 kbps (recibiendo 1713.44 kbps, una pérdida efectiva de vídeo del 47.6\%).
\vspace{\baselineskip}

Las ``Estadísticas de FPS'' muestran que el módulo apenas alcanza 1.5 FPS reales frente al objetivo de 12 FPS, con una eficiencia del 12.7\%. Este es el peor rendimiento de FPS de los tres módulos bajo esta condición. El ``Tiempo de reescalado promedio'' es de 2.81 ms, con un impacto en rendimiento del 3.4\%, pero estos datos son irrelevantes dado el colapso general de la comunicación. La combinación de reescalado y control de FPS no ofrece ninguna ventaja frente a una pérdida de paquetes tan extrema.
\vspace{\baselineskip}

\textbf{Conclusiones para pérdida de paquetes del 50\%:}

Una pérdida de paquetes del 50\% representa un escenario donde la comunicación audiovisual en tiempo real, con las implementaciones actuales de estos módulos, es imposible:
\vspace{\baselineskip}

\begin{itemize}
\item \textbf{Colapso de la transmisión de audio:} Los tres módulos experimentan pérdidas efectivas de audio superiores al 82\% (entre 82.2\% y 85.4\%). Esto resulta en un audio prácticamente inexistente, con solo fragmentos ininteligibles que llegan esporádicamente.
\item \textbf{FPS prácticamente inexistentes:} Con tasas de FPS reales entre 1.5 y 1.7, la componente visual de la comunicación se reduce a una presentación de imágenes extremadamente lenta y fragmentada. La eficiencia de FPS oscilan en niveles de entre el 12.7\% y el 14.0\%.
\item \textbf{Pérdida masiva de datos de vídeo:} Aunque la pérdida efectiva de vídeo (entre 47.6\% y 52.5\%) está más cerca del 50\% inducido que la del audio, sigue siendo una cantidad de información perdida tan grande que los fotogramas recibidos estarían severamente corruptos o incompletos.
\item \textbf{Inviabilidad de la comunicación:} La combinación de audio casi ausente y vídeo con FPS extremadamente bajos hace que cualquier intento de videoconferencia sea inútil. No se puede mantener una conversación ni seguir ninguna acción visual.
\end{itemize}

Estos resultados demuestran que una pérdida de paquetes del 50\% resulta en la imposibilidad de una comunicación audiovisual en tiempo real. Para una videoconferencia estándar, la comunicación es impracticable.
\newpage

% Reducir el espacio entre columnas para ganar más espacio para el texto
\setlength{\tabcolsep}{3pt} 

\begin{table} [htpb]
    \centering
    \renewcommand{\arraystretch}{1.25}
    \begin{tabularx}{\textwidth}{|>{\itshape\arraybackslash\scriptsize}p{3.2cm}|*{9}{>{\raggedright\arraybackslash\tiny}X|}}
    \hline
    & \multicolumn{3}{c|}{\textbf{\footnotesize Ancho de Banda Limitado}} & \multicolumn{3}{c|}{\textbf{\footnotesize Latencia (Delay)}} & \multicolumn{3}{c|}{\textbf{\footnotesize Pérdida de Paquetes}} \\
    \cline{2-10}
    \textbf{\footnotesize Módulo} & 
    \makecell[tc]{\textbf{\footnotesize 1 Mbps}} & \makecell[tc]{\textbf{\footnotesize 10 Mbps}} & \makecell[tc]{\textbf{\footnotesize 50 Mbps}} & 
    \makecell[tc]{\textbf{\footnotesize 0 ms}} & \makecell[tc]{\textbf{\footnotesize 100 ms}} & \makecell[tc]{\textbf{\footnotesize 250 ms}} & 
    \makecell[tc]{\textbf{\footnotesize 5\%}} & \makecell[tc]{\textbf{\footnotesize 25\%}} & \makecell[tc]{\textbf{\footnotesize 50\%}} \\
    \hline
    Minimal\_Video & 
    AT:954.02\newline AR:345.96\newline VT:1771.67\newline VR:762.21 &
    AT:860.65\newline AR:798.34\newline VT:5481.62\newline VR:4927.59 &
    AT:1000.25\newline AR:966.08\newline VT:8587.79\newline VR:7898.40 &
    AT:882.48\newline AR:799.36\newline VT:6587.96\newline VR:6353.34 &
    AT:962.34\newline AR:923.46\newline VT:11432.49\newline VR:10215.23 &
    AT:956.82\newline AR:929.90\newline VT:11274.19\newline VR:10111.51 &
    AT:937.22\newline AR:898.27\newline VT:6708.33\newline VR:6244.18 &
    AT:1002.06\newline AR:438.86\newline VT:3977.00\newline VR:2476.75 &
    AT:984.90\newline AR:174.74\newline VT:3425.69\newline VR:1773.91 \\
    \hline
    Minimal\_Video\_FPS & 
    AT:867.25\newline AR:391.50\newline VT:1862.96\newline VR:593.89\newline FPS~R:1.0\newline FPS~E:8.3\% &
    AT:980.18\newline AR:939.41\newline VT:2191.73\newline VR:1973.56\newline FPS~R:1.2\newline FPS~E:9.9\% &
    AT:976.67\newline AR:923.22\newline VT:8209.75\newline VR:7629.43\newline FPS~R:5.0\newline FPS~E:41.5\% &
    AT:955.12\newline AR:911.37\newline VT:7834.12\newline VR:7395.49\newline FPS~R:5.4\newline FPS~E:45.2\% &
    AT:936.37\newline AR:936.37\newline VT:13682.08\newline VR:13276.25\newline FPS~R:11.4\newline FPS~E:95.0\% &
    AT:945.70\newline AR:938.35\newline VT:9929.23\newline VR:9173.34\newline FPS~R:8.3\newline FPS~E:69.1\% &
    AT:833.44\newline AR:833.44\newline VT:7869.79\newline VR:7498.43\newline FPS~R:5.7\newline FPS~E:47.3\% &
    AT:877.05\newline AR:419.06\newline VT:3988.65\newline VR:3078.19\newline FPS~R:3.0\newline FPS~E:25.2\% &
    AT:1070.38\newline AR:170.11\newline VT:3215.71\newline VR:1526.27\newline FPS~R:1.7\newline FPS~E:14.0\% \\
    \hline
    Minimal\_Video\_Resolution & 
    AT:738.20\newline AR:360.10\newline VT:2311.06\newline VR:664.16\newline FPS~R:1.1\newline FPS~E:8.9\%\newline TR:6.47ms\newline IR:7.8\% &
    AT:546.99\newline AR:546.99\newline VT:3477.81\newline VR:3234.63\newline FPS~R:1.6\newline FPS~E:13.7\%\newline TR:6.47ms\newline IR:7.8\% &
    AT:904.14\newline AR:752.51\newline VT:7688.82\newline VR:6840.18\newline FPS~R:3.6\newline FPS~E:29.8\%\newline TR:5.53ms\newline IR:6.6\% &
    AT:933.82\newline AR:933.82\newline VT:7866.30\newline VR:7599.87\newline FPS~R:5.2\newline FPS~E:43.0\%\newline TR:3.06ms\newline IR:3.7\% &
    AT:896.90\newline AR:860.39\newline VT:12506.10\newline VR:11479.56\newline FPS~R:6.6\newline FPS~E:55.4\%\newline TR:5.67ms\newline IR:6.8\% &
    AT:932.17\newline AR:910.32\newline VT:10691.15\newline VR:9532.74\newline FPS~R:6.0\newline FPS~E:50.3\%\newline TR:6.43ms\newline IR:7.7\% &
    AT:877.27\newline AR:877.27\newline VT:7252.29\newline VR:7060.40\newline FPS~R:5.0\newline FPS~E:42.1\%\newline TR:5.30ms\newline IR:6.4\% &
    AT:948.40\newline AR:487.54\newline VT:4352.12\newline VR:3244.00\newline FPS~R:2.1\newline FPS~E:17.3\%\newline TR:3.98ms\newline IR:4.8\% &
    AT:931.34\newline AR:135.82\newline VT:3267.15\newline VR:1713.44\newline FPS~R:1.5\newline FPS~E:12.7\%\newline TR:2.81ms\newline IR:3.4\% \\
    \hline
    \multicolumn{10}{l}{\textit{\footnotesize AT: Audio Transmitido, AR: Audio Recibido, VT: Vídeo Transmitido, VR: Vídeo Recibido (kbps).}}\\
    \multicolumn{10}{l}{\textit{\footnotesize FPS R: FPS Real Promedio, FPS E: Eficiencia de FPS (\%). TR: Tiempo Reescalado (ms). IR: Impacto en Rendimiento (\%).}}
    \end{tabularx}
    \caption{Resumen de los experimentos realizados.}
    \label{tab:resumen_pruebas_globales}
\end{table}
\vspace{\baselineskip}

\newpage



